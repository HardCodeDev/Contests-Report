%----------------------------------------------------------------------------------------
%	PACKAGES AND OTHER DOCUMENT CONFIGURATIONS
%----------------------------------------------------------------------------------------

\documentclass[a4paper,12pt]{article}
%\usepackage[english]{babel}
\usepackage{amsmath}
\usepackage{graphicx}
\usepackage[colorinlistoftodos]{todonotes}
\usepackage{fullpage}
\usepackage{multicol,multirow}
\usepackage{tabularx}
\usepackage{ulem}
\usepackage[utf8]{inputenc}
\usepackage[russian]{babel}
\usepackage{amsmath}
\usepackage{amssymb}
\usepackage{listings}
\usepackage{titlesec}

\usepackage{longtable}
\usepackage{ltxtable}

\usepackage{listings}
\usepackage{color}
 
\definecolor{codegreen}{rgb}{0,0.6,0}
\definecolor{codegray}{rgb}{0.5,0.5,0.5}
\definecolor{codepurple}{rgb}{0.58,0,0.82}
\definecolor{backcolour}{rgb}{0.95,0.95,0.92}
 
\lstdefinestyle{mystyle}{
    %backgroundcolor=\color{backcolour},   
    commentstyle=\color{codegreen},
    keywordstyle=\color{magenta},
    numberstyle=\tiny\color{codegray},
    stringstyle=\color{codepurple},
    basicstyle=\footnotesize,
    breakatwhitespace=false,         
    breaklines=true,                 
    captionpos=b,                    
    keepspaces=true,                 
    numbers=left,                    
    numbersep=5pt,                  
    showspaces=false,                
    showstringspaces=false,
    showtabs=false,                  
    tabsize=2
}
\lstset{style=mystyle}

\usepackage{geometry}
\geometry{top=2cm}
\geometry{bottom=2cm}
\geometry{left=1.5cm}
\geometry{right=1.5cm}


\begin{document}

\begin{titlepage}

\newcommand{\HRule}{\rule{\linewidth}{0.5mm}} % Defines a new command for the horizontal lines, change thickness here

\center % Center everything on the page
 
 
 
%----------------------------------------------------------------------------------------
%	HEADING SECTIONS
%----------------------------------------------------------------------------------------

\textsc{\large Московский Авиационный Институт\\(национальный исследовательский университет)}\\[1.5cm] % Name of your university/college



%----------------------------------------------------------------------------------------
%	LOGO SECTION
%----------------------------------------------------------------------------------------

\includegraphics[width=0.25\textwidth]{mai_logo.png}\\[1cm]
% Include a department/university logo - this will require the graphicx package
 
%----------------------------------------------------------------------------------------

\vspace{40px}

\textsc{\Large Отчет по индивидуальному учебному плану}\\[0.5cm] % Major heading such as course name
%\textsc{\large Алгоритмы на графах}\\[0.5cm] % Minor heading such as course title



%----------------------------------------------------------------------------------------
%	TITLE SECTION
%----------------------------------------------------------------------------------------

\HRule \\[0.4cm]
{ \huge \bfseries Алгоритмы на графах}\\[0.4cm] % Title of your document
\HRule \\[1.5cm]



%----------------------------------------------------------------------------------------
%	AUTHOR SECTION
%----------------------------------------------------------------------------------------

\begin{minipage}{0.4\textwidth}
\begin{flushleft} \large
\emph{Студенты:}\\
%John \textsc{Smith} % Your name
Макаров Никита\\Якименко Антон
\end{flushleft}
\end{minipage}
~
\begin{minipage}{0.4\textwidth}
\begin{flushright} \large
\emph{Руководитель:} \\
%Dr. James \textsc{Smith} % Supervisor's Name
Зайцев В.Е.
\end{flushright}
\end{minipage}\\[2cm]

%----------------------------------------------------------------------------------------
%	DATE SECTION
%----------------------------------------------------------------------------------------

%{\large \today}\\[2cm] % Date, change the \today to a set date if you want to be precise

\vfill % Fill the rest of the page with whitespace

\end{titlepage}



%----------------------------------------------------------------------------------------
%	СОДЕРЖАНИЕ
%----------------------------------------------------------------------------------------
\tableofcontents
\newpage



%----------------------------------------------------------------------------------------
%	ЛИЧНЫЕ ОТЧЕТЫ
%----------------------------------------------------------------------------------------
\section{Личные отчеты}

% Мой отчет
Отчет о работе студента Макарова Н.А. по индивидуальному учебному плану в V-VI семестрах 2014-2015 учебного года.

% Таблица - часть 1
\begin{table}[ht!]
\centering
\label{my-tab1}
\begin{tabular}{|c|c|c|c|c|c|c|}
\hline

% Заголовки столбцов
№ & 
Дата & 
Контест & 
\begin{tabular}[c]{@{}c@{}}Место\\ проведения\end{tabular} & 
\begin{tabular}[c]{@{}c@{}}Кол-во\\ участников\end{tabular} & 
\begin{tabular}[c]{@{}c@{}}Решено\\ задач\end{tabular} & 
\begin{tabular}[c]{@{}c@{}}Задач на\\ участника\end{tabular} \\ \hline

% Строки таблицы
1 & 18.09.2014 & Codeforces Round 267 Div 2 & Дом & 1 & 2 & 2 \\ \hline

2 & 21.09.2014 & Codeforces Round 268 Div 2 & Дом & 1 & 2 & 2 \\ \hline

3 & 22.09.2014 & \begin{tabular}[c]{@{}c@{}}Codeforces Отборочный контест\\ СГАУ на 1/4 ACM-ICPC\end{tabular} & Дом & 1 & 3 & 3 \\ \hline

4 & 25.09.2014 & Codeforces Training S02E03 & МАИ & 3 & 3 & 1 \\ \hline

5 & 28.09.2014 & Codeforces Round 270 Div 2 & Дом & 1 & 2 & 2 \\ \hline

6 & 02.10.2014 & Codeforces Training S02E04 & МАИ & 3 & 2 & 0.66 \\ \hline

7 & 05.10.2014 & \begin{tabular}[c]{@{}c@{}}XV Открытая Всесибирская\\ Олимпиада по\\ Программированию\end{tabular} & МАИ & 3 & 1 & 0.33 \\ \hline

8 & 09.10.2014 & Codeforces Training S02E05 & МАИ & 3 & 4 & 1.33 \\ \hline

9 & 16.10.2014 & Codeforces Round 273 Div 2 & Дом & 1 & 2 & 2 \\ \hline

10 & 17.10.2014 & Codeforces Training S02E06 & МАИ & 3 & 0 & 0 \\ \hline

11 & 18.10.2014 & \begin{tabular}[c]{@{}c@{}}Codeforces Тренировка СПбГУ\\ графы и DFS\end{tabular} & Дом & 3 & 2 & 0.66 \\ \hline

12 & 19.10.2014 & OpenCup GP of SPb. Div 2 & МАИ & 3 & 1 & 0.33 \\ \hline

13 & 20.10.2014 & Codeforces Round 274 Div 2 & МАИ & 1 & 3 & 3 \\ \hline

14 & 23.10.2014 & \begin{tabular}[c]{@{}c@{}}Codeforces Самарский\\Аэрокосмический Лицей\\ тренировка №1\end{tabular} & Дом & 2 & 1 & 0.5 \\ \hline

15 & 23.10.2014 & \begin{tabular}[c]{@{}c@{}}Codeforces ACM, NEERC,\\ Восточный четвертьфинал\end{tabular} & Дом & 3 & 4 & 1.33 \\ \hline

16 & 24.10.2014 & Codeforces Round 275 Div 2 & Дом & 1 & 1 & 1 \\ \hline

17 & 25.10.2014 & \begin{tabular}[c]{@{}c@{}}Codeforces ACM, NEERC,\\ Южный четвертьфинал\end{tabular} & Дом & 3 & 3 & 1 \\ \hline

18 & 26.10.2014 & ACM-ICPC 1/4 Final & МГУ & 3 & 3 & 1 \\ \hline

19 & 30.10.2014 & Codeforces Training S02E07 & МАИ & 3 & 2 & 0.66 \\ \hline

20 & 01.11.2014 & Codeforces Crypto Cup & Дом & 3 & 9 & 3 \\ \hline

21 & 02.11.2014 & OpenCup GP of Siberia Div 2 & МАИ & 2 & 3 & 1.5 \\ \hline

22 & 06.11.2014 & Codeforces Training S02E08 & МАИ & 3 & 2 & 0.66 \\ \hline

23 & 13.11.2014 & Codeforces Training S02E09 & МАИ & 3 & 3 & 1 \\ \hline

24 & 15.11.2014 & \begin{tabular}[c]{@{}c@{}}Codeforces Олимпиада\\ школьников\\ Нижегородской области\end{tabular} & Дом & 3 & 3 & 1 \\ \hline

25 & 16.11.2014 & \begin{tabular}[c]{@{}c@{}}OpenCup GP\\ of Central Europe. Div 2\end{tabular} & МАИ & 3 & 1 & 0.33 \\ \hline

26 & 20.11.2014 & Codeforces Training S02E10 & МАИ & 3 & 3 & 1 \\ \hline

27 & 23.11.2014 & OpenCup GP of Europe Div 2 & МАИ & 3 & 5 & 1.66 \\ \hline

28 & 14.12.2014 & OpenCup GP of Peterhof Div 2 & МАИ & 3 & 1 & 0.33 \\ \hline

29 & 01.02.2015 & OpenCup GP of Japan Div 2 & МАИ & 3 & 4 & 1.33 \\ \hline

30 & 08.02.2015 & OpenCup Northern GP Div 2 & МАИ & 3 & 2 & 0.66 \\ \hline

31 & 15.02.2015 & OpenCup GP of Karelia Div 2 & МАИ & 3 & 4 & 1.33 \\ \hline

\end{tabular}
\end{table}

Продолжение таблицы.
% Таблица - часть 2
\begin{table}[ht!]
\centering
%\caption{My caption}
\label{my-tab2}
\begin{tabular}{|c|c|c|c|c|c|c|}
\hline

% Заголовки столбцов
№ & 
Дата & 
Контест & 
\begin{tabular}[c]{@{}c@{}}Место\\ проведения\end{tabular} & 
\begin{tabular}[c]{@{}c@{}}Кол-во\\ участников\end{tabular} & 
\begin{tabular}[c]{@{}c@{}}Решено\\ задач\end{tabular} & 
\begin{tabular}[c]{@{}c@{}}Задач на\\ участника\end{tabular} \\ \hline

% Строки таблицы
32 & 22.02.2015 & OpenCup GP of Udmurtia Div 2 & МАИ & 3 & 4 & 1.33 \\ \hline

33 & 01.03.2015 & OpenCup GP of China Div 2 & МАИ & 3 & 1 & 0.33 \\ \hline

34 & 08.02.2015 & OpenCup GP of Tatarstan Div 2 & МАИ & 3 & 1 & 0.33 \\ \hline

35 & 07.03.2015 & VK Cup 2015 Квалификация & Дом & 2 & 2 & 1 \\ \hline

36 & 21.03.2015 & VK Cup 2015 Раунд 1 & Дом & 2 & 2 & 1 \\ \hline

37 & 29.03.2015 & OpenCup Gp of America Div 2 & МАИ & 3 & 4 & 1.33 \\ \hline

38 & 18.04.2015 & Vekua Cup Личный этап & МФТИ-1С & 1 & 1 & 1 \\ \hline

39 & 19.04.2015 & Vekua Cup Командный этап & МФТИ-1С & 3 & 3 & 1 \\ \hline

40 & 26.04.2015 & OpenCup GP of Ural Div 2 & МАИ & 2 & 2 & 1 \\ \hline

41 & 31.05.2105 & Mail.ru RCC Квалификация & Дом & 1 & 1 & 1 \\ \hline

\end{tabular}
\end{table}

Итого: 41 контест, 47 решенных задач.
\newpage

% Отчет Антона

Отчет о работе студента Якименко А.В. по индивидуальному учебному плану в V-VI семестрах 2014-2015 учебного года.



\newpage
%----------------------------------------------------------------------------------------
%	КОМАНДНЫЕ КОНТЕСТЫ
%----------------------------------------------------------------------------------------
\section{Журнал по командным контестам}

%----------------------------------------------------------------------------------------
%
%	Codeforces Training S02E03
%
%----------------------------------------------------------------------------------------
\subsection{Codeforces Training S02E03}

\textbf{{\large Результаты}} \\
\begin{center}
\includegraphics[width=0.95\textwidth]{CT_S02E03/CT_S02E03_result.png}\\ [1cm]
\end{center}

\newpage
\textbf{{\large Задача B - Best Compression Ever}} \\
\begin{center}
\includegraphics[width=0.9\textwidth]{CT_S02E03/CT_S02E03_B.png}\\ [1cm]
\end{center}
\textbf{{\large Алгоритм}} \\
Решение довольно простое. Можно заметить, что если логарифм по основаню 2 числа n меньше или равен b, то ответ yes, иначе ответ no.
\newpage
\textbf{{\large Исходный код}} \\
\begin{lstlisting}[language=C]
#include <iostream>
#include <cmath>

using namespace std;

int main() {
    ios_base::sync_with_stdio(false);
    
    unsigned long long n;
    int b;
    cin >> n >> b;
    
    if ((int)log2((double)n) <= b) {
        cout << "yes" << endl;
    }
    else {
        cout << "no" << endl;
    }

    return 0;
}
\end{lstlisting}


\newpage
\textbf{{\large Задача E - Event Planning}} \\
\begin{center}
\includegraphics[width=0.9\textwidth]{CT_S02E03/CT_S02E03_E.png}\\ [1cm]
\end{center}
\textbf{{\large Алгоритм}} \\
{\Huge ???????????????????}
\newpage
\textbf{{\large Исходный код}} \\
\begin{lstlisting}[language=C]
#include <iostream>
#include <vector>

#define ll long long
using namespace std;

int main () {
    ll N,B,H,W,p,a;
    ll min_cost = 5000000;

    cin >> N >> B >> H >> W;

    for (ll i = 0; i < H; i++) {
        cin >> p;
            for (ll k = 0; k < W; k++) {
                cin >> a;
                if((a >= N) && (p * N <= B) && (p * N <= min_cost))
                    min_cost = p * N;
            }
    }
    if(min_cost < 5000000)
        cout << min_cost << endl;
    else cout << "stay home" << endl;


    return 0;
}
\end{lstlisting}


\newpage
\textbf{{\large Задача K - Best Cow Line}} \\
\begin{center}
\includegraphics[width=0.9\textwidth]{CT_S02E03/CT_S02E03_K.png}\\ [1cm]
\end{center}
\textbf{{\large Алгоритм}} \\
{\Huge ???????????????????}
\newpage
\textbf{{\large Исходный код}} \\
\begin{lstlisting}[language=C]
#include <iostream>
#include <cmath>
#include <vector>
#include <stack>
#include <algorithm>

using namespace std;

int main() {
    int n;
    cin >> n;
    vector<char> cows(n);
    char symb;
    
    for (int i = 0; i < n; i++) {
        cin >> symb;
        cows[i] = symb;
    }
    
    vector<char> newLine;
    int cowsInOldLine = n;
    int begin = 0;
    int end   = n - 1;
    int tempBegin = begin;
    int tempEnd   = end;
    
    while (cowsInOldLine) {
        tempBegin = begin;
        tempEnd   = end;
        if (cows[begin] == cows[end]) {
            while (cows[tempBegin] == cows[tempEnd]) {
                tempBegin++;
                tempEnd--;
                if (tempBegin > tempEnd || tempBegin == tempEnd) {
                    tempBegin = begin;
                    tempEnd = end;
                    break;
                }
            }
        }
        if (cows[tempBegin] < cows[tempEnd]) {
            newLine.push_back(cows[begin]);
            begin++;
        }
        else {
            newLine.push_back(cows[end]);
            end--;
        }
        cowsInOldLine--;
    }
    for (int i = 0; i < n; i++) {
        cout << newLine[i];
        if ((i + 1) % 80 == 0) cout << endl;
    }
    cout << endl;
    return 0;
}
\end{lstlisting}



%----------------------------------------------------------------------------------------
%
%	Codeforces Training S02E04
%
%----------------------------------------------------------------------------------------
\newpage
\subsection{Codeforces Training S02E04}

{\Huge у Антона}



%----------------------------------------------------------------------------------------
%
%	XV Открытая Всесибирская олимпиада по программированию им И.В. Поттосина
%
%----------------------------------------------------------------------------------------
\newpage
\subsection{XV Открытая Всесибирская олимпиада по программированию им И.В. Поттосина}

\textbf{{\large Результаты}} \\
\begin{center}
\includegraphics[width=0.95\textwidth]{Siberia/Siberia_result.png}\\ [1cm]
\end{center}

%\newpage
\textbf{{\large Задача 2 - Копировальный аппарат}} \\
\begin{center}
\includegraphics[width=0.9\textwidth]{Siberia/Siberia_1.png}\\ [1cm]
\includegraphics[width=0.6\textwidth]{Siberia/Siberia_2.png}\\ [1cm]
\end{center}
\textbf{{\large Алгоритм}} \\
Задача на реализацию. Нужно считать входные данные в двумерный массив символов, затем пройтись по всем элементам и запомнить наибольшие позиции, на которых находятся решетки. \\ \\
%\newpage
\textbf{{\large Исходный код}}
\begin{lstlisting}[language=C]
#include <iostream>
#include <algorithm>
#include <cmath>

#include <sstream>
#include <fstream>

#define LL  long long

using namespace std;

int main() {
    ifstream in;
    ofstream out;
    in.open("input.txt");
    out.open("output.txt");
    
    LL a, b;
    in >> a >> b;
    char pic[a + 1][b + 1];
    LL XMax = 0;
    LL YMax = 0;
    
    for (LL i = 1; i <= a; i++) {
        for (LL j = 1; j <= b; j++) {
            in >> pic[i][j];
            if (pic[i][j] == '#' && j > XMax) XMax = j;
            if (pic[i][j] == '#' && i > YMax) YMax = i;
        }
        in.get();
    }
    LL n, y, x;
    in >> n;
    for (LL i = 1; i <= n; i++) {
        in >> y >> x;
        if (y >= YMax && x >= XMax) {
            out << i << endl;
            return 0;
        }
    }
    in.close();
    out.close();
    
    return 0;
}
\end{lstlisting}



%----------------------------------------------------------------------------------------
%
%	Codeforces Training S02E05
%
%----------------------------------------------------------------------------------------
\newpage
\subsection{Codeforces Training S02E05}

\textbf{{\large Результаты}} \\
\begin{center}
\includegraphics[width=0.95\textwidth]{CT_S02E05/CT_S02E05_result.png}\\ [1cm]
\end{center}

\newpage
\textbf{{\large Задача A - Walking around Berhattan}} \\
\begin{center}
\includegraphics[width=0.9\textwidth]{CT_S02E05/CT_S02E05_A1.png}\\ [1cm]
\includegraphics[width=0.9\textwidth]{CT_S02E05/CT_S02E05_A2.png}\\ [1cm]
\end{center}
\textbf{{\large Алгоритм}} \\
{\Huge ???????????????????} \\ 
\\
%\newpage
\textbf{{\large Исходный код}}
\begin{lstlisting}[language=C]
#include <iostream>
#include <algorithm>
#include <iomanip>
#include <vector>
#include <sstream>
#include <fstream>
#define LL  long long

using namespace std;

enum dtype {
    UP,
    DOWN,
    LEFT,
    RIGHT,
};

int main() {
    ifstream in;
    ofstream out;
    in.open("input.txt");
    out.open("output.txt");

    LL n, m;
    LL answer = 0;
    in >> n >> m;
    char t;
    in.get();
    
    vector< vector<int> > map(n + 2, vector<int>(m + 2, 0));
    vector< vector<bool> > used(n + 2, vector<bool>(m + 2, false));
    
    for (LL i = 1; i <= n; i++) {
        for (LL j = 1; j <= m; j++) {
            t = in.get();
            map[i][j] = t - '0';
        }
        in.get();
    }
    
    int x = 1, y = 1;
    dtype dir = RIGHT; // last dir

    while ((t = in.get()) != EOF) {
        if (t == 'M') {
            if (dir == RIGHT) {
 				answer += map[x][y];
                answer += map[x - 1][y];
            	if (!used[x][y]) {
                	map[x][y] /= 2;
                	used[x][y] = true;
                }
                if (!used[x - 1][y]) {
                	map[x - 1][y] /= 2;
                	used[x - 1][y] = true;
                }
                y++;
            }
            else if (dir == LEFT) {
            	answer += map[x][y - 1];
                answer += map[x - 1][y - 1];
            	if (!used[x][y - 1]) {
            		map[x][y - 1] /= 2;
            		used[x][y - 1] = true;
            	}
            	if (!used[x - 1][y - 1]) {
            		map[x - 1][y - 1] /= 2;
                	used[x - 1][y - 1] = true;
            	}
                y--;
            }
            else if (dir == UP) {
                answer += map[x - 1][y - 1];
                answer += map[x - 1][y];
                if (!used[x - 1][y - 1]) {
                	map[x - 1][y - 1] /= 2;
                	used[x - 1][y - 1] = true;
                }
                if (!used[x - 1][y]) {
                	map[x - 1][y] /= 2;
                	used[x - 1][y] = true;
                }
                x--;
            }
            else if (dir == DOWN) {
                answer += map[x][y];
                answer += map[x][y - 1];
                if (!used[x][y]) {
                	map[x][y] /= 2;
                	used[x][y] = true;
                }
                if (!used[x][y - 1]) {
                	map[x][y - 1] /= 2;
               		used[x][y - 1] = true;
                }
                x++;
            }
            
        }
        else if (t == 'R') {
            if (dir == UP) dir = RIGHT;
            else if (dir == DOWN) dir = LEFT;
            else if (dir == LEFT) dir = UP;
            else dir = DOWN;
        }
        else if (t == 'L'){
            if (dir == UP) dir = LEFT;
            else if (dir == DOWN) dir = RIGHT;
            else if (dir == LEFT) dir = DOWN;
            else dir = UP;
        }
    }
    out << answer << endl;
    in.close();
    out.close();
    return 0;
}
\end{lstlisting}


\newpage
\textbf{{\large Задача G - Plural Form of Nouns}} \\
\begin{center}
\includegraphics[width=0.9\textwidth]{CT_S02E05/CT_S02E05_G.png}\\ [1cm]
\end{center}
\textbf{{\large Алгоритм}} \\
{\Huge ???????????????????}
\newpage
\textbf{{\large Исходный код}} \\
\begin{lstlisting}[language=C]
#include <iostream>
#include <sstream>
#include <fstream>
#define LL  long long

using namespace std;

int main() {
    ifstream in;
    ofstream out;
    in.open("input.txt");
    out.open("output.txt");

    LL n;
    string s;
    
    in >> n;
    
    for (LL i = 0; i < n; i++) {
        in >> s;
        size_t l = s.size() - 1;
        if ((s[l] == 'h' && s[l - 1] == 'c') || s[l] == 's' || s[l] == 'x' || s[l] == 'o') {
            out << s;
            out << "es" << endl;
        }
        else if (s[l] == 'f') {
            for (size_t j = 0; j < l; j++) {
                out << s[j];
            }
            out << "ves" << endl;
        }
        else if (s[l] == 'e' && s[l - 1] == 'f') {
            for (size_t j = 0; j < l - 1; j++) {
                out << s[j];
            }
            out << "ves" << endl;
        }
        else if (s[l] == 'y') {
            for (size_t j = 0; j < l; j++) {
                out << s[j];
            }
            out << "ies" << endl;
        }
        else {
            out << s << "s" << endl;
        }
    }
    
    in.close();
    out.close();
    
    return 0;
}
\end{lstlisting}



%----------------------------------------------------------------------------------------
%
%	Codeforces Training S02E06
%
%----------------------------------------------------------------------------------------
\newpage
\subsection{Codeforces Training S02E06}

\textbf{{\large Результаты}} \\
\begin{center}
\includegraphics[width=0.95\textwidth]{CT_S02E06/CT_S02E06_result.png}\\ [1cm]
\end{center}



%----------------------------------------------------------------------------------------
%
%	Тренировка СПбГУ B #3 Поиск кратчайшего пути и DFS
%
%----------------------------------------------------------------------------------------
\newpage
\subsection{Тренировка СПбГУ Поиск кратчайшего пути и DFS}

\textbf{{\large Результаты}} \\
\begin{center}
\includegraphics[width=0.95\textwidth]{SPBGU_GRAPHS/SPBGU_GRAPHS_result.png}\\ [1cm]
\end{center}

\newpage
\textbf{{\large Задача C - Флойд}} \\
\begin{center}
\includegraphics[width=0.9\textwidth]{SPBGU_GRAPHS/SPBGU_GRAPHS_C.png}\\ [1cm]
\end{center}
\textbf{{\large Алгоритм}} \\
В задаче требуется найти кратчайшие пути между всеми парами вершин и представить их матрицей смежности. Ее можно решить используя алгоритм Флойда-Уоршелла за $O(n^3)$. \\ 
\\
\newpage
\textbf{{\large Исходный код}}
\begin{lstlisting}[language=C]
#include <iomanip>
#include <iostream>
#include <algorithm>
#include <fstream>

using namespace std;

int main() {
    
    more_speed
    ifstream in("floyd.in");
    ofstream out("floyd.out");
    
    int n;
    in >> n;
    vector<vector<int> > m(n, vector<int>(n, 0));
    for (int i = 0; i < n; i++) {
        for (int j = 0; j < n; j++) {
            in >> m[i][j];
        }
    }
    
    for (int k = 0; k < n; k++) {
        for (int i = 0; i < n; i++) {
            for (int j = 0; j < n; j++) {
                m[i][j] = min(m[i][j], m[i][k] + m[k][j]);
            }
        }
    }
    
    for (int i = 0; i < n; i++) {
        for (int j = 0; j < n; j++) {
            out << m[i][j];
            if (j < n - 1) out << " ";
            else out << endl;
        }
    }

    in.close();
    out.close();
    
    return 0;
}
\end{lstlisting}


\newpage
\textbf{{\large Задача D - Поиск цикла}} \\
\begin{center}
\includegraphics[width=0.9\textwidth]{SPBGU_GRAPHS/SPBGU_GRAPHS_D.png}\\ [1cm]
\end{center}
\textbf{{\large Алгоритм}} \\
Задача решается поиском в глубину. Нужно сделать серию поисков в глубину, заходя в новую вершину будем красить ее в серый цвет, а выходя в черный. Если заходим в серую вершину, то цикл найден. \\
%\newpage
\textbf{{\large Исходный код}} \\
\begin{lstlisting}[language=C]
#include <iomanip>
#include <iostream>
#include <algorithm>
#include <fstream>

vector<set<LL> > g;
vector<char> color;
vector<LL> p;
LL cycle_st, cycle_end;

bool dfs (LL v) {
    color[v] = 1;
    for (set<LL>::iterator i = g[v].begin(); i != g[v].end(); i++) {
        LL to = *i;
        if (color[to] == 0) {
            p[to] = v;
            if (dfs(to)) return true;
        }
        else if (color[to] == 1){
            cycle_st = to;
            cycle_end = v;
            return true;
        }
    }
    color[v] = 2;
    return false;
}

using namespace std;
int main() {
    
    more_speed
    ifstream in("cycle.in");
    ofstream out("cycle.out");
    
    LL n, m, f, t;
    in >> n >> m;
    g.resize(n);
    
    for (LL i = 0; i < m; i++) {
        in >> f >> t;
        g[f - 1].insert(t - 1);
    }
    
    p.assign(n, -1);
    color.assign(n, 0);
    cycle_st = -1;
    for (LL i = 0; i < n; i++) {
        if (dfs(i)) break;
    }
    if (cycle_st == -1) {
        out << "NO" << endl;
    }
    else {
        out << "YES" << endl;
        vector<LL> cycle;
        for (LL v = cycle_end; v != cycle_st; v = p[v]) {
            cycle.push_back(v);
        }
        cycle.push_back(cycle_st);
        reverse(cycle.begin(), cycle.end());
        for (size_t i = 0; i < cycle.size(); i++) {
            out << cycle[i] + 1 << " ";
        }
        out << endl;
    }
    in.close();
    out.close();
    
    return 0;
}
\end{lstlisting}



%----------------------------------------------------------------------------------------
%
%	OpenCup GP of SPb
%
%----------------------------------------------------------------------------------------
\newpage
\subsection{OpenCup GrandPrix of SPb.}

\textbf{{\large Задача А - Барабашка}} \\
\begin{center}
\includegraphics[width=0.9\textwidth]{OC_SPB/OC_SPB_A1.png}\\ [1cm]
\includegraphics[width=0.9\textwidth]{OC_SPB/OC_SPB_A2.png}\\ [1cm]
\end{center}
\newpage

\textbf{{\large Алгоритм}} \\
Задача на реализацию. Нужно считать строки и каждой строке сопоставить правильное сочетание цвета и предмета по заданым в условии правилам. Сначала определим, какие сочетания уже имеются в предложении, затем проверим, есть ли среди них корректные, если есть, то это ответ, иначе нужно выбрать любое правильное сочетание. \\ 
\\
%\newpage
\textbf{{\large Исходный код}}
\begin{lstlisting}[language=C++]
#include <iomanip>
#include <iostream>
#include <algorithm>

bool isChar(char c) {
    return (c >= 'A' && c <= 'Z') || (c >= 'a' && c <= 'z');
}

int num(string s) {
    if      (s == "white" || s == "barabashka") return 0;
    else if (s == "blue"  || s == "book")       return 1;
    else if (s == "red"   || s == "chair")      return 2;
    else if (s == "gray"  || s == "mouse")      return 3;
    else return 4;
}
 
int main() {
    ifstream in("barabashka.in");
    ofstream out("barabashka.out");
    
    string white = "white",
           blue  = "blue",
           red   = "red",
           gray  = "gray",
           green = "green";
    string barab = "barabashka",
           book  = "book",
           chair = "chair",
           mouse = "mouse",
           bottle = "bottle";
    
    string current;
    string firstColor, firstObject;
    string secondColor, secondObject;
    bool needFirstColor, needFirstObject;
    bool needSecondColor, needSecondObject;
    bool complete;
    
    
    for (int i = 0; i < 5; i++) {
        char cSymb = in.get();
        bool used[5] = {false};
        needFirstColor   = true;
        needFirstObject  = false;
        needSecondColor  = false;
        needSecondObject = false;
        complete         = false;
        
        while (cSymb != '0') {
            while (cSymb != ' ' && isChar(cSymb)) {
                current += tolower(cSymb);
                cSymb = in.get();
            }
            if (needFirstColor) {
                if (current == white)      { firstColor = white; needFirstObject = true; needFirstColor = false; }
                else if (current == blue)  { firstColor = blue;  needFirstObject = true; needFirstColor = false; }
                else if (current == red)   { firstColor = red;   needFirstObject = true; needFirstColor = false; }
                else if (current == gray)  { firstColor = gray;  needFirstObject = true; needFirstColor = false; }
                else if (current == green) { firstColor = green; needFirstObject = true; needFirstColor = false; }
            }
            else if (needFirstObject) {
                if (current == barab)       { firstObject = barab;  needSecondColor = true; needFirstObject = false; }
                else if (current == book)   { firstObject = book;   needSecondColor = true; needFirstObject = false; }
                else if (current == chair)  { firstObject = chair;  needSecondColor = true; needFirstObject = false; }
                else if (current == mouse)  { firstObject = mouse;  needSecondColor = true; needFirstObject = false; }
                else if (current == bottle) { firstObject = bottle; needSecondColor = true; needFirstObject = false; }
            }
            else if (needSecondColor) {
                if (current == white)      { secondColor = white; needSecondObject = true; needSecondColor = false; }
                else if (current == blue)  { secondColor = blue;  needSecondObject = true; needSecondColor = false; }
                else if (current == red)   { secondColor = red;   needSecondObject = true; needSecondColor = false; }
                else if (current == gray)  { secondColor = gray;  needSecondObject = true; needSecondColor = false; }
                else if (current == green) { secondColor = green; needSecondObject = true; needSecondColor = false; }
            }
            else if (needSecondObject) {
                if (current == barab)       { secondObject = barab;  complete = true; needSecondObject = false; }
                else if (current == book)   { secondObject = book;   complete = true; needSecondObject = false; }
                else if (current == chair)  { secondObject = chair;  complete = true; needSecondObject = false; }
                else if (current == mouse)  { secondObject = mouse;  complete = true; needSecondObject = false; }
                else if (current == bottle) { secondObject = bottle; complete = true; needSecondObject = false; }
            }
            if (complete) {
                cSymb = '0';
                used[num(firstColor)] = true;
                used[num(firstObject)] = true;
                used[num(secondColor)] = true;
                used[num(secondObject)] = true;
                
                if ((firstColor == white && firstObject == barab) ||
                    (secondColor == white && secondObject == barab)) { out << white << " " << "Barabashka" << endl; }
                
                else if ((firstColor == blue && firstObject == book) ||
                         (secondColor == blue && secondObject == book)) { out << blue << " " << book << endl; }
                
                else if ((firstColor == red && firstObject == chair) ||
                         (secondColor == red && secondObject == chair)) { out << red << " " << chair << endl; }
                
                else if ((firstColor == gray && firstObject == mouse) ||
                         (secondColor == gray && secondObject == mouse)) { out << gray << " " << mouse << endl; }
                
                else if ((firstColor == green && firstObject == bottle) ||
                         (secondColor == green && secondObject == bottle)) { out << green << " " << bottle << endl; }
                
                else {
                    for (int j = 0; j < 5; j++) {
                        if (!used[j]) {
                            if (j == 0) { out << white << " " << "Barabashka" << endl; }
                            else if (j == 1) { out << blue << " " << book << endl; }
                            else if (j == 2) { out << red << " " << chair << endl; }
                            else if (j == 3) { out << gray << " " << mouse << endl; }
                            else             { out << green << " " << bottle << endl; }
                            break;
                        }
                    }
                }
                
                
            }
            current.clear();
            if (cSymb != '\0') cSymb = in.get();
        }
    }
    return 0;
}
\end{lstlisting}

\textbf{{\large Результаты}} \\
\begin{center}
\includegraphics[width=0.95\textwidth]{OC_SPB/OC_SPB_result.png}\\ [1cm]
\end{center}



%----------------------------------------------------------------------------------------
%
%	Самарский Международный Аэрокосмический Лицей, тренировка №1
%
%----------------------------------------------------------------------------------------
\newpage
\subsection{Самарский Международный Аэрокосмический Лицей, тренировка №1}

\textbf{{\large Задача D - Пивной вор}} \\
\begin{center}
\includegraphics[width=0.9\textwidth]{CT_SAMARA/CT_SAMARA_D.png}\\ [1cm]
\end{center}
\textbf{{\large Алгоритм}} \\
В этой задаче нужно отсортировать массив стоимостей по убыванию и сложить первые $k$ чисел. Это и будет ответом. \\ 
\\
\newpage
\textbf{{\large Исходный код}}
\begin{lstlisting}[language=C]
#include <iostream>
#include <vector>
#include <algorithm>
#include <fstream>

using namespace std;

int main() {
	long n, k;
	ifstream in("input.txt");
	ofstream out("output.txt");
	in >> n >> k;
	vector<long long> v(n);
	for (long i = 0; i < n; i++) {
		in >> v[i];
	}
	sort(v.begin(), v.end(), greater<long long>());
	long long answer = 0;
	for (long i = 0; i < k && i < v.size(); i++) {
		answer += v[i];
	}
	out << answer << endl;
	in.close();
	out.close();
	return 0;
}
\end{lstlisting}

\textbf{{\large Результаты}} \\
\begin{center}
\includegraphics[width=0.95\textwidth]{CT_SAMARA/CT_SAMARA_result.png}\\ [1cm]
\end{center}



%----------------------------------------------------------------------------------------
%
%	Codeforces ACM Восточный четвертьфинал
%
%----------------------------------------------------------------------------------------
\newpage
\subsection{Codeforces ACM-ICPC Восточный четвертьфинал}

\textbf{{\large Задача A - About Grisha N.}} \\
\begin{center}
\includegraphics[width=0.9\textwidth]{CT_ACM_EAST/CT_ACM_EAST_A.png}\\ [1cm]
\end{center}
\textbf{{\large Алгоритм}} \\
{\Huge ???????????????????} \\ 
\\
\newpage
\textbf{{\large Исходный код}}
\begin{lstlisting}[language=C]
#include <iostream>
#include <vector>
#include <algorithm>
#include <fstream>

using namespace std;

int main() {
	long n, k;
	ifstream in("input.txt");
	ofstream out("output.txt");
	in >> n >> k;
	vector<long long> v(n);
	for (long i = 0; i < n; i++) {
		in >> v[i];
	}
	sort(v.begin(), v.end(), greater<long long>());
	long long answer = 0;
	for (long i = 0; i < k && i < v.size(); i++) {
		answer += v[i];
	}
	out << answer << endl;
	in.close();
	out.close();
	return 0;
}
\end{lstlisting}

\newpage
\textbf{{\large Задача D - Zhenya moves from the dormitory}} \\
\begin{center}
\includegraphics[width=0.9\textwidth]{CT_ACM_EAST/CT_ACM_EAST_D1.png}\\ [1cm]
\includegraphics[width=0.5\textwidth]{CT_ACM_EAST/CT_ACM_EAST_D2.png}\\ [1cm]
\end{center}
\textbf{{\large Алгоритм}} \\
{\Huge ???????????????????} \\ 
\\
\newpage
\textbf{{\large Исходный код}}
\begin{lstlisting}[language=C]

#include <iostream>
#include <algorithm>
#include <vector>
#include <fstream>

using namespace std;

typedef struct {
    long money;
    long ad;
    int num;
}frd;

typedef struct {
    long rooms;
    long price;
    long ad;
    int num;
}apartment;

bool cmp_ap(const apartment &a1, const apartment &a2) {
    return a1.ad > a2.ad;
}

bool cmp_fr(const frd &f1, const frd &f2) {
    return f1.ad > f2.ad;
}

int find_good_ap(const vector<apartment> &a, long money1, long money2) {
    for (int i = 0; i < a.size(); i++) {
        long price_for_each = a[i].price / 2;
        if (a[i].price % 2 == 1) {
            price_for_each++;
        }
        if (money1 >= price_for_each && money2 >= price_for_each && a[i].rooms == 2)
            return i;
    }
    return -1;
}

int main() {
    long money, ad1, ad2;
    long n, m;
    cin >> money >> ad1 >> ad2;
    cin >> n;
    vector<frd> fr(n);
    for (int i = 0; i < n; i++) {
        cin >> fr[i].money >> fr[i].ad;
        fr[i].num = i + 1;
    }
    cin >> m;
    vector<apartment> ap(m);
    long max_ad_in_1room = -1, max_ad_in_2room = -1;
    int  in_1room_num = -1, in_2room_num = -1;
    bool can_buy_alone = false;
    for (int i = 0; i < m; i++) {
        cin >> ap[i].rooms >> ap[i].price >> ap[i].ad;
        ap[i].num = i + 1;
        if (ap[i].rooms == 1) {
            if (money >= ap[i].price && ad1 + ap[i].ad > max_ad_in_1room) {
                max_ad_in_1room = ad1 + ap[i].ad;
                in_1room_num = i + 1;
            }
        }
        else {
            if (money >= ap[i].price && ad2 + ap[i].ad > max_ad_in_2room) {
                max_ad_in_2room = ad2 + ap[i].ad;
                in_2room_num = i + 1;
            }
        }
    }
    sort(ap.begin(), ap.end(), cmp_ap);
    int ans_ap = -1, ans_fr = -1;
    long max_ad_tog = -1;
    int found_ap = -1;
    for (int i = 0; i < n; i++) {
        found_ap = find_good_ap(ap, money, fr[i].money);
        if (found_ap != -1) {
            long cur_ad = fr[i].ad + ap[found_ap].ad;
            if (cur_ad > max_ad_tog) {
                ans_ap = ap[found_ap].num;
                ans_fr = fr[i].num;
                max_ad_tog = cur_ad;
            }
        }
    }
    
    long alone = 0, whereAlone = 0;
    if (max_ad_in_1room != -1 || max_ad_in_2room != -1) {
        if (max_ad_in_1room > max_ad_in_2room) {
            alone = max_ad_in_1room;
            whereAlone = in_1room_num;
        }
        else {
            alone = max_ad_in_2room;
            whereAlone = in_2room_num;
        }
        can_buy_alone = true;
    }
    
    if (found_ap == -1 && !can_buy_alone) {
        cout << "Forget about apartments. Live in the dormitory." << endl;
        return 0;
    }
    
    if (alone > max_ad_tog)
        cout << "You should rent the apartment #" << whereAlone << " alone." << endl;
    else
        cout << "You should rent the apartment #" << ans_ap << " with the friend #" << ans_fr << "." << endl;
        
    return 0;
}
\end{lstlisting}

\newpage
\textbf{{\large Задача I - Traffic Jam in Flower Town}} \\
\begin{center}
\includegraphics[width=0.9\textwidth]{CT_ACM_EAST/CT_ACM_EAST_I.png}\\ [1cm]
\end{center}
\textbf{{\large Алгоритм}} \\
{\Huge ???????????????????} \\ 
\\
\newpage
\textbf{{\large Исходный код}}
\begin{lstlisting}[language=C]
#include <iostream>
#include <deque>

using namespace std;

int main() 
{
    deque <char> south;
    deque <char> north;

    char temp;
    int time = 0;
    temp = cin.get();
    while(temp != '\n'){
        south.push_back(temp);
        temp = cin.get();
    }
    temp = cin.get();
    while(temp != '\n'){
        north.push_back(temp);
        temp = cin.get();
    }
    char s, n;
    while(south.size() > 0 && north.size() > 0){
        s = south[0];
        n = north[0];
        time++;
        if(s == 'R' && n == 'L')
            south.pop_front();
        else if(s == 'L' && n == 'R')
            north.pop_front();
         else if(s == 'L' && n == 'F')
            north.pop_front();
         else if(s == 'F' && n == 'L')
            south.pop_front();
         else {
            south.pop_front();
            north.pop_front();
         }
    }
    
    if(south.size() > 0)
        time += south.size();
    if(north.size() > 0)
        time += north.size();

    cout << time << endl;
    return 0;
}
\end{lstlisting}

\newpage
\textbf{{\large Задача L - Donald is a postman}} \\
\begin{center}
\includegraphics[width=0.9\textwidth]{CT_ACM_EAST/CT_ACM_EAST_L.png}\\ [1cm]
\end{center}
\textbf{{\large Алгоритм}} \\
{\Huge ???????????????????} \\ 
\\
\newpage
\textbf{{\large Исходный код}}
\begin{lstlisting}[language=C]
#include <iostream>

using namespace std;

int main() 
{
    int n;
    cin >> n;
    string name;
    int state = 1;
    int answer = 0;
    for (int i = 0; i < n; i++) {
        cin >> name;
        char t = name[0];
        if (state == 1) {
            if (t == 'B' || t == 'M' || t == 'S') {
                answer++;
                state = 2;
            }
            else if (t == 'D' || t == 'J' || t == 'K' || t == 'T' || t == 'W' || t == 'G') {
                answer += 2;
                state = 3;
            }
        }
        else if (state == 2) {
            if (t == 'A' || t == 'P' || t == 'O' || t == 'R') {
                answer++;
                state = 1;
            }
            else if (t == 'D' || t == 'J' || t == 'K' || t == 'T' || t == 'W' || t == 'G') {
                answer++;
                state = 3;
            }
                
        }
        else {
            if (t == 'A' || t == 'P' || t == 'O' || t == 'R') {
                answer += 2;
                state = 1;
            }
            else if (t == 'B' || t == 'M' || t == 'S') {
                answer++;
                state = 2;
            }
        }
    }
    
    cout << answer << endl;
    return 0;
}
\end{lstlisting}

\newpage
\textbf{{\large Результаты}} \\
\begin{center}
\includegraphics[width=0.95\textwidth]{CT_ACM_EAST/CT_ACM_EAST_result.png}\\ [1cm]
\end{center}



%----------------------------------------------------------------------------------------
%
%	Codeforces ACM Южный четвертьфинал
%
%----------------------------------------------------------------------------------------
\newpage
\subsection{Codeforces ACM-ICPC Южный четвертьфинал}

\textbf{{\large Задача D - Data Center}} \\
\begin{center}
\includegraphics[width=0.9\textwidth]{CT_ACM_WEST/CT_ACM_WEST_D.png}\\ [1cm]
\end{center}
\textbf{{\large Алгоритм}} \\
{\Huge ???????????????????} \\ 
\\
\newpage
\textbf{{\large Исходный код}}
\begin{lstlisting}[language=C]
#include <iostream>
#include <vector>
#include <algorithm>
#include <fstream>

using namespace std;

typedef struct {
    LL num;
    ULL cap;
    short low;
} server;

bool cmp_low(const server &a, const server &b) {
    return a.low > b.low;
}

bool cmp_cap(const server &a, const server &b) {
    if (a.cap == b.cap) return a.low > b.low;
    return a.cap > b.cap;
}

int main() {
    LL n;
    ULL m;
    cin >> n >> m;
    vector<server> s(n);
    for (LL i = 0; i < n; i++) {
        s[i].num = i + 1;
        cin >> s[i].cap;
        cin >> s[i].low;
    }
    
    sort(s.begin(), s.end(), cmp_cap);
    LL ind = 0;
    ULL curr_cap = 0;
    LL count = 0;

    while (curr_cap < m) {
        curr_cap += s[ind].cap;
        if (s[ind].low) count++;
        ind++;
    }
    
    ULL rem = curr_cap - m;
    
    if (rem == 0) {
        cout << ind << " ";
        cout << count << endl;
        for (LL i = 0; i < ind; i++) {
            cout << s[i].num << " ";
        }
        cout << endl;
        return 0;
    }
    
    for (LL i = ind - 1; i >= 0; i--) {
        if (!s[i].low && rem > 0) {
            bool azaza = false;
            for (LL j = ind; j < n; j++) {
                if (s[j].low && (s[j].cap + rem >= s[i].cap)) {
                    rem -= s[i].cap - s[j].cap;
                    swap(s[i], s[j]);
                    count++;
                    azaza = true;
                    break;
                }
            }
            if (!azaza) break;
        }
        else if (rem == 0) break;
    }
    
    cout << ind << " " << count << endl;
    for (LL i = 0; i < ind; i++) {
        cout << s[i].num << " ";
    }
    cout << endl;;
	return 0;
}
\end{lstlisting}

\newpage
\textbf{{\large Задача I - Sales in GameStore}} \\
\begin{center}
\includegraphics[width=0.9\textwidth]{CT_ACM_WEST/CT_ACM_WEST_I.png}\\ [1cm]
\end{center}
\textbf{{\large Алгоритм}} \\
{\Huge ???????????????????} \\ 
\\
\newpage
\textbf{{\large Исходный код}}
\begin{lstlisting}[language=C]
#include <iostream>
#include <vector>
#include <algorithm>
using namespace std;
bool compare(const int &a, const int &b)
{
    return a<b;
}
int main(int argc, const char * argv[]) {
    vector<int> p(2001);
    int n;
    cin >> n;
    for(int i=0; i<n; ++i)
        cin >> p[i];
    sort(p.begin(), p.begin()+n);
    int sum = 0;
    int i=0;
    while(i<n-1 && sum+p[i]<=p[n-1])
        sum += p[i++];
    cout << i+1;
    return 0;
}
\end{lstlisting}

\newpage
\textbf{{\large Задача M - Variable Shadowing}} \\
\begin{center}
\includegraphics[width=0.9\textwidth]{CT_ACM_WEST/CT_ACM_WEST_M1.png}\\ [1cm]
\includegraphics[width=0.9\textwidth]{CT_ACM_WEST/CT_ACM_WEST_M2.png}\\ [1cm]
\end{center}
\textbf{{\large Алгоритм}} \\
{\Huge ???????????????????} \\ 
\\
\newpage
\textbf{{\large Исходный код}}
\begin{lstlisting}[language=C]
#include <iostream>
#include <vector>
#include <algorithm>
using namespace std;
typedef struct {
    char ch;
    int line;
    int sym;
} var;
int main() {
    int f,n;
    cin >> n;
    vector <stack <var> > all_alph(26);
    stack <var> curr;
    char temp;
    int symbol = 0;
    var to_put;
    temp = cin.get();
    for (int i = 1; i <= n; i++){
        temp = cin.get();
        symbol = 1;
        while(temp != '\n') {
            if(temp == '}') {
                to_put = curr.top();
                curr.pop();
                while (to_put.ch != '{') {
                    all_alph[to_put.ch - 97].pop();
                    to_put = curr.top();
                    curr.pop();
                }
            }
            else if(temp == '{') {
                to_put.ch = temp;
                to_put.sym = symbol;
                to_put.line = i;
                curr.push(to_put);
            }
            else if (temp != ' '){
                to_put.sym = symbol;
                to_put.line = i;
                to_put.ch = temp;
                curr.push(to_put);
                if(all_alph[temp - 97].size() != 0)
                    cout << to_put.line << ":" << to_put.sym 
                    << ": warning: shadowed declaration of "<< to_put.ch 
                    << ", the shadowed position is " << all_alph[temp - 97].top().line
                    << ":" << all_alph[temp - 97].top().sym << endl;
                all_alph[temp - 97].push(to_put);
            }
            
            temp = cin.get();
            symbol++;
        }
    }
    return 0;
}
\end{lstlisting}

\newpage
\textbf{{\large Результаты}} \\
\begin{center}
\includegraphics[width=0.95\textwidth]{CT_ACM_WEST/CT_ACM_WEST_result.png}\\ [1cm]
\end{center}



%----------------------------------------------------------------------------------------
%
%	ACM 1/4 Final
%
%----------------------------------------------------------------------------------------
\newpage
\subsection{ACM-ICPC Московский четвертьфинал}

Так как соревнование проводилось в МГУ, то турнирная таблица с результатами и исходные коды программ не доступны. \\



%----------------------------------------------------------------------------------------
%
%	Codeforces Training S02E07
%
%----------------------------------------------------------------------------------------
\newpage
\subsection{Codeforces Training S02E07}

\textbf{{\large Задача C - Will It Stop?}} \\
\begin{center}
\includegraphics[width=0.9\textwidth]{CT_S02E07/CT_S02E07_C.png}\\ [1cm]
\end{center}
\textbf{{\large Алгоритм}} \\
{\Huge ???????????????????} \\ 

\textbf{{\large Исходный код}}
\begin{lstlisting}[language=C]
#include <iostream>
using namespace std;
int main()
{
    unsigned long long a;
    cin >> a;
    while(!(a%2))a/=2;
    if(a==1)
        cout << "TAK";
    else
        cout << "NIE";
    return 0;
}
\end{lstlisting}

\newpage
\textbf{{\large Задача H - Afternoon Tea}} \\
\begin{center}
\includegraphics[width=0.9\textwidth]{CT_S02E07/CT_S02E07_H.png}\\ [1cm]
\end{center}
\textbf{{\large Алгоритм}} \\
{\Huge ???????????????????} \\ 

\textbf{{\large Исходный код}}
\begin{lstlisting}[language=C]
#include <iostream>
#include <cmath>
using namespace std;
int main()
{
    int n;
    cin >> n;
    if(n==1)
    {
        cout << "HM";
        return 0;
    }
    cin.get();
    char c;
    long double hDrunked = 0, mDrunked = 0;
    hDrunked = mDrunked = (1-pow(0.5, n))*0.5;
    for(int i=0; i<n-1; ++i)
    {
        c=cin.get();
        if(c=='H')
            hDrunked += (1-pow(0.5, n-i-1))*0.5;
        else
            mDrunked += (1-pow(0.5, n-i-1))*0.5;
    }
    if(hDrunked>mDrunked)
        cout << "H";
    else if(hDrunked<mDrunked)
        cout << "M";
    return 0;
}
\end{lstlisting}

\textbf{{\large Результаты}} \\
\begin{center}
\includegraphics[width=0.95\textwidth]{CT_S02E07/CT_S02E07_result.png}\\ [1cm]
\end{center}



%----------------------------------------------------------------------------------------
%
%	Codeforces Crypto Cup
%
%----------------------------------------------------------------------------------------
\newpage
\subsection{Codeforces Crypto Cup 1.0}

\textbf{{\large Задача B - :-P}} \\
\begin{center}
\includegraphics[width=0.9\textwidth]{CT_Crypto/CT_Crypto_B.png}\\ [1cm]
\end{center}
\textbf{{\large Алгоритм}} \\
{\Huge ???????????????????} \\ 

\textbf{{\large Исходный код}}
\begin{lstlisting}[language=C]
#include <iostream>
#include <cmath>
#include <vector>
using namespace std;
int main(int argc, const char * argv[]) {
    char str[100001];
    long p, len;
    while((str[len]=cin.get())!='\n') ++len;
    str[len] = '\0';
    cin >> p;
    vector< vector <char> > v(p);
    long vSize = len/p, r = len%p;
    for(long i=r; i<p; ++i) {
        v[i].resize(vSize);
    }
    for(long i=0; i<r; ++i) {
        v[i].resize(vSize+1);
    }
    for(long i=0, k=0; i<p; ++i) {
        for(long j=0; j<v[i].size(); ++j, ++k) {
            v[i][j] = str[k];
        }
    }
    for(long i=0; i<len; ++i) {
        cout.put(v[i%p][i/p]);
    }
    return 0;
}
\end{lstlisting}

\newpage
\textbf{{\large Задача C - Pgkpxumgs}} \\
\begin{center}
\includegraphics[width=0.9\textwidth]{CT_Crypto/CT_Crypto_C.png}\\ [1cm]
\end{center}
\textbf{{\large Алгоритм}} \\
{\Huge ???????????????????} \\ 

\textbf{{\large Исходный код}}
\begin{lstlisting}[language=C]
#include <iostream>
#include <cstdio>
using namespace std;
int main() {
	char cur, prev;
	cout.put(prev = cin.get());
	while ((cur = cin.get()) != '\n') {
		if ((int)(cur - prev) < 0) cout << (char)(cur - prev + '{');
		else cout << (char)(cur - prev + 'a');
		prev = cur;
 	}
	return 0;
}
\end{lstlisting}

\newpage
\textbf{{\large Задача H - Peace of AmericaReunion}} \\
\begin{center}
\includegraphics[width=0.9\textwidth]{CT_Crypto/CT_Crypto_H.png}\\ [1cm]
\end{center}
\textbf{{\large Алгоритм}} \\
{\Huge ???????????????????} \\ 

\textbf{{\large Исходный код}}
\begin{lstlisting}[language=C]
#include <iostream>
using namespace std;
int main() {
	vector<long> v(26);
	long length = 0;
	for (int i = 0; i < 26; i++) {
		cin >> v[i];
		length += v[i];
	}
	vector<char> answer(length);
	long pos;
	int nextSymb = -1;
	for (int i = 0; i < 26; i++) {
		if (v[i]) {
			nextSymb = i;
			break;
		}
	}
	for (long i = 0; i < length; i++) {
		cin >> pos;
		if (!v[nextSymb]) {
			for (int i = nextSymb; i < 26; i++) {
				if (v[i]) {
					nextSymb = i;
					break;
				}
			}
		}
		answer[--pos] = (char)(nextSymb + 'a');
		v[nextSymb]--;
	}
	for (auto n : answer) cout << n;

	return 0;
}
\end{lstlisting}

\newpage
\textbf{{\large Задача I - Peace of AmericanPie}} \\
\begin{center}
\includegraphics[width=0.9\textwidth]{CT_Crypto/CT_Crypto_I.png}\\ [1cm]
\end{center}
\textbf{{\large Алгоритм}} \\
{\Huge ???????????????????} \\ 

\textbf{{\large Исходный код}}
\begin{lstlisting}[language=C]
#include <iostream>
using namespace std;
int main() {
	int byte = 0;
	int spow = 256;
	int collected = 0;
	char currentBit;
	while ((currentBit = cin.get()) != '\n') {
		cin.unget();
		while (collected < 8) {
			currentBit = cin.get();
			byte += (currentBit - '0') * spow;
			spow /= 2;
			collected++;
		}
		cout << (char)(byte / 2);
		byte = 0;
		spow = 256;
		collected = 0;
	}
	return 0;
}
\end{lstlisting}

\newpage
\textbf{{\large Задача J - Common}} \\
\begin{center}
\includegraphics[width=0.9\textwidth]{CT_Crypto/CT_Crypto_J.png}\\ [1cm]
\end{center}
\textbf{{\large Алгоритм}} \\
{\Huge ???????????????????} \\ 

\textbf{{\large Исходный код}}
\begin{lstlisting}[language=C]
#include <iostream>
using namespace std;
int main() {
	char t;
	while ((t = cin.get()) != '\n') {
		switch (t) {
			case 'a':
				cout << "n";
				break;
			case 'b':
				cout << "h";
				break;
			case 'c':
				cout << "r";
				break;
			case 'd':
				cout << "x";
				break;
			case 'e':
				cout << "k";
				break;
			case 'f':
				cout << "e";
				break;
			case 'g':
				cout << "y";
				break;
			case 'h':
				cout << "o";
				break;
			case 'i':
				cout << "q"; 
				break;
			case 'j':
				cout << "m";
				break;
			case 'k':
				cout << "j";
				break;
			case 'l':
				cout << "b";
				break;
			case 'm':
				cout << "d";
				break;
			case 'n':
				cout << "u";
				break;
			case 'o':
				cout << "v";
				break;
			case 'p':
				cout << "a";
				break;
			case 'q':
				cout << "p";
				break;
			case 'r':
				cout << "w";
				break;
			case 's':
				cout << "g";
				break;
			case 't':
				cout << "z";
				break;
			case 'u':
				cout << "f";
				break;
			case 'v':
				cout << "i";
				break;
			case 'w':
				cout << "c";
				break;
			case 'x':
				cout << "s";
				break;
			case 'y':
				cout << "t";
				break;
			case 'z':
				cout << "l";
				break;
		}
	}
	return 0;
}
\end{lstlisting}

\newpage
\textbf{{\large Задача M - oPlus}} \\
\begin{center}
\includegraphics[width=0.9\textwidth]{CT_Crypto/CT_Crypto_M.png}\\ [1cm]
\end{center}
\textbf{{\large Алгоритм}} \\
{\Huge ???????????????????} \\ 

\textbf{{\large Исходный код}}
\begin{lstlisting}[language=C]
#include <iostream>
using namespace std;
int main() {
	long n;
	int curr, sum;
	cin >> n;
	for (long i = 0; i < n; i++) {
		cin >> curr;
		if (curr % 2) sum = 400;
		else sum = 398;
		cout << (char)(sum - curr - 112);
	}
	return 0;
}
\end{lstlisting}

\newpage
\textbf{{\large Задача N - tirnaoeumPt}} \\
\begin{center}
\includegraphics[width=0.9\textwidth]{CT_Crypto/CT_Crypto_N.png}\\ [1cm]
\end{center}
\textbf{{\large Алгоритм}} \\
{\Huge ???????????????????} \\ 

\textbf{{\large Исходный код}}
\begin{lstlisting}[language=C]
#include <iostream>
using namespace std;

int main()
{
    int m[] = {0, 1, 16, 17, 8, 9, 24, 25, 2, 3, 18, 19, 10, 11, 22, 23, 4,  5,  20, 21, 12, 13, 22, 23, 6,  7,  22, 23, 14, 15};
    int n, d;
    cin >> n;
    for(int i=0; i<n; ++i)
    {
        cin >> d;
        cout.put(m[d]+'a');
    }
    return 0;
}
\end{lstlisting}

\newpage
\textbf{{\large Задача Q - Peace of bzjd}} \\
\begin{center}
\includegraphics[width=0.9\textwidth]{CT_Crypto/CT_Crypto_Q.png}\\ [1cm]
\end{center}
\textbf{{\large Алгоритм}} \\
{\Huge ???????????????????} \\ 

\textbf{{\large Исходный код}}
\begin{lstlisting}[language=C]
#include <iostream>

using namespace std;

int main()
{
    char temp;
    temp = cin.get();
    while (temp != '\n' && temp != EOF) {
        if (temp == 'z') 
            temp = 'a';
        else temp++;
        cout << temp;
        temp = cin.get();
    }
    return 0;
}
\end{lstlisting}

\newpage
\textbf{{\large Задача R - 6227020800}} \\
\begin{center}
\includegraphics[width=0.9\textwidth]{CT_Crypto/CT_Crypto_R.png}\\ [1cm]
\end{center}
\textbf{{\large Алгоритм}} \\
{\Huge ???????????????????} \\ 

\textbf{{\large Исходный код}}
\begin{lstlisting}[language=C]
#include <iostream>
#include <algorithm>
#include <deque>
#include <cstdio>
using namespace std;
void p(string s) {
	cout << s << endl;
}
int gcd(int a, int b){
    if (b == 0)
        return a;
    return gcd(b, a%b);
}
int main() {
	char t;
	while ((t = cin.get()) != '\n') {
		t -= 13;
		if (t < 'a') {
			cout << (char)('z' - ('a' - t) + 1);
		}
		else {
			cout << t;
		}
	}
	return 0;
}
\end{lstlisting}

\newpage
\textbf{{\large Результаты}} \\
\begin{center}
\includegraphics[width=0.95\textwidth]{CT_Crypto/CT_Crypto_result.png}\\ [1cm]
\end{center}



%----------------------------------------------------------------------------------------
%
%	OpenCup GP of Siberia
%
%----------------------------------------------------------------------------------------
\newpage
\subsection{OpenCup GrandPrix of Siberia}

\textbf{{\large Задача 12 - Construction of Chand Baori}} \\
\begin{center}
\includegraphics[width=0.9\textwidth]{OC_Siberia/OC_Siberia_12.png}\\ [1cm]
\end{center}
\newpage

\textbf{{\large Алгоритм}} \\
{\Huge ???????????????????} \\ 
\\
%\newpage
\textbf{{\large Исходный код}}
\begin{lstlisting}[language=C++]
#include <iostream>
using namespace std;
#define ULL unsigned long long
int main(int argc, const char * argv[]) {
    ULL n, m;
    cin >> n >> m;
    ULL res = 1;
    for(int i=2; i<=n*2; i+=2)
    {
        res *= i;
    }
    if(res<m)
        cout << "Nope";
    else
        cout << "Harshat Mata";
    return 0;
}
\end{lstlisting}

\textbf{{\large Задача 13 - Sum}} \\
\begin{center}
\includegraphics[width=0.9\textwidth]{OC_Siberia/OC_Siberia_13.png}\\ [1cm]
\end{center}
\newpage

\textbf{{\large Алгоритм}} \\
{\Huge ???????????????????} \\ 
\\
%\newpage
\textbf{{\large Исходный код}}
\begin{lstlisting}[language=C++]
#include <iostream>
#include <fstream>
#include <cmath>
#include <vector>
using namespace std;
#define ULL unsigned long long
#define LL long long
int main(int argc, const char * argv[]) {
    ifstream in("input.txt");
    ofstream out("output.txt");
    ULL a, k, p;
    cin >> a >> k >> p;
    ULL sum = 0, prev = 1;
    for(ULL i = 0; i<k; ++i)
    {
        prev = (prev*a)%p;
        sum += prev;
    }
    cout << sum;
    out.close();
    in.close();
    return 0;
}
\end{lstlisting}

\textbf{{\large Задача 14 - Coinquerors}} \\
\begin{center}
\includegraphics[width=0.9\textwidth]{OC_Siberia/OC_Siberia_14_1.png}\\ [1cm]
\includegraphics[width=0.9\textwidth]{OC_Siberia/OC_Siberia_14_2.png}\\ [1cm]

\end{center}
\newpage

\textbf{{\large Алгоритм}} \\
{\Huge ???????????????????} \\ 
\\
%\newpage
\textbf{{\large Исходный код}}
\begin{lstlisting}[language=C++]
#include <iostream>
#include <fstream>
#include <cmath>
#include <vector>
using namespace std;
#define ULL unsigned long long
#define LL long long
#define eps 0.0001
struct player
{
    char name[256];
    LL x, y, r;
};
int main(int argc, const char * argv[]) {
    ifstream in("input.txt");
    ofstream out("output.txt");
    ULL T;
    in >> T;
    double pi81 = M_PI/81, pi2 = M_PI*2;
    for(ULL TT = 0; TT<T; ++TT)
    {
        ULL n;
        in >> n;
        vector<player> pl(n);
        for(int i=0; i<n; ++i)
        {
            in >> pl[i].name >> pl[i].x >> pl[i].y >> pl[i].r;
        }
        ULL maxInd = -1, max = 0, forTie = -1;
        for(int i=0; i<n; ++i)
        {
            ULL count = 0;
            LL rr = pl[i].r*pl[i].r;
            for(int j=0; j<n; ++j)
            {
                for(double pi = 0; pi<=pi2; pi += pi81)
                {
                    double x = (pl[j].r-eps)*cos(pi);
                    double y = (pl[j].r-eps)*sin(pi);
                    if((pl[j].x-pl[i].x+x)*(pl[j].x-pl[i].x+x)+(pl[j].y-pl[i].y+y)*(pl[j].y-pl[i].y+y)<=rr)
                    {
                        ++count;
                        break;
                    }
                }
            }
            if(count > max)
            {
                max = count;
                maxInd = i;
                forTie = -1;
            }
            else if(count == max)
            {
                max = count;
                forTie = maxInd;
                maxInd = i;
            }
        }
        if(maxInd == -1 || forTie != -1)
            out << "TIE" << '\n';
        else
            out << pl[maxInd].name << '\n';
    }
    out.close();
    in.close();
    return 0;
}
\end{lstlisting}

\textbf{{\large Результаты}} \\
\begin{center}
\includegraphics[width=0.95\textwidth]{OC_Siberia/OC_Siberia_result.png}\\ [1cm]
\end{center}




%----------------------------------------------------------------------------------------
%
%	Codeforces Training S02E08
%
%----------------------------------------------------------------------------------------
\newpage
\subsection{Codeforces Training S02E08}

\textbf{{\large Задача G - Growling Gears}} \\
\begin{center}
\includegraphics[width=0.9\textwidth]{CT_S02E08/CT_S02E08_G1.png}\\ [1cm]
\includegraphics[width=0.9\textwidth]{CT_S02E08/CT_S02E08_G2.png}\\ [1cm]
\end{center}
\textbf{{\large Алгоритм}} \\
{\Huge ???????????????????} \\ 
\\
%\newpage
\textbf{{\large Исходный код}}
\begin{lstlisting}[language=C]
#include <iostream>
#include <algorithm>

#include <fstream>

using namespace std;

int main() {
    int k;
    cin >> k;
    int n, a, b, c;
    double max_T = -1000000;
    int max_T_num;
    double temp;

    for(int i = 0; i < k; i++) {
        cin >> n;
        for(int j = 1; j <= n; j++) {
            cin >> a >> b >> c;
            temp = (b * b) / (4 * a) + c;
            if(temp > max_T) {
                max_T = temp;
                max_T_num = j;
            }
        }
        cout << max_T_num << endl;
        max_T = -1000000;
    }
	int q;
	cin >> q;
    return 0;
}
\end{lstlisting}


\newpage
\textbf{{\large Задача J - Jury Jeopardy}} \\
\begin{center}
\includegraphics[width=0.9\textwidth]{CT_S02E08/CT_S02E08_J1.png}\\ [1cm]
\includegraphics[width=0.9\textwidth]{CT_S02E08/CT_S02E08_J2.png}\\ [1cm]
\end{center}
\textbf{{\large Алгоритм}} \\
{\Huge ???????????????????}
\newpage
\textbf{{\large Исходный код}} \\
\begin{lstlisting}[language=C]
#include <iostream>
using namespace std;
int main(int argc, const char * argv[]) {
    long T;
    cin >> T;
    cout << T << '\n';
    char map[201][201];
    cin.get();
    while(T--)
    {
        for(long i=0; i<201; ++i)
            for(long j=0; j<201; ++j)
                map[i][j] = '#';
        char c;
        long x, y, minX, minY, maxX, maxY;
        maxX = maxY = minX = minY = x = y = 100;
        int dir = 0;
        while((c=cin.get())!='\n')
        {
            switch(dir)
            {
                case 0:
                    if(c=='R')
                        dir = 3;
                    else if(c=='L')
                        dir = 1;
                    else if(c=='B')
                        dir = 2;
                    break;
                case 1:
                    if(c=='R')
                        dir = 0;
                    else if(c=='L')
                        dir = 2;
                    else if(c=='B')
                        dir = 3;
                    break;
                case 2:
                    if(c=='R')
                        dir = 1;
                    else if(c=='L')
                        dir = 3;
                    else if(c=='B')
                        dir = 0;
                    break;
                case 3:
                    if(c=='R')
                        dir = 2;
                    else if(c=='L')
                        dir = 0;
                    else if(c=='B')
                        dir = 1;
                    break;
            }
            switch(dir)
            {
                case 0:
                    ++x;
                    if(x>maxX)
                        maxX = x;
                    break;
                case 1:
                    --y;
                    if(y<minY)
                        minY = y;
                    break;
                case 2:
                    --x;
                    if(x<minX)
                        minX = x;
                    break;
                case 3:
                    ++y;
                    if(y>maxY)
                        maxY = y;
                    break;
            }
            map[x][y] = '.';
        }
        cout << maxY-minY+3 << ' ' << maxX-minX+2 << '\n';
        for(long i = minY-1; i<=maxY+1; ++i)
        {
            for(long j = minX; j<=maxX+1; ++j)
                cout.put(map[j][i]);
            cout.put('\n');
        }
    }
    return 0;
}
\end{lstlisting}

\textbf{{\large Результаты}} \\
\begin{center}
\includegraphics[width=0.95\textwidth]{CT_S02E05/CT_S02E05_result.png}\\ [1cm]
\end{center}


%----------------------------------------------------------------------------------------
%
%	ЛИЧНЫЕ КОНТЕСТЫ
%
%----------------------------------------------------------------------------------------

\newpage
\section{Журнал по личным контестам Макарова Н.А.}

%----------------------------------------------------------------------------------------
%
%	Codeforces Отборочный контест СГАУ на четвертьфинал ACM ICPC
%
%----------------------------------------------------------------------------------------

\subsection{Codeforces Отборочный контест СГАУ на четвертьфинал ACM-ICPC}

\textbf{{\large Задача D - Игрушечные солдатики}} \\
\begin{center}
\includegraphics[width=0.9\textwidth]{CT_SGAU/CT_SGAU_D.png}\\ [1cm]
\end{center}
\textbf{{\large Алгоритм}} \\
{\Huge ???????????????????} \\ 
\\
\newpage
\textbf{{\large Исходный код}}
\begin{lstlisting}[language=C]
#include <iostream>
#include <vector>
#include <algorithm>
#define ll unsigned long long

using namespace std;

int main() {
ll n, m;
    ll remColor = 0;
    ll currentSoldier, currentColor, answer = 0;
    bool sameColors = true;
    cin >> n;
    vector<ll> soldiers(n + 1);
    for (ll i = 1; i <= n; i++) {
        cin >> soldiers[i];
        if (i == 1) remColor = soldiers[i];
        else if (soldiers[i] != remColor) sameColors = false;
    }
    if (sameColors) {
        cout << "0" << endl;
        return 0;
    }
    cin >> m;
    sameColors = true;
    bool flag = true;
    
    for (ll i = 0; i < m; i++) {
        sameColors = false;
        flag = true;
        cin >> currentSoldier >> currentColor;
        soldiers[currentSoldier] = currentColor;
        if (i == 0) remColor = currentColor;
        if (currentColor == remColor) {
            for (ll j = 1; j <= n; j++) {
                if (soldiers[j] != remColor) {
                    flag = false;
                    break;
                }
            }
            if (flag) sameColors = true;
        }
        if (sameColors) {
            answer = i + 1;
            break;
        }
        remColor = currentColor;
    }
    
    if (sameColors) {
        cout << answer << endl;
    }
    else {
        cout << "-1" << endl;
    }
    
	return 0;
}
\end{lstlisting}

\newpage
\textbf{{\large Задача F - Два конверта}} \\
\begin{center}
\includegraphics[width=0.9\textwidth]{CT_SGAU/CT_SGAU_F.png}\\ [1cm]
\end{center}
\textbf{{\large Алгоритм}} \\
{\Huge ???????????????????} \\ 
\\
\textbf{{\large Исходный код}}
\begin{lstlisting}[language=C]
#include <iostream>

using namespace std;

int main() {
    long long b, c;
    cin >> b >> b >> c;
    if (c > b) cout << "Stay with this envelope" << endl;
    else cout << "Take another envelope" << endl;
    return 0;
}
\end{lstlisting}

\newpage
\textbf{{\large Задача G - Задача о размене монет}} \\
\begin{center}
\includegraphics[width=0.9\textwidth]{CT_SGAU/CT_SGAU_G.png}\\ [1cm]
\end{center}
\textbf{{\large Алгоритм}} \\
{\Huge ???????????????????} \\ 
\\
\newpage
\textbf{{\large Исходный код}}
\begin{lstlisting}[language=C]
#include <iostream>
#include <cmath>
#include <vector>
#include <algorithm>

using namespace std;

int main() {
    unsigned long long n, s, startIndex, coinsNeeded = 0;
    cin >> n >> s;
    vector<unsigned long long> k(n + 1);
    k[0] = 1;
    startIndex = n;
    for (unsigned long long i = 1; i <= n; i++) {
        cin >> k[i];
        if (k[i] * k[i - 1] > s) {
            startIndex = i - 1;
            break;
        }
        k[i] *= k[i - 1];
    }
    for (unsigned long long i = startIndex; ; i--) {
        coinsNeeded += s / k[i];
        s %= k[i];
        if (s == 0 || i == 0)
            break;
    }
    cout << coinsNeeded << endl;
    return 0;
}
\end{lstlisting}

\newpage
\textbf{{\large Результаты}} \\
\begin{center}
\includegraphics[width=0.95\textwidth]{CT_SGAU/CT_SGAU_result.png}\\ [1cm]
\end{center}



\newpage
\section{Журнал по личным контестам Якименко А.В.}


\end{document}