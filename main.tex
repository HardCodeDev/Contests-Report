%----------------------------------------------------------------------------------------
%	PACKAGES AND OTHER DOCUMENT CONFIGURATIONS
%----------------------------------------------------------------------------------------

\documentclass[a4paper,12pt]{article}
%\usepackage[english]{babel}
\usepackage{amsmath}
\usepackage{graphicx}
\usepackage[colorinlistoftodos]{todonotes}
\usepackage{fullpage}
\usepackage{multicol,multirow}
\usepackage{tabularx}
\usepackage{ulem}
\usepackage[utf8]{inputenc}
\usepackage[russian]{babel}
\usepackage{amsmath}
\usepackage{amssymb}
\usepackage{listings}
\usepackage{titlesec}

\usepackage{longtable}
\usepackage{ltxtable}

\usepackage{listings}
\usepackage{color}
 
\definecolor{codegreen}{rgb}{0,0.6,0}
\definecolor{codegray}{rgb}{0.5,0.5,0.5}
\definecolor{codepurple}{rgb}{0.58,0,0.82}
\definecolor{backcolour}{rgb}{0.95,0.95,0.92}
 
\lstdefinestyle{mystyle}{
    %backgroundcolor=\color{backcolour},   
    commentstyle=\color{codegreen},
    keywordstyle=\color{magenta},
    numberstyle=\tiny\color{codegray},
    stringstyle=\color{codepurple},
    basicstyle=\footnotesize,
    breakatwhitespace=false,         
    breaklines=true,                 
    captionpos=b,                    
    keepspaces=true,                 
    numbers=left,                    
    numbersep=5pt,                  
    showspaces=false,                
    showstringspaces=false,
    showtabs=false,                  
    tabsize=2
}
\lstset{style=mystyle}

\usepackage{hyperref}
\hypersetup{
    colorlinks=true,
    linkcolor=blue,
    filecolor=magenta,      
    urlcolor=blue,
}
 
\urlstyle{same}

\usepackage{geometry}
\geometry{top=2cm}
\geometry{bottom=2cm}
\geometry{left=1.5cm}
\geometry{right=1.5cm}


\begin{document}

\begin{titlepage}

\newcommand{\HRule}{\rule{\linewidth}{0.5mm}} % Defines a new command for the horizontal lines, change thickness here

\center % Center everything on the page
 
 
 
%----------------------------------------------------------------------------------------
%	HEADING SECTIONS
%----------------------------------------------------------------------------------------

\textsc{\large Московский Авиационный Институт\\(национальный исследовательский университет)}\\[1.5cm] % Name of your university/college



%----------------------------------------------------------------------------------------
%	LOGO SECTION
%----------------------------------------------------------------------------------------

\includegraphics[width=0.25\textwidth]{mai_logo.png}\\[1cm]
% Include a department/university logo - this will require the graphicx package
 
%----------------------------------------------------------------------------------------

\vspace{40px}

\textsc{\Large Отчет по индивидуальному учебному плану}\\[0.5cm] % Major heading such as course name
%\textsc{\large Алгоритмы на графах}\\[0.5cm] % Minor heading such as course title



%----------------------------------------------------------------------------------------
%	TITLE SECTION
%----------------------------------------------------------------------------------------

\HRule \\[0.4cm]
{ \huge \bfseries Алгоритмы на графах}\\[0.4cm] % Title of your document
\HRule \\[1.5cm]



%----------------------------------------------------------------------------------------
%	AUTHOR SECTION
%----------------------------------------------------------------------------------------

\begin{minipage}{0.4\textwidth}
\begin{flushleft} \large
\emph{Студенты:}\\
%John \textsc{Smith} % Your name
Макаров Никита\\Якименко Антон
\end{flushleft}
\end{minipage}
~
\begin{minipage}{0.4\textwidth}
\begin{flushright} \large
\emph{Руководитель:} \\
%Dr. James \textsc{Smith} % Supervisor's Name
Зайцев В.Е.
\end{flushright}
\end{minipage}\\[2cm]

%----------------------------------------------------------------------------------------
%	DATE SECTION
%----------------------------------------------------------------------------------------

%{\large \today}\\[2cm] % Date, change the \today to a set date if you want to be precise

\vfill % Fill the rest of the page with whitespace

\end{titlepage}



%----------------------------------------------------------------------------------------
%	СОДЕРЖАНИЕ
%----------------------------------------------------------------------------------------
\tableofcontents
\newpage



%----------------------------------------------------------------------------------------
%	ЛИЧНЫЕ ОТЧЕТЫ
%----------------------------------------------------------------------------------------
\section{Личные отчеты}

% Мой отчет
Отчет о работе студента Макарова Н.А. по индивидуальному учебному плану в V-VI семестрах 2014-2015 учебного года.

% Таблица - часть 1
\begin{table}[ht!]
\centering
\label{my-tab1}
\begin{tabular}{|c|c|c|c|c|c|c|}
\hline

% Заголовки столбцов
№ & 
Дата & 
Контест & 
\begin{tabular}[c]{@{}c@{}}Место\\ проведения\end{tabular} & 
\begin{tabular}[c]{@{}c@{}}Кол-во\\ участников\end{tabular} & 
\begin{tabular}[c]{@{}c@{}}Решено\\ задач\end{tabular} & 
\begin{tabular}[c]{@{}c@{}}Задач на\\ участника\end{tabular} \\ \hline

% Строки таблицы
1 & 18.09.2014 & Codeforces Round 267 Div 2 & Дом & 1 & 2 & 2 \\ \hline

2 & 21.09.2014 & Codeforces Round 268 Div 2 & Дом & 1 & 2 & 2 \\ \hline

3 & 22.09.2014 & \begin{tabular}[c]{@{}c@{}}Codeforces Отборочный контест\\ СГАУ на 1/4 ACM-ICPC\end{tabular} & Дом & 1 & 3 & 3 \\ \hline

4 & 25.09.2014 & Codeforces Training S02E03 & МАИ & 3 & 3 & 1 \\ \hline

5 & 28.09.2014 & Codeforces Round 270 Div 2 & Дом & 1 & 2 & 2 \\ \hline

6 & 02.10.2014 & Codeforces Training S02E04 & МАИ & 3 & 2 & 0.66 \\ \hline

7 & 05.10.2014 & \begin{tabular}[c]{@{}c@{}}XV Открытая Всесибирская\\ Олимпиада по\\ Программированию\end{tabular} & МАИ & 3 & 1 & 0.33 \\ \hline

8 & 09.10.2014 & Codeforces Training S02E05 & МАИ & 3 & 2 & 0.66 \\ \hline

9 & 16.10.2014 & Codeforces Round 273 Div 2 & Дом & 1 & 2 & 2 \\ \hline

10 & 17.10.2014 & Codeforces Training S02E06 & МАИ & 3 & 0 & 0 \\ \hline

11 & 18.10.2014 & \begin{tabular}[c]{@{}c@{}}Codeforces Тренировка СПбГУ\\ графы и DFS\end{tabular} & Дом & 3 & 2 & 0.66 \\ \hline

12 & 19.10.2014 & OpenCup GP of SPb. Div 2 & МАИ & 3 & 1 & 0.33 \\ \hline

13 & 20.10.2014 & Codeforces Round 274 Div 2 & Дом & 1 & 3 & 3 \\ \hline

14 & 23.10.2014 & \begin{tabular}[c]{@{}c@{}}Codeforces Самарский\\Аэрокосмический Лицей\\ тренировка №1\end{tabular} & Дом & 2 & 1 & 0.5 \\ \hline

15 & 23.10.2014 & \begin{tabular}[c]{@{}c@{}}Codeforces ACM, NEERC,\\ Восточный четвертьфинал\end{tabular} & Дом & 3 & 4 & 1.33 \\ \hline

16 & 24.10.2014 & Codeforces Round 275 Div 2 & Дом & 1 & 1 & 1 \\ \hline

17 & 25.10.2014 & \begin{tabular}[c]{@{}c@{}}Codeforces ACM, NEERC,\\ Южный четвертьфинал\end{tabular} & Дом & 3 & 3 & 1 \\ \hline

18 & 26.10.2014 & ACM-ICPC 1/4 Final & МГУ & 3 & 3 & 1 \\ \hline

19 & 30.10.2014 & Codeforces Training S02E07 & МАИ & 3 & 2 & 0.66 \\ \hline

20 & 01.11.2014 & Codeforces Crypto Cup & Дом & 3 & 9 & 3 \\ \hline

21 & 02.11.2014 & OpenCup GP of Siberia Div 2 & МАИ & 2 & 3 & 1.5 \\ \hline

22 & 06.11.2014 & Codeforces Training S02E08 & МАИ & 3 & 2 & 0.66 \\ \hline

23 & 13.11.2014 & Codeforces Training S02E09 & МАИ & 3 & 3 & 1 \\ \hline

24 & 15.11.2014 & \begin{tabular}[c]{@{}c@{}}Codeforces Олимпиада\\ школьников\\ Нижегородской области\end{tabular} & Дом & 2 & 3 & 1.5 \\ \hline

25 & 16.11.2014 & \begin{tabular}[c]{@{}c@{}}OpenCup GP\\ of Central Europe. Div 2\end{tabular} & МАИ & 3 & 1 & 0.33 \\ \hline

26 & 20.11.2014 & Codeforces Training S02E10 & МАИ & 3 & 3 & 1 \\ \hline

27 & 23.11.2014 & OpenCup GP of Europe Div 2 & МАИ & 3 & 5 & 1.66 \\ \hline

28 & 14.12.2014 & OpenCup GP of Peterhof Div 2 & МАИ & 3 & 1 & 0.33 \\ \hline

29 & 01.02.2015 & OpenCup GP of Japan Div 2 & МАИ & 3 & 4 & 1.33 \\ \hline

30 & 08.02.2015 & OpenCup Northern GP Div 2 & МАИ & 3 & 2 & 0.66 \\ \hline

31 & 15.02.2015 & OpenCup GP of Karelia Div 2 & МАИ & 3 & 4 & 1.33 \\ \hline

\end{tabular}
\end{table}

Продолжение таблицы.
% Таблица - часть 2
\begin{table}[ht!]
\centering
%\caption{My caption}
\label{my-tab2}
\begin{tabular}{|c|c|c|c|c|c|c|}
\hline

% Заголовки столбцов
№ & 
Дата & 
Контест & 
\begin{tabular}[c]{@{}c@{}}Место\\ проведения\end{tabular} & 
\begin{tabular}[c]{@{}c@{}}Кол-во\\ участников\end{tabular} & 
\begin{tabular}[c]{@{}c@{}}Решено\\ задач\end{tabular} & 
\begin{tabular}[c]{@{}c@{}}Задач на\\ участника\end{tabular} \\ \hline

% Строки таблицы
32 & 22.02.2015 & OpenCup GP of Udmurtia Div 2 & МАИ & 3 & 4 & 1.33 \\ \hline

33 & 01.03.2015 & OpenCup GP of China Div 2 & МАИ & 3 & 1 & 0.33 \\ \hline

34 & 07.03.2015 & VK Cup 2015 Квалификация & Дом & 1 & 2 & 2 \\ \hline

35 & 15.03.2015 & OpenCup GP of Tatarstan Div 2 & МАИ & 3 & 1 & 0.33 \\ \hline

36 & 21.03.2015 & VK Cup 2015 Раунд 1 & Дом & 1 & 2 & 2 \\ \hline

37 & 29.03.2015 & OpenCup Gp of America Div 2 & МАИ & 3 & 4 & 1.33 \\ \hline

38 & 18.04.2015 & Vekua Cup Личный этап & МФТИ-1С & 1 & 1 & 1 \\ \hline

39 & 19.04.2015 & Vekua Cup Командный этап & МФТИ-1С & 3 & 3 & 1 \\ \hline

40 & 26.04.2015 & OpenCup GP of Ural Div 2 & МАИ & 2 & 2 & 1 \\ \hline

41 & 31.05.2105 & Mail.ru RCC Квалификация & Дом & 1 & 1 & 1 \\ \hline

\end{tabular}
\end{table}

Итого: 41 контест, $\approx$49 решенных задач.
\newpage

% Отчет Антона

Отчет о работе студента Якименко А.В. по индивидуальному учебному плану в V-VI семестрах 2014-2015 учебного года.



\newpage
%----------------------------------------------------------------------------------------
%	КОМАНДНЫЕ КОНТЕСТЫ
%----------------------------------------------------------------------------------------
\section{Журнал по командным контестам}

%----------------------------------------------------------------------------------------
%
%	Codeforces Training S02E03
%
%----------------------------------------------------------------------------------------
\subsection{Codeforces Training S02E03}

\textbf{{\large Результаты}} \\
\begin{center}
\includegraphics[width=0.95\textwidth]{CT_S02E03/CT_S02E03_result.png}\\ [1cm]
\end{center}

\textbf{{\large Ссылка на контест: \url{http://codeforces.com/gym/100494}}}

\newpage
\textbf{{\large Задача B - Best Compression Ever}}

\begin{center}
\includegraphics[width=0.9\textwidth]{CT_S02E03/CT_S02E03_B.png}\\ [1cm]
\end{center}

\textbf{{\large Алгоритм}}

Решение довольно простое. Можно заметить, что если логарифм по основаню 2 числа n меньше или равен b, то ответ yes, иначе ответ no. Cложность $O(1)$.

\newpage
\textbf{{\large Исходный код}} \\
\begin{lstlisting}[language=C]
#include <iostream>
#include <cmath>

using namespace std;

int main() {
    ios_base::sync_with_stdio(false);
    
    unsigned long long n;
    int b;
    cin >> n >> b;
    
    if ((int)log2((double)n) <= b) {
        cout << "yes" << endl;
    }
    else {
        cout << "no" << endl;
    }

    return 0;
}
\end{lstlisting}


\newpage
\textbf{{\large Задача E - Event Planning}}

\begin{center}
\includegraphics[width=0.9\textwidth]{CT_S02E03/CT_S02E03_E.png}\\ [1cm]
\end{center}

\textbf{{\large Алгоритм}}

{\Huge ???????????????????}

\newpage
\textbf{{\large Исходный код}} \\
\begin{lstlisting}[language=C]
#include <iostream>
#include <vector>

#define ll long long
using namespace std;

int main () {
    ll N,B,H,W,p,a;
    ll min_cost = 5000000;

    cin >> N >> B >> H >> W;

    for (ll i = 0; i < H; i++) {
        cin >> p;
            for (ll k = 0; k < W; k++) {
                cin >> a;
                if((a >= N) && (p * N <= B) && (p * N <= min_cost))
                    min_cost = p * N;
            }
    }
    if(min_cost < 5000000)
        cout << min_cost << endl;
    else cout << "stay home" << endl;


    return 0;
}
\end{lstlisting}


\newpage
\textbf{{\large Задача K - Best Cow Line}}

\begin{center}
\includegraphics[width=0.9\textwidth]{CT_S02E03/CT_S02E03_K.png}\\ [1cm]
\end{center}

\textbf{{\large Алгоритм}}

Для решения этой задачи нужно чтобы построить последовательность букв по заданным правилам. Будем смотреть на первую и последнюю буквы и брать лексикографически наименьую. Если буквы совпадают, то надо посмотреть следующие буквы до первого несовпадения и взять букву с той стороны, с которой несовпавшая буква оказалась лексикографически меньше. Сложность $O(n)$.


\newpage
\textbf{{\large Исходный код}} \\
\begin{lstlisting}[language=C]
#include <iostream>
#include <cmath>
#include <vector>
#include <stack>
#include <algorithm>

using namespace std;

int main() {
    int n;
    cin >> n;
    vector<char> cows(n);
    char symb;
    
    for (int i = 0; i < n; i++) {
        cin >> symb;
        cows[i] = symb;
    }
    
    vector<char> newLine;
    int cowsInOldLine = n;
    int begin = 0;
    int end   = n - 1;
    int tempBegin = begin;
    int tempEnd   = end;
    
    while (cowsInOldLine) {
        tempBegin = begin;
        tempEnd   = end;
        if (cows[begin] == cows[end]) {
            while (cows[tempBegin] == cows[tempEnd]) {
                tempBegin++;
                tempEnd--;
                if (tempBegin > tempEnd || tempBegin == tempEnd) {
                    tempBegin = begin;
                    tempEnd = end;
                    break;
                }
            }
        }
        if (cows[tempBegin] < cows[tempEnd]) {
            newLine.push_back(cows[begin]);
            begin++;
        }
        else {
            newLine.push_back(cows[end]);
            end--;
        }
        cowsInOldLine--;
    }
    for (int i = 0; i < n; i++) {
        cout << newLine[i];
        if ((i + 1) % 80 == 0) cout << endl;
    }
    cout << endl;
    return 0;
}
\end{lstlisting}



%----------------------------------------------------------------------------------------
%
%	Codeforces Training S02E04
%
%----------------------------------------------------------------------------------------
\newpage
\subsection{Codeforces Training S02E04}

{\Huge у Антона}



%----------------------------------------------------------------------------------------
%
%	XV Открытая Всесибирская олимпиада по программированию им И.В. Поттосина
%
%----------------------------------------------------------------------------------------

\newpage
\subsection{XV Открытая Всесибирская олимпиада по программированию им И.В. Поттосина}

\textbf{{\large Результаты}} \\
\begin{center}
\includegraphics[width=0.95\textwidth]{Siberia/Siberia_result.png}\\ [1cm]
\end{center}

\textbf{{\large Ссылка на контест: \url{https://olympic.nsu.ru/nsuts-new/news.cgi}}}

\newpage
\textbf{{\large Задача 2 - Копировальный аппарат}}

\begin{center}
\includegraphics[width=0.9\textwidth]{Siberia/Siberia_1.png}\\ [1cm]
\includegraphics[width=0.6\textwidth]{Siberia/Siberia_2.png}\\ [1cm]
\end{center}

\textbf{{\large Алгоритм}}

Задача на реализацию. Нужно считать входные данные в двумерный массив символов, затем пройтись по всем элементам и запомнить наибольшие позиции, на которых находятся решетки.

\newpage
\textbf{{\large Исходный код}} \\
\begin{lstlisting}[language=C]
#include <iostream>
#include <algorithm>
#include <cmath>

#include <sstream>
#include <fstream>

#define LL  long long

using namespace std;

int main() {
    ifstream in;
    ofstream out;
    in.open("input.txt");
    out.open("output.txt");
    
    LL a, b;
    in >> a >> b;
    char pic[a + 1][b + 1];
    LL XMax = 0;
    LL YMax = 0;
    
    for (LL i = 1; i <= a; i++) {
        for (LL j = 1; j <= b; j++) {
            in >> pic[i][j];
            if (pic[i][j] == '#' && j > XMax) XMax = j;
            if (pic[i][j] == '#' && i > YMax) YMax = i;
        }
        in.get();
    }
    LL n, y, x;
    in >> n;
    for (LL i = 1; i <= n; i++) {
        in >> y >> x;
        if (y >= YMax && x >= XMax) {
            out << i << endl;
            return 0;
        }
    }
    in.close();
    out.close();
    
    return 0;
}
\end{lstlisting}



%----------------------------------------------------------------------------------------
%
%	Codeforces Training S02E05
%
%----------------------------------------------------------------------------------------

\newpage
\subsection{Codeforces Training S02E05}

\textbf{{\large Результаты}} \\
\begin{center}
\includegraphics[width=0.95\textwidth]{CT_S02E05/CT_S02E05_result.png}\\ [1cm]
\end{center}

\textbf{{\large Ссылка на контест: \url{http://codeforces.com/gym/100503}}}

\newpage
\textbf{{\large Задача A - Walking around Berhattan}}

\begin{center}
\includegraphics[width=0.9\textwidth]{CT_S02E05/CT_S02E05_A1.png}\\ [1cm]
\includegraphics[width=0.9\textwidth]{CT_S02E05/CT_S02E05_A2.png}\\ [1cm]
\end{center}

\textbf{{\large Алгоритм}}

Задача на реализацию. Нужно построить матрицу заданного размера и симулировать передвижение между элементами, записывая изменения. Сложность алгоритма $O(nm)$. \\

\textbf{{\large Исходный код}} \\
\begin{lstlisting}[language=C]
#include <iostream>
#include <algorithm>
#include <iomanip>
#include <vector>
#include <sstream>
#include <fstream>
#define LL  long long

using namespace std;

enum dtype {
    UP,
    DOWN,
    LEFT,
    RIGHT,
};

int main() {
    ifstream in;
    ofstream out;
    in.open("input.txt");
    out.open("output.txt");

    LL n, m;
    LL answer = 0;
    in >> n >> m;
    char t;
    in.get();
    
    vector< vector<int> > map(n + 2, vector<int>(m + 2, 0));
    vector< vector<bool> > used(n + 2, vector<bool>(m + 2, false));
    
    for (LL i = 1; i <= n; i++) {
        for (LL j = 1; j <= m; j++) {
            t = in.get();
            map[i][j] = t - '0';
        }
        in.get();
    }
    
    int x = 1, y = 1;
    dtype dir = RIGHT; // last dir

    while ((t = in.get()) != EOF) {
        if (t == 'M') {
            if (dir == RIGHT) {
 				answer += map[x][y];
                answer += map[x - 1][y];
            	if (!used[x][y]) {
                	map[x][y] /= 2;
                	used[x][y] = true;
                }
                if (!used[x - 1][y]) {
                	map[x - 1][y] /= 2;
                	used[x - 1][y] = true;
                }
                y++;
            }
            else if (dir == LEFT) {
            	answer += map[x][y - 1];
                answer += map[x - 1][y - 1];
            	if (!used[x][y - 1]) {
            		map[x][y - 1] /= 2;
            		used[x][y - 1] = true;
            	}
            	if (!used[x - 1][y - 1]) {
            		map[x - 1][y - 1] /= 2;
                	used[x - 1][y - 1] = true;
            	}
                y--;
            }
            else if (dir == UP) {
                answer += map[x - 1][y - 1];
                answer += map[x - 1][y];
                if (!used[x - 1][y - 1]) {
                	map[x - 1][y - 1] /= 2;
                	used[x - 1][y - 1] = true;
                }
                if (!used[x - 1][y]) {
                	map[x - 1][y] /= 2;
                	used[x - 1][y] = true;
                }
                x--;
            }
            else if (dir == DOWN) {
                answer += map[x][y];
                answer += map[x][y - 1];
                if (!used[x][y]) {
                	map[x][y] /= 2;
                	used[x][y] = true;
                }
                if (!used[x][y - 1]) {
                	map[x][y - 1] /= 2;
               		used[x][y - 1] = true;
                }
                x++;
            }
            
        }
        else if (t == 'R') {
            if (dir == UP) dir = RIGHT;
            else if (dir == DOWN) dir = LEFT;
            else if (dir == LEFT) dir = UP;
            else dir = DOWN;
        }
        else if (t == 'L'){
            if (dir == UP) dir = LEFT;
            else if (dir == DOWN) dir = RIGHT;
            else if (dir == LEFT) dir = DOWN;
            else dir = UP;
        }
    }
    out << answer << endl;
    in.close();
    out.close();
    return 0;
}
\end{lstlisting}


\newpage
\textbf{{\large Задача G - Plural Form of Nouns}}

\begin{center}
\includegraphics[width=0.9\textwidth]{CT_S02E05/CT_S02E05_G.png}\\ [1cm]
\end{center}

\textbf{{\large Алгоритм}}

Задача на реализацию. Нужно считать слова и в зависимости от окончания изменить его на нужное. Сложность $O(n)$.

\newpage
\textbf{{\large Исходный код}} \\
\begin{lstlisting}[language=C]
#include <iostream>
#include <sstream>
#include <fstream>
#define LL  long long

using namespace std;

int main() {
    ifstream in;
    ofstream out;
    in.open("input.txt");
    out.open("output.txt");

    LL n;
    string s;
    
    in >> n;
    
    for (LL i = 0; i < n; i++) {
        in >> s;
        size_t l = s.size() - 1;
        if ((s[l] == 'h' && s[l - 1] == 'c') || s[l] == 's' || s[l] == 'x' || s[l] == 'o') {
            out << s;
            out << "es" << endl;
        }
        else if (s[l] == 'f') {
            for (size_t j = 0; j < l; j++) {
                out << s[j];
            }
            out << "ves" << endl;
        }
        else if (s[l] == 'e' && s[l - 1] == 'f') {
            for (size_t j = 0; j < l - 1; j++) {
                out << s[j];
            }
            out << "ves" << endl;
        }
        else if (s[l] == 'y') {
            for (size_t j = 0; j < l; j++) {
                out << s[j];
            }
            out << "ies" << endl;
        }
        else {
            out << s << "s" << endl;
        }
    }
    
    in.close();
    out.close();
    
    return 0;
}
\end{lstlisting}



%----------------------------------------------------------------------------------------
%
%	Codeforces Training S02E06
%
%----------------------------------------------------------------------------------------
\newpage
\subsection{Codeforces Training S02E06}

\textbf{{\large Результаты}} \\
\begin{center}
\includegraphics[width=0.95\textwidth]{CT_S02E06/CT_S02E06_result.png}\\ [1cm]
\end{center}



%----------------------------------------------------------------------------------------
%
%	Тренировка СПбГУ B #3 Поиск кратчайшего пути и DFS
%
%----------------------------------------------------------------------------------------
\newpage
\subsection{Тренировка СПбГУ Поиск кратчайшего пути и DFS}

\textbf{{\large Результаты}} \\
\begin{center}
\includegraphics[width=0.95\textwidth]{SPBGU_GRAPHS/SPBGU_GRAPHS_result.png}\\ [1cm]
\end{center}

\newpage
\textbf{{\large Задача C - Флойд}} \\
\begin{center}
\includegraphics[width=0.9\textwidth]{SPBGU_GRAPHS/SPBGU_GRAPHS_C.png}\\ [1cm]
\end{center}
\textbf{{\large Алгоритм}} \\
В задаче требуется найти кратчайшие пути между всеми парами вершин и представить их матрицей смежности. Ее можно решить используя алгоритм Флойда-Уоршелла за $O(n^3)$. \\ 
\\
\newpage
\textbf{{\large Исходный код}}
\begin{lstlisting}[language=C]
#include <iomanip>
#include <iostream>
#include <algorithm>
#include <fstream>

using namespace std;

int main() {
    
    more_speed
    ifstream in("floyd.in");
    ofstream out("floyd.out");
    
    int n;
    in >> n;
    vector<vector<int> > m(n, vector<int>(n, 0));
    for (int i = 0; i < n; i++) {
        for (int j = 0; j < n; j++) {
            in >> m[i][j];
        }
    }
    
    for (int k = 0; k < n; k++) {
        for (int i = 0; i < n; i++) {
            for (int j = 0; j < n; j++) {
                m[i][j] = min(m[i][j], m[i][k] + m[k][j]);
            }
        }
    }
    
    for (int i = 0; i < n; i++) {
        for (int j = 0; j < n; j++) {
            out << m[i][j];
            if (j < n - 1) out << " ";
            else out << endl;
        }
    }

    in.close();
    out.close();
    
    return 0;
}
\end{lstlisting}


\newpage
\textbf{{\large Задача D - Поиск цикла}} \\
\begin{center}
\includegraphics[width=0.9\textwidth]{SPBGU_GRAPHS/SPBGU_GRAPHS_D.png}\\ [1cm]
\end{center}
\textbf{{\large Алгоритм}} \\
Задача решается поиском в глубину. Нужно сделать серию поисков в глубину, заходя в новую вершину будем красить ее в серый цвет, а выходя в черный. Если заходим в серую вершину, то цикл найден. \\
%\newpage
\textbf{{\large Исходный код}} \\
\begin{lstlisting}[language=C]
#include <iomanip>
#include <iostream>
#include <algorithm>
#include <fstream>

vector<set<LL> > g;
vector<char> color;
vector<LL> p;
LL cycle_st, cycle_end;

bool dfs (LL v) {
    color[v] = 1;
    for (set<LL>::iterator i = g[v].begin(); i != g[v].end(); i++) {
        LL to = *i;
        if (color[to] == 0) {
            p[to] = v;
            if (dfs(to)) return true;
        }
        else if (color[to] == 1){
            cycle_st = to;
            cycle_end = v;
            return true;
        }
    }
    color[v] = 2;
    return false;
}

using namespace std;
int main() {
    
    more_speed
    ifstream in("cycle.in");
    ofstream out("cycle.out");
    
    LL n, m, f, t;
    in >> n >> m;
    g.resize(n);
    
    for (LL i = 0; i < m; i++) {
        in >> f >> t;
        g[f - 1].insert(t - 1);
    }
    
    p.assign(n, -1);
    color.assign(n, 0);
    cycle_st = -1;
    for (LL i = 0; i < n; i++) {
        if (dfs(i)) break;
    }
    if (cycle_st == -1) {
        out << "NO" << endl;
    }
    else {
        out << "YES" << endl;
        vector<LL> cycle;
        for (LL v = cycle_end; v != cycle_st; v = p[v]) {
            cycle.push_back(v);
        }
        cycle.push_back(cycle_st);
        reverse(cycle.begin(), cycle.end());
        for (size_t i = 0; i < cycle.size(); i++) {
            out << cycle[i] + 1 << " ";
        }
        out << endl;
    }
    in.close();
    out.close();
    
    return 0;
}
\end{lstlisting}



%----------------------------------------------------------------------------------------
%
%	OpenCup GP of SPb
%
%----------------------------------------------------------------------------------------
\newpage
\subsection{OpenCup GrandPrix of SPb.}

\textbf{{\large Задача А - Барабашка}} \\
\begin{center}
\includegraphics[width=0.9\textwidth]{OC_SPB/OC_SPB_A1.png}\\ [1cm]
\includegraphics[width=0.9\textwidth]{OC_SPB/OC_SPB_A2.png}\\ [1cm]
\end{center}
\newpage

\textbf{{\large Алгоритм}} \\
Задача на реализацию. Нужно считать строки и каждой строке сопоставить правильное сочетание цвета и предмета по заданым в условии правилам. Сначала определим, какие сочетания уже имеются в предложении, затем проверим, есть ли среди них корректные, если есть, то это ответ, иначе нужно выбрать любое правильное сочетание. \\ 
\\
%\newpage
\textbf{{\large Исходный код}}
\begin{lstlisting}[language=C++]
#include <iomanip>
#include <iostream>
#include <algorithm>

bool isChar(char c) {
    return (c >= 'A' && c <= 'Z') || (c >= 'a' && c <= 'z');
}

int num(string s) {
    if      (s == "white" || s == "barabashka") return 0;
    else if (s == "blue"  || s == "book")       return 1;
    else if (s == "red"   || s == "chair")      return 2;
    else if (s == "gray"  || s == "mouse")      return 3;
    else return 4;
}
 
int main() {
    ifstream in("barabashka.in");
    ofstream out("barabashka.out");
    
    string white = "white",
           blue  = "blue",
           red   = "red",
           gray  = "gray",
           green = "green";
    string barab = "barabashka",
           book  = "book",
           chair = "chair",
           mouse = "mouse",
           bottle = "bottle";
    
    string current;
    string firstColor, firstObject;
    string secondColor, secondObject;
    bool needFirstColor, needFirstObject;
    bool needSecondColor, needSecondObject;
    bool complete;
    
    
    for (int i = 0; i < 5; i++) {
        char cSymb = in.get();
        bool used[5] = {false};
        needFirstColor   = true;
        needFirstObject  = false;
        needSecondColor  = false;
        needSecondObject = false;
        complete         = false;
        
        while (cSymb != '0') {
            while (cSymb != ' ' && isChar(cSymb)) {
                current += tolower(cSymb);
                cSymb = in.get();
            }
            if (needFirstColor) {
                if (current == white)      { firstColor = white; needFirstObject = true; needFirstColor = false; }
                else if (current == blue)  { firstColor = blue;  needFirstObject = true; needFirstColor = false; }
                else if (current == red)   { firstColor = red;   needFirstObject = true; needFirstColor = false; }
                else if (current == gray)  { firstColor = gray;  needFirstObject = true; needFirstColor = false; }
                else if (current == green) { firstColor = green; needFirstObject = true; needFirstColor = false; }
            }
            else if (needFirstObject) {
                if (current == barab)       { firstObject = barab;  needSecondColor = true; needFirstObject = false; }
                else if (current == book)   { firstObject = book;   needSecondColor = true; needFirstObject = false; }
                else if (current == chair)  { firstObject = chair;  needSecondColor = true; needFirstObject = false; }
                else if (current == mouse)  { firstObject = mouse;  needSecondColor = true; needFirstObject = false; }
                else if (current == bottle) { firstObject = bottle; needSecondColor = true; needFirstObject = false; }
            }
            else if (needSecondColor) {
                if (current == white)      { secondColor = white; needSecondObject = true; needSecondColor = false; }
                else if (current == blue)  { secondColor = blue;  needSecondObject = true; needSecondColor = false; }
                else if (current == red)   { secondColor = red;   needSecondObject = true; needSecondColor = false; }
                else if (current == gray)  { secondColor = gray;  needSecondObject = true; needSecondColor = false; }
                else if (current == green) { secondColor = green; needSecondObject = true; needSecondColor = false; }
            }
            else if (needSecondObject) {
                if (current == barab)       { secondObject = barab;  complete = true; needSecondObject = false; }
                else if (current == book)   { secondObject = book;   complete = true; needSecondObject = false; }
                else if (current == chair)  { secondObject = chair;  complete = true; needSecondObject = false; }
                else if (current == mouse)  { secondObject = mouse;  complete = true; needSecondObject = false; }
                else if (current == bottle) { secondObject = bottle; complete = true; needSecondObject = false; }
            }
            if (complete) {
                cSymb = '0';
                used[num(firstColor)] = true;
                used[num(firstObject)] = true;
                used[num(secondColor)] = true;
                used[num(secondObject)] = true;
                
                if ((firstColor == white && firstObject == barab) ||
                    (secondColor == white && secondObject == barab)) { out << white << " " << "Barabashka" << endl; }
                
                else if ((firstColor == blue && firstObject == book) ||
                         (secondColor == blue && secondObject == book)) { out << blue << " " << book << endl; }
                
                else if ((firstColor == red && firstObject == chair) ||
                         (secondColor == red && secondObject == chair)) { out << red << " " << chair << endl; }
                
                else if ((firstColor == gray && firstObject == mouse) ||
                         (secondColor == gray && secondObject == mouse)) { out << gray << " " << mouse << endl; }
                
                else if ((firstColor == green && firstObject == bottle) ||
                         (secondColor == green && secondObject == bottle)) { out << green << " " << bottle << endl; }
                
                else {
                    for (int j = 0; j < 5; j++) {
                        if (!used[j]) {
                            if (j == 0) { out << white << " " << "Barabashka" << endl; }
                            else if (j == 1) { out << blue << " " << book << endl; }
                            else if (j == 2) { out << red << " " << chair << endl; }
                            else if (j == 3) { out << gray << " " << mouse << endl; }
                            else             { out << green << " " << bottle << endl; }
                            break;
                        }
                    }
                }
                
                
            }
            current.clear();
            if (cSymb != '\0') cSymb = in.get();
        }
    }
    return 0;
}
\end{lstlisting}

\textbf{{\large Результаты}} \\
\begin{center}
\includegraphics[width=0.95\textwidth]{OC_SPB/OC_SPB_result.png}\\ [1cm]
\end{center}



%----------------------------------------------------------------------------------------
%
%	Самарский Международный Аэрокосмический Лицей, тренировка №1
%
%----------------------------------------------------------------------------------------
\newpage
\subsection{Самарский Международный Аэрокосмический Лицей, тренировка №1}

\textbf{{\large Задача D - Пивной вор}} \\
\begin{center}
\includegraphics[width=0.9\textwidth]{CT_SAMARA/CT_SAMARA_D.png}\\ [1cm]
\end{center}
\textbf{{\large Алгоритм}} \\
В этой задаче нужно отсортировать массив стоимостей по убыванию и сложить первые $k$ чисел. Это и будет ответом. \\ 
\\
\newpage
\textbf{{\large Исходный код}}
\begin{lstlisting}[language=C]
#include <iostream>
#include <vector>
#include <algorithm>
#include <fstream>

using namespace std;

int main() {
	long n, k;
	ifstream in("input.txt");
	ofstream out("output.txt");
	in >> n >> k;
	vector<long long> v(n);
	for (long i = 0; i < n; i++) {
		in >> v[i];
	}
	sort(v.begin(), v.end(), greater<long long>());
	long long answer = 0;
	for (long i = 0; i < k && i < v.size(); i++) {
		answer += v[i];
	}
	out << answer << endl;
	in.close();
	out.close();
	return 0;
}
\end{lstlisting}

\textbf{{\large Результаты}} \\
\begin{center}
\includegraphics[width=0.95\textwidth]{CT_SAMARA/CT_SAMARA_result.png}\\ [1cm]
\end{center}



%----------------------------------------------------------------------------------------
%
%	Codeforces ACM Восточный четвертьфинал
%
%----------------------------------------------------------------------------------------
\newpage
\subsection{Codeforces ACM-ICPC Восточный четвертьфинал}

\textbf{{\large Задача A - About Grisha N.}} \\
\begin{center}
\includegraphics[width=0.9\textwidth]{CT_ACM_EAST/CT_ACM_EAST_A.png}\\ [1cm]
\end{center}
\textbf{{\large Алгоритм}} \\
Для решения задачи достаточно увидеть закономерность, что если $f > 6$, то ответ будет $YES$, иначе ответ будет $NO$. Сложность, очевидно, $O(1)$.\\ 
\\
\textbf{{\large Исходный код}}
\begin{lstlisting}[language=C]
#include <iostream>
using namespace std;

int main(int argc, const char * argv[]) {
    int f;
    cin >> f;
    if(f>6)
        cout << "YES";
    else
        cout << "NO";
    return 0;
}
\end{lstlisting}

\newpage
\textbf{{\large Задача D - Zhenya moves from the dormitory}} \\
\begin{center}
\includegraphics[width=0.9\textwidth]{CT_ACM_EAST/CT_ACM_EAST_D1.png}\\ [1cm]
\includegraphics[width=0.5\textwidth]{CT_ACM_EAST/CT_ACM_EAST_D2.png}\\ [1cm]
\end{center}
\textbf{{\large Алгоритм}} \\
Для решения задачи сразу при считывании найдем лучшеие однокомнатные и двухкомнатные квартиры по комфорту, если Женя будет жить один. Затем отсортируем все квартиры по комфорту и для каждого друга Жени будем подбирать лучший вариант при совместной покупке. После этого сравним найденные варианты и выберем лучший. Сложность алгоритма $O(n^2)$.\\
\\
\newpage
\textbf{{\large Исходный код}}
\begin{lstlisting}[language=C]
#include <iostream>
#include <algorithm>
#include <vector>
#include <fstream>

using namespace std;

typedef struct {
    long money;
    long ad;
    int num;
}frd;

typedef struct {
    long rooms;
    long price;
    long ad;
    int num;
}apartment;

bool cmp_ap(const apartment &a1, const apartment &a2) {
    return a1.ad > a2.ad;
}

bool cmp_fr(const frd &f1, const frd &f2) {
    return f1.ad > f2.ad;
}

int find_good_ap(const vector<apartment> &a, long money1, long money2) {
    for (int i = 0; i < a.size(); i++) {
        long price_for_each = a[i].price / 2;
        if (a[i].price % 2 == 1) {
            price_for_each++;
        }
        if (money1 >= price_for_each && money2 >= price_for_each && a[i].rooms == 2)
            return i;
    }
    return -1;
}

int main() {
    long money, ad1, ad2;
    long n, m;
    cin >> money >> ad1 >> ad2;
    cin >> n;
    vector<frd> fr(n);
    for (int i = 0; i < n; i++) {
        cin >> fr[i].money >> fr[i].ad;
        fr[i].num = i + 1;
    }
    cin >> m;
    vector<apartment> ap(m);
    long max_ad_in_1room = -1, max_ad_in_2room = -1;
    int  in_1room_num = -1, in_2room_num = -1;
    bool can_buy_alone = false;
    for (int i = 0; i < m; i++) {
        cin >> ap[i].rooms >> ap[i].price >> ap[i].ad;
        ap[i].num = i + 1;
        if (ap[i].rooms == 1) {
            if (money >= ap[i].price && ad1 + ap[i].ad > max_ad_in_1room) {
                max_ad_in_1room = ad1 + ap[i].ad;
                in_1room_num = i + 1;
            }
        }
        else {
            if (money >= ap[i].price && ad2 + ap[i].ad > max_ad_in_2room) {
                max_ad_in_2room = ad2 + ap[i].ad;
                in_2room_num = i + 1;
            }
        }
    }
    sort(ap.begin(), ap.end(), cmp_ap);
    int ans_ap = -1, ans_fr = -1;
    long max_ad_tog = -1;
    int found_ap = -1;
    for (int i = 0; i < n; i++) {
        found_ap = find_good_ap(ap, money, fr[i].money);
        if (found_ap != -1) {
            long cur_ad = fr[i].ad + ap[found_ap].ad;
            if (cur_ad > max_ad_tog) {
                ans_ap = ap[found_ap].num;
                ans_fr = fr[i].num;
                max_ad_tog = cur_ad;
            }
        }
    }
    
    long alone = 0, whereAlone = 0;
    if (max_ad_in_1room != -1 || max_ad_in_2room != -1) {
        if (max_ad_in_1room > max_ad_in_2room) {
            alone = max_ad_in_1room;
            whereAlone = in_1room_num;
        }
        else {
            alone = max_ad_in_2room;
            whereAlone = in_2room_num;
        }
        can_buy_alone = true;
    }
    
    if (found_ap == -1 && !can_buy_alone) {
        cout << "Forget about apartments. Live in the dormitory." << endl;
        return 0;
    }
    
    if (alone > max_ad_tog)
        cout << "You should rent the apartment #" << whereAlone << " alone." << endl;
    else
        cout << "You should rent the apartment #" << ans_ap << " with the friend #" << ans_fr << "." << endl;
        
    return 0;
}
\end{lstlisting}

\newpage
\textbf{{\large Задача I - Traffic Jam in Flower Town}} \\
\begin{center}
\includegraphics[width=0.9\textwidth]{CT_ACM_EAST/CT_ACM_EAST_I.png}\\ [1cm]
\end{center}
\textbf{{\large Алгоритм}} \\
{\Huge ???????????????????} \\ 
\\
\newpage
\textbf{{\large Исходный код}}
\begin{lstlisting}[language=C]
#include <iostream>
#include <deque>

using namespace std;

int main() 
{
    deque <char> south;
    deque <char> north;

    char temp;
    int time = 0;
    temp = cin.get();
    while(temp != '\n'){
        south.push_back(temp);
        temp = cin.get();
    }
    temp = cin.get();
    while(temp != '\n'){
        north.push_back(temp);
        temp = cin.get();
    }
    char s, n;
    while(south.size() > 0 && north.size() > 0){
        s = south[0];
        n = north[0];
        time++;
        if(s == 'R' && n == 'L')
            south.pop_front();
        else if(s == 'L' && n == 'R')
            north.pop_front();
         else if(s == 'L' && n == 'F')
            north.pop_front();
         else if(s == 'F' && n == 'L')
            south.pop_front();
         else {
            south.pop_front();
            north.pop_front();
         }
    }
    
    if(south.size() > 0)
        time += south.size();
    if(north.size() > 0)
        time += north.size();

    cout << time << endl;
    return 0;
}
\end{lstlisting}

\newpage
\textbf{{\large Задача L - Donald is a postman}} \\
\begin{center}
\includegraphics[width=0.9\textwidth]{CT_ACM_EAST/CT_ACM_EAST_L.png}\\ [1cm]
\end{center}
\textbf{{\large Алгоритм}} \\
Задача на реализацию. Нужно симулировать перемещение между состояниями в зависимости от первой буквы имени. Сложность $O(n)$.\\ 
\\
\newpage
\textbf{{\large Исходный код}}
\begin{lstlisting}[language=C]
#include <iostream>

using namespace std;

int main() 
{
    int n;
    cin >> n;
    string name;
    int state = 1;
    int answer = 0;
    for (int i = 0; i < n; i++) {
        cin >> name;
        char t = name[0];
        if (state == 1) {
            if (t == 'B' || t == 'M' || t == 'S') {
                answer++;
                state = 2;
            }
            else if (t == 'D' || t == 'J' || t == 'K' || t == 'T' || t == 'W' || t == 'G') {
                answer += 2;
                state = 3;
            }
        }
        else if (state == 2) {
            if (t == 'A' || t == 'P' || t == 'O' || t == 'R') {
                answer++;
                state = 1;
            }
            else if (t == 'D' || t == 'J' || t == 'K' || t == 'T' || t == 'W' || t == 'G') {
                answer++;
                state = 3;
            }
                
        }
        else {
            if (t == 'A' || t == 'P' || t == 'O' || t == 'R') {
                answer += 2;
                state = 1;
            }
            else if (t == 'B' || t == 'M' || t == 'S') {
                answer++;
                state = 2;
            }
        }
    }
    
    cout << answer << endl;
    return 0;
}
\end{lstlisting}

\newpage
\textbf{{\large Результаты}} \\
\begin{center}
\includegraphics[width=0.95\textwidth]{CT_ACM_EAST/CT_ACM_EAST_result.png}\\ [1cm]
\end{center}



%----------------------------------------------------------------------------------------
%
%	Codeforces ACM Южный четвертьфинал
%
%----------------------------------------------------------------------------------------
\newpage
\subsection{Codeforces ACM-ICPC Южный четвертьфинал}

\textbf{{\large Задача D - Data Center}} \\
\begin{center}
\includegraphics[width=0.9\textwidth]{CT_ACM_WEST/CT_ACM_WEST_D.png}\\ [1cm]
\end{center}
\textbf{{\large Алгоритм}} \\
Сначала отсортируем сервера по объему памяти. Наберем необходимое количество серверов и сравним набранную память с минимальным объемом. Если разница равна $0$, то ответ найден. Иначе будем пытаться поменять сервера с высоким напряжением на сервера с низким напряжением. Сложность $O(n^2)$.\\ 
\\
\newpage
\textbf{{\large Исходный код}}
\begin{lstlisting}[language=C]
#include <iostream>
#include <vector>
#include <algorithm>
#include <fstream>

using namespace std;

typedef struct {
    LL num;
    ULL cap;
    short low;
} server;

bool cmp_low(const server &a, const server &b) {
    return a.low > b.low;
}

bool cmp_cap(const server &a, const server &b) {
    if (a.cap == b.cap) return a.low > b.low;
    return a.cap > b.cap;
}

int main() {
    LL n;
    ULL m;
    cin >> n >> m;
    vector<server> s(n);
    for (LL i = 0; i < n; i++) {
        s[i].num = i + 1;
        cin >> s[i].cap;
        cin >> s[i].low;
    }
    
    sort(s.begin(), s.end(), cmp_cap);
    LL ind = 0;
    ULL curr_cap = 0;
    LL count = 0;

    while (curr_cap < m) {
        curr_cap += s[ind].cap;
        if (s[ind].low) count++;
        ind++;
    }
    
    ULL rem = curr_cap - m;
    
    if (rem == 0) {
        cout << ind << " ";
        cout << count << endl;
        for (LL i = 0; i < ind; i++) {
            cout << s[i].num << " ";
        }
        cout << endl;
        return 0;
    }
    
    for (LL i = ind - 1; i >= 0; i--) {
        if (!s[i].low && rem > 0) {
            bool azaza = false;
            for (LL j = ind; j < n; j++) {
                if (s[j].low && (s[j].cap + rem >= s[i].cap)) {
                    rem -= s[i].cap - s[j].cap;
                    swap(s[i], s[j]);
                    count++;
                    azaza = true;
                    break;
                }
            }
            if (!azaza) break;
        }
        else if (rem == 0) break;
    }
    
    cout << ind << " " << count << endl;
    for (LL i = 0; i < ind; i++) {
        cout << s[i].num << " ";
    }
    cout << endl;;
	return 0;
}
\end{lstlisting}

\newpage
\textbf{{\large Задача I - Sales in GameStore}} \\
\begin{center}
\includegraphics[width=0.9\textwidth]{CT_ACM_WEST/CT_ACM_WEST_I.png}\\ [1cm]
\end{center}
\textbf{{\large Алгоритм}} \\
{\Huge ???????????????????} \\ 
\\
\newpage
\textbf{{\large Исходный код}}
\begin{lstlisting}[language=C]
#include <iostream>
#include <vector>
#include <algorithm>
using namespace std;
bool compare(const int &a, const int &b)
{
    return a<b;
}
int main(int argc, const char * argv[]) {
    vector<int> p(2001);
    int n;
    cin >> n;
    for(int i=0; i<n; ++i)
        cin >> p[i];
    sort(p.begin(), p.begin()+n);
    int sum = 0;
    int i=0;
    while(i<n-1 && sum+p[i]<=p[n-1])
        sum += p[i++];
    cout << i+1;
    return 0;
}
\end{lstlisting}

\newpage
\textbf{{\large Задача M - Variable Shadowing}} \\
\begin{center}
\includegraphics[width=0.9\textwidth]{CT_ACM_WEST/CT_ACM_WEST_M1.png}\\ [1cm]
\includegraphics[width=0.9\textwidth]{CT_ACM_WEST/CT_ACM_WEST_M2.png}\\ [1cm]
\end{center}
\textbf{{\large Алгоритм}} \\
Задачу можно решить с помощью стека и вектора стеков. Предупреждение нужно выводить когда новая переменная пытается попасть в стек, который не пуст, это значит уже была объявлена переменная с таким же именем. Когда появляется закрывающая скобка убираем все до открывающей. \\ 
\\
\newpage
\textbf{{\large Исходный код}}
\begin{lstlisting}[language=C]
#include <iostream>
#include <vector>
#include <algorithm>
using namespace std;
typedef struct {
    char ch;
    int line;
    int sym;
} var;
int main() {
    int f,n;
    cin >> n;
    vector <stack <var> > all_alph(26);
    stack <var> curr;
    char temp;
    int symbol = 0;
    var to_put;
    temp = cin.get();
    for (int i = 1; i <= n; i++){
        temp = cin.get();
        symbol = 1;
        while(temp != '\n') {
            if(temp == '}') {
                to_put = curr.top();
                curr.pop();
                while (to_put.ch != '{') {
                    all_alph[to_put.ch - 97].pop();
                    to_put = curr.top();
                    curr.pop();
                }
            }
            else if(temp == '{') {
                to_put.ch = temp;
                to_put.sym = symbol;
                to_put.line = i;
                curr.push(to_put);
            }
            else if (temp != ' '){
                to_put.sym = symbol;
                to_put.line = i;
                to_put.ch = temp;
                curr.push(to_put);
                if(all_alph[temp - 97].size() != 0)
                    cout << to_put.line << ":" << to_put.sym 
                    << ": warning: shadowed declaration of "<< to_put.ch 
                    << ", the shadowed position is " << all_alph[temp - 97].top().line
                    << ":" << all_alph[temp - 97].top().sym << endl;
                all_alph[temp - 97].push(to_put);
            }
            
            temp = cin.get();
            symbol++;
        }
    }
    return 0;
}
\end{lstlisting}

\newpage
\textbf{{\large Результаты}} \\
\begin{center}
\includegraphics[width=0.95\textwidth]{CT_ACM_WEST/CT_ACM_WEST_result.png}\\ [1cm]
\end{center}



%----------------------------------------------------------------------------------------
%
%	ACM 1/4 Final
%
%----------------------------------------------------------------------------------------
\newpage
\subsection{ACM-ICPC Московский четвертьфинал}

Так как соревнование проводилось в МГУ, то турнирная таблица с результатами и исходные коды программ не доступны. \\



%----------------------------------------------------------------------------------------
%
%	Codeforces Training S02E07
%
%----------------------------------------------------------------------------------------
\newpage
\subsection{Codeforces Training S02E07}

\textbf{{\large Задача C - Will It Stop?}} \\
\begin{center}
\includegraphics[width=0.9\textwidth]{CT_S02E07/CT_S02E07_C.png}\\ [1cm]
\end{center}
\textbf{{\large Алгоритм}} \\
{\Huge ???????????????????} \\ 

\textbf{{\large Исходный код}}
\begin{lstlisting}[language=C]
#include <iostream>
using namespace std;
int main()
{
    unsigned long long a;
    cin >> a;
    while(!(a%2))a/=2;
    if(a==1)
        cout << "TAK";
    else
        cout << "NIE";
    return 0;
}
\end{lstlisting}

\newpage
\textbf{{\large Задача H - Afternoon Tea}} \\
\begin{center}
\includegraphics[width=0.9\textwidth]{CT_S02E07/CT_S02E07_H.png}\\ [1cm]
\end{center}
\textbf{{\large Алгоритм}} \\
{\Huge ???????????????????} \\ 

\textbf{{\large Исходный код}}
\begin{lstlisting}[language=C]
#include <iostream>
#include <cmath>
using namespace std;
int main()
{
    int n;
    cin >> n;
    if(n==1)
    {
        cout << "HM";
        return 0;
    }
    cin.get();
    char c;
    long double hDrunked = 0, mDrunked = 0;
    hDrunked = mDrunked = (1-pow(0.5, n))*0.5;
    for(int i=0; i<n-1; ++i)
    {
        c=cin.get();
        if(c=='H')
            hDrunked += (1-pow(0.5, n-i-1))*0.5;
        else
            mDrunked += (1-pow(0.5, n-i-1))*0.5;
    }
    if(hDrunked>mDrunked)
        cout << "H";
    else if(hDrunked<mDrunked)
        cout << "M";
    return 0;
}
\end{lstlisting}

\textbf{{\large Результаты}} \\
\begin{center}
\includegraphics[width=0.95\textwidth]{CT_S02E07/CT_S02E07_result.png}\\ [1cm]
\end{center}



%----------------------------------------------------------------------------------------
%
%	Codeforces Crypto Cup
%
%----------------------------------------------------------------------------------------
\newpage
\subsection{Codeforces Crypto Cup 1.0}

\textbf{{\large Задача B - :-P}} \\
\begin{center}
\includegraphics[width=0.9\textwidth]{CT_Crypto/CT_Crypto_B.png}\\ [1cm]
\end{center}

\textbf{{\large Исходный код}}
\begin{lstlisting}[language=C]
#include <iostream>
#include <cmath>
#include <vector>
using namespace std;
int main(int argc, const char * argv[]) {
    char str[100001];
    long p, len;
    while((str[len]=cin.get())!='\n') ++len;
    str[len] = '\0';
    cin >> p;
    vector< vector <char> > v(p);
    long vSize = len/p, r = len%p;
    for(long i=r; i<p; ++i) {
        v[i].resize(vSize);
    }
    for(long i=0; i<r; ++i) {
        v[i].resize(vSize+1);
    }
    for(long i=0, k=0; i<p; ++i) {
        for(long j=0; j<v[i].size(); ++j, ++k) {
            v[i][j] = str[k];
        }
    }
    for(long i=0; i<len; ++i) {
        cout.put(v[i%p][i/p]);
    }
    return 0;
}
\end{lstlisting}

\newpage
\textbf{{\large Задача C - Pgkpxumgs}} \\
\begin{center}
\includegraphics[width=0.9\textwidth]{CT_Crypto/CT_Crypto_C.png}\\ [1cm]
\end{center}

\textbf{{\large Исходный код}}
\begin{lstlisting}[language=C]
#include <iostream>
#include <cstdio>
using namespace std;
int main() {
	char cur, prev;
	cout.put(prev = cin.get());
	while ((cur = cin.get()) != '\n') {
		if ((int)(cur - prev) < 0) cout << (char)(cur - prev + '{');
		else cout << (char)(cur - prev + 'a');
		prev = cur;
 	}
	return 0;
}
\end{lstlisting}

\newpage
\textbf{{\large Задача H - Peace of AmericaReunion}} \\
\begin{center}
\includegraphics[width=0.9\textwidth]{CT_Crypto/CT_Crypto_H.png}\\ [1cm]
\end{center}

\textbf{{\large Исходный код}}
\begin{lstlisting}[language=C]
#include <iostream>
using namespace std;
int main() {
	vector<long> v(26);
	long length = 0;
	for (int i = 0; i < 26; i++) {
		cin >> v[i];
		length += v[i];
	}
	vector<char> answer(length);
	long pos;
	int nextSymb = -1;
	for (int i = 0; i < 26; i++) {
		if (v[i]) {
			nextSymb = i;
			break;
		}
	}
	for (long i = 0; i < length; i++) {
		cin >> pos;
		if (!v[nextSymb]) {
			for (int i = nextSymb; i < 26; i++) {
				if (v[i]) {
					nextSymb = i;
					break;
				}
			}
		}
		answer[--pos] = (char)(nextSymb + 'a');
		v[nextSymb]--;
	}
	for (auto n : answer) cout << n;

	return 0;
}
\end{lstlisting}

\newpage
\textbf{{\large Задача I - Peace of AmericanPie}} \\
\begin{center}
\includegraphics[width=0.9\textwidth]{CT_Crypto/CT_Crypto_I.png}\\ [1cm]
\end{center} 

\textbf{{\large Исходный код}}
\begin{lstlisting}[language=C]
#include <iostream>
using namespace std;
int main() {
	int byte = 0;
	int spow = 256;
	int collected = 0;
	char currentBit;
	while ((currentBit = cin.get()) != '\n') {
		cin.unget();
		while (collected < 8) {
			currentBit = cin.get();
			byte += (currentBit - '0') * spow;
			spow /= 2;
			collected++;
		}
		cout << (char)(byte / 2);
		byte = 0;
		spow = 256;
		collected = 0;
	}
	return 0;
}
\end{lstlisting}

\newpage
\textbf{{\large Задача J - Common}} \\
\begin{center}
\includegraphics[width=0.9\textwidth]{CT_Crypto/CT_Crypto_J.png}\\ [1cm]
\end{center}

\textbf{{\large Исходный код}}
\begin{lstlisting}[language=C]
#include <iostream>
using namespace std;
int main() {
	char t;
	while ((t = cin.get()) != '\n') {
		switch (t) {
			case 'a':
				cout << "n";
				break;
			case 'b':
				cout << "h";
				break;
			case 'c':
				cout << "r";
				break;
			case 'd':
				cout << "x";
				break;
			case 'e':
				cout << "k";
				break;
			case 'f':
				cout << "e";
				break;
			case 'g':
				cout << "y";
				break;
			case 'h':
				cout << "o";
				break;
			case 'i':
				cout << "q"; 
				break;
			case 'j':
				cout << "m";
				break;
			case 'k':
				cout << "j";
				break;
			case 'l':
				cout << "b";
				break;
			case 'm':
				cout << "d";
				break;
			case 'n':
				cout << "u";
				break;
			case 'o':
				cout << "v";
				break;
			case 'p':
				cout << "a";
				break;
			case 'q':
				cout << "p";
				break;
			case 'r':
				cout << "w";
				break;
			case 's':
				cout << "g";
				break;
			case 't':
				cout << "z";
				break;
			case 'u':
				cout << "f";
				break;
			case 'v':
				cout << "i";
				break;
			case 'w':
				cout << "c";
				break;
			case 'x':
				cout << "s";
				break;
			case 'y':
				cout << "t";
				break;
			case 'z':
				cout << "l";
				break;
		}
	}
	return 0;
}
\end{lstlisting}

\newpage
\textbf{{\large Задача M - oPlus}} \\
\begin{center}
\includegraphics[width=0.9\textwidth]{CT_Crypto/CT_Crypto_M.png}\\ [1cm]
\end{center}

\textbf{{\large Исходный код}}
\begin{lstlisting}[language=C]
#include <iostream>
using namespace std;
int main() {
	long n;
	int curr, sum;
	cin >> n;
	for (long i = 0; i < n; i++) {
		cin >> curr;
		if (curr % 2) sum = 400;
		else sum = 398;
		cout << (char)(sum - curr - 112);
	}
	return 0;
}
\end{lstlisting}

\newpage
\textbf{{\large Задача N - tirnaoeumPt}} \\
\begin{center}
\includegraphics[width=0.9\textwidth]{CT_Crypto/CT_Crypto_N.png}\\ [1cm]
\end{center}

\textbf{{\large Исходный код}}
\begin{lstlisting}[language=C]
#include <iostream>
using namespace std;

int main()
{
    int m[] = {0, 1, 16, 17, 8, 9, 24, 25, 2, 3, 18, 19, 10, 11, 22, 23, 4,  5,  20, 21, 12, 13, 22, 23, 6,  7,  22, 23, 14, 15};
    int n, d;
    cin >> n;
    for(int i=0; i<n; ++i)
    {
        cin >> d;
        cout.put(m[d]+'a');
    }
    return 0;
}
\end{lstlisting}

\newpage
\textbf{{\large Задача Q - Peace of bzjd}} \\
\begin{center}
\includegraphics[width=0.9\textwidth]{CT_Crypto/CT_Crypto_Q.png}\\ [1cm]
\end{center}

\textbf{{\large Исходный код}}
\begin{lstlisting}[language=C]
#include <iostream>

using namespace std;

int main()
{
    char temp;
    temp = cin.get();
    while (temp != '\n' && temp != EOF) {
        if (temp == 'z') 
            temp = 'a';
        else temp++;
        cout << temp;
        temp = cin.get();
    }
    return 0;
}
\end{lstlisting}

\newpage
\textbf{{\large Задача R - 6227020800}} \\
\begin{center}
\includegraphics[width=0.9\textwidth]{CT_Crypto/CT_Crypto_R.png}\\ [1cm]
\end{center}

\textbf{{\large Исходный код}}
\begin{lstlisting}[language=C]
#include <iostream>
#include <algorithm>
#include <deque>
#include <cstdio>
using namespace std;
void p(string s) {
	cout << s << endl;
}
int gcd(int a, int b){
    if (b == 0)
        return a;
    return gcd(b, a%b);
}
int main() {
	char t;
	while ((t = cin.get()) != '\n') {
		t -= 13;
		if (t < 'a') {
			cout << (char)('z' - ('a' - t) + 1);
		}
		else {
			cout << t;
		}
	}
	return 0;
}
\end{lstlisting}

\newpage
\textbf{{\large Результаты}} \\
\begin{center}
\includegraphics[width=0.95\textwidth]{CT_Crypto/CT_Crypto_result.png}\\ [1cm]
\end{center}



%----------------------------------------------------------------------------------------
%
%	OpenCup GP of Siberia
%
%----------------------------------------------------------------------------------------
\newpage
\subsection{OpenCup GrandPrix of Siberia}

\textbf{{\large Задача 12 - Construction of Chand Baori}} \\
\begin{center}
\includegraphics[width=0.9\textwidth]{OC_Siberia/OC_Siberia_12.png}\\ [1cm]
\end{center}
\newpage

\textbf{{\large Алгоритм}} \\
{\Huge ???????????????????} \\ 
\\
%\newpage
\textbf{{\large Исходный код}}
\begin{lstlisting}[language=C++]
#include <iostream>
using namespace std;
#define ULL unsigned long long
int main(int argc, const char * argv[]) {
    ULL n, m;
    cin >> n >> m;
    ULL res = 1;
    for(int i=2; i<=n*2; i+=2)
    {
        res *= i;
    }
    if(res<m)
        cout << "Nope";
    else
        cout << "Harshat Mata";
    return 0;
}
\end{lstlisting}

\textbf{{\large Задача 13 - Sum}} \\
\begin{center}
\includegraphics[width=0.9\textwidth]{OC_Siberia/OC_Siberia_13.png}\\ [1cm]
\end{center}
\newpage

\textbf{{\large Алгоритм}} \\
{\Huge ???????????????????} \\ 
\\
%\newpage
\textbf{{\large Исходный код}}
\begin{lstlisting}[language=C++]
#include <iostream>
#include <fstream>
#include <cmath>
#include <vector>
using namespace std;
#define ULL unsigned long long
#define LL long long
int main(int argc, const char * argv[]) {
    ifstream in("input.txt");
    ofstream out("output.txt");
    ULL a, k, p;
    cin >> a >> k >> p;
    ULL sum = 0, prev = 1;
    for(ULL i = 0; i<k; ++i)
    {
        prev = (prev*a)%p;
        sum += prev;
    }
    cout << sum;
    out.close();
    in.close();
    return 0;
}
\end{lstlisting}

\textbf{{\large Задача 14 - Coinquerors}} \\
\begin{center}
\includegraphics[width=0.9\textwidth]{OC_Siberia/OC_Siberia_14_1.png}\\ [1cm]
\includegraphics[width=0.9\textwidth]{OC_Siberia/OC_Siberia_14_2.png}\\ [1cm]

\end{center}
\newpage

\textbf{{\large Алгоритм}} \\
{\Huge ???????????????????} \\ 
\\
%\newpage
\textbf{{\large Исходный код}}
\begin{lstlisting}[language=C++]
#include <iostream>
#include <fstream>
#include <cmath>
#include <vector>
using namespace std;
#define ULL unsigned long long
#define LL long long
#define eps 0.0001
struct player
{
    char name[256];
    LL x, y, r;
};
int main(int argc, const char * argv[]) {
    ifstream in("input.txt");
    ofstream out("output.txt");
    ULL T;
    in >> T;
    double pi81 = M_PI/81, pi2 = M_PI*2;
    for(ULL TT = 0; TT<T; ++TT)
    {
        ULL n;
        in >> n;
        vector<player> pl(n);
        for(int i=0; i<n; ++i)
        {
            in >> pl[i].name >> pl[i].x >> pl[i].y >> pl[i].r;
        }
        ULL maxInd = -1, max = 0, forTie = -1;
        for(int i=0; i<n; ++i)
        {
            ULL count = 0;
            LL rr = pl[i].r*pl[i].r;
            for(int j=0; j<n; ++j)
            {
                for(double pi = 0; pi<=pi2; pi += pi81)
                {
                    double x = (pl[j].r-eps)*cos(pi);
                    double y = (pl[j].r-eps)*sin(pi);
                    if((pl[j].x-pl[i].x+x)*(pl[j].x-pl[i].x+x)+(pl[j].y-pl[i].y+y)*(pl[j].y-pl[i].y+y)<=rr)
                    {
                        ++count;
                        break;
                    }
                }
            }
            if(count > max)
            {
                max = count;
                maxInd = i;
                forTie = -1;
            }
            else if(count == max)
            {
                max = count;
                forTie = maxInd;
                maxInd = i;
            }
        }
        if(maxInd == -1 || forTie != -1)
            out << "TIE" << '\n';
        else
            out << pl[maxInd].name << '\n';
    }
    out.close();
    in.close();
    return 0;
}
\end{lstlisting}

\textbf{{\large Результаты}} \\
\begin{center}
\includegraphics[width=0.95\textwidth]{OC_Siberia/OC_Siberia_result.png}\\ [1cm]
\end{center}




%----------------------------------------------------------------------------------------
%
%	Codeforces Training S02E08
%
%----------------------------------------------------------------------------------------
\newpage
\subsection{Codeforces Training S02E08}

\textbf{{\large Задача G - Growling Gears}} \\
\begin{center}
\includegraphics[width=0.9\textwidth]{CT_S02E08/CT_S02E08_G1.png}\\ [1cm]
\includegraphics[width=0.9\textwidth]{CT_S02E08/CT_S02E08_G2.png}\\ [1cm]
\end{center}
\textbf{{\large Алгоритм}} \\
{\Huge ???????????????????} \\ 
\\
%\newpage
\textbf{{\large Исходный код}}
\begin{lstlisting}[language=C]
#include <iostream>
#include <algorithm>

#include <fstream>

using namespace std;

int main() {
    int k;
    cin >> k;
    int n, a, b, c;
    double max_T = -1000000;
    int max_T_num;
    double temp;

    for(int i = 0; i < k; i++) {
        cin >> n;
        for(int j = 1; j <= n; j++) {
            cin >> a >> b >> c;
            temp = (b * b) / (4 * a) + c;
            if(temp > max_T) {
                max_T = temp;
                max_T_num = j;
            }
        }
        cout << max_T_num << endl;
        max_T = -1000000;
    }
	int q;
	cin >> q;
    return 0;
}
\end{lstlisting}


\newpage
\textbf{{\large Задача J - Jury Jeopardy}} \\
\begin{center}
\includegraphics[width=0.9\textwidth]{CT_S02E08/CT_S02E08_J1.png}\\ [1cm]
\includegraphics[width=0.9\textwidth]{CT_S02E08/CT_S02E08_J2.png}\\ [1cm]
\end{center}
\textbf{{\large Алгоритм}} \\
{\Huge ???????????????????}
\newpage
\textbf{{\large Исходный код}} \\
\begin{lstlisting}[language=C]
#include <iostream>
using namespace std;
int main(int argc, const char * argv[]) {
    long T;
    cin >> T;
    cout << T << '\n';
    char map[201][201];
    cin.get();
    while(T--)
    {
        for(long i=0; i<201; ++i)
            for(long j=0; j<201; ++j)
                map[i][j] = '#';
        char c;
        long x, y, minX, minY, maxX, maxY;
        maxX = maxY = minX = minY = x = y = 100;
        int dir = 0;
        while((c=cin.get())!='\n')
        {
            switch(dir)
            {
                case 0:
                    if(c=='R')
                        dir = 3;
                    else if(c=='L')
                        dir = 1;
                    else if(c=='B')
                        dir = 2;
                    break;
                case 1:
                    if(c=='R')
                        dir = 0;
                    else if(c=='L')
                        dir = 2;
                    else if(c=='B')
                        dir = 3;
                    break;
                case 2:
                    if(c=='R')
                        dir = 1;
                    else if(c=='L')
                        dir = 3;
                    else if(c=='B')
                        dir = 0;
                    break;
                case 3:
                    if(c=='R')
                        dir = 2;
                    else if(c=='L')
                        dir = 0;
                    else if(c=='B')
                        dir = 1;
                    break;
            }
            switch(dir)
            {
                case 0:
                    ++x;
                    if(x>maxX)
                        maxX = x;
                    break;
                case 1:
                    --y;
                    if(y<minY)
                        minY = y;
                    break;
                case 2:
                    --x;
                    if(x<minX)
                        minX = x;
                    break;
                case 3:
                    ++y;
                    if(y>maxY)
                        maxY = y;
                    break;
            }
            map[x][y] = '.';
        }
        cout << maxY-minY+3 << ' ' << maxX-minX+2 << '\n';
        for(long i = minY-1; i<=maxY+1; ++i)
        {
            for(long j = minX; j<=maxX+1; ++j)
                cout.put(map[j][i]);
            cout.put('\n');
        }
    }
    return 0;
}
\end{lstlisting}

\textbf{{\large Результаты}} \\
\begin{center}
\includegraphics[width=0.95\textwidth]{CT_S02E05/CT_S02E05_result.png}\\ [1cm]
\end{center}



%----------------------------------------------------------------------------------------
%
%	Codeforces Training S02E09
%
%----------------------------------------------------------------------------------------
\newpage
\subsection{Codeforces Training S02E09}

\textbf{{\large Задача G - Graveyard}} \\
\begin{center}
\includegraphics[width=0.9\textwidth]{CT_S02E09/CT_S02E09_G.png}\\ [1cm]
\end{center}
\textbf{{\large Алгоритм}} \\
{\Huge ???????????????????} \\ 
\\
%\newpage
\textbf{{\large Исходный код}}
\begin{lstlisting}[language=C]
#include <iostream>
#include <vector>
#include <string>
#include <algorithm>
#include <queue>
#include <climits>
#include <cctype>
#include <cmath>
#include <fstream>
#include <iomanip>
#define ll long long
#define ull unsigned long long
using namespace std;
int main()
{
    ifstream in("graveyard.in");
    ofstream out("graveyard.out");
    ll n, m;
    in >> n >> m;
    ll nm = n+m;
    vector<double> a(n);
    vector<double> b(nm);
    double s = 0;
    for(ll i=0; i<n; ++i)
    {
        a[i] = s;
        s += 10000.0/n;
    }
    s = 0;
    for(ll i=0; i<nm; ++i)
    {
        b[i] = s;
        s += 10000.0/nm;
    }
    vector<bool> u(nm, 0);
    vector<double> r(n);
    double res = 0;
    for(ll i=0; i<n; ++i)
    {
        double min = 200000.0;
        for(ll j=0; j<nm; ++j)
        {
            
            if(!u[j] && fabs(a[i]-b[j])<min)
            {
                u[j] = 1;
                min = fabs(a[i]-b[j]);
            }
        }
        res += min;
    }
    out << fixed << setprecision(4) << res;
    in.close();
    out.close();
    return 0;
}
\end{lstlisting}


\newpage
\textbf{{\large Задача J - Java vs C++}} \\
\begin{center}
\includegraphics[width=0.9\textwidth]{CT_S02E09/CT_S02E09_J.png}\\ [1cm]
\end{center}
\textbf{{\large Алгоритм}} \\
{\Huge ???????????????????}
\newpage
\textbf{{\large Исходный код}} \\
\begin{lstlisting}[language=C]
#include <iostream>
#define ll long long
#define ull unsigned long long
#define ERROR {out << "Error!"; return 0;}
using namespace std;
int main()
{
    ifstream in("java_c.in");
    ofstream out("java_c.out");
    char c;
    bool java = 0, cpp = 0, us = 0;
    char str[1000];
    ll n=0;
    while(!in.eof() && (c=in.get())!='\n')
    {
        if(in.eof())
            break;
        if(c=='_')
        {
            if(us || !n)
                ERROR
            cpp = 1;
            us = 1;
        }
        else if(isupper(c))
        {
            if(!n)
                ERROR
            java = 1;
            str[n++] = '_';
            str[n++] = tolower(c);
        }
        else if(us)
            str[n++] = toupper(c);
        else
            str[n++] = c;
        if(c!='_')
            us = 0;
        if(java && cpp)
            ERROR
    }
    str[n] = '\0';
    if(n&&!us)
        out << str;
    else
        out << "Error!";
    in.close();
    out.close();
    return 0;
}
\end{lstlisting}

\newpage
\textbf{{\large Задача K - Kickdown}} \\
\begin{center}
\includegraphics[width=0.9\textwidth]{CT_S02E09/CT_S02E09_K.png}\\ [1cm]
\end{center}
\textbf{{\large Алгоритм}} \\
Для решения этой задачи нужно подвигать шестеренки влево и вправо, проверить совпадение и выбрать ответ с наибольшим совпадением.
\newpage
\textbf{{\large Исходный код}} \\
\begin{lstlisting}[language=C]
#include <iostream>
#include <fstream>

bool checkGears1(string master, string driven, int pos) {
    for (int i = 0; pos < master.length() && i < driven.length(); i++) {
        if (master[pos] == driven[i])
            if (master[pos] == '2')
                return false;
        pos++;
    }
    return true;
}

bool checkGears2(string master, string driven, int pos) {
    for (int i = 0; i < master.length() && pos < driven.length(); i++) {
        if (master[i] == driven[pos])
            if (master[i] == '2')
                return false;
        pos++;
    }
    return true;
}

int main() {
	ifstream in("kickdown.in");
    ofstream out("kickdown.out");
    string master, driven;
    in >> master >> driven;
    if (master.length() < driven.length()) swap(master, driven);
    
    int pos1 = 0;
    while (pos1 < master.length() && !checkGears1(master, driven, pos1)) pos1++;
    
    int pos2 = 0;
    while (pos2 < master.length() && !checkGears2(master, driven, pos2)) pos2++;
    
    int d1 = (int)(driven.length() + pos1 - master.length());
    int diff1 = (d1 >= 0) ? d1 : 0;
    int diff2 = pos2;
    
    if (diff1 < diff2) {
        out << max(master.length(), driven.length() + pos1) << endl;
    }
    else {
        out << master.length() + pos2 << endl;
    }
    
    in.close();
    out.close();
    
    return 0;
}
\end{lstlisting}

\textbf{{\large Результаты}} \\
\begin{center}
\includegraphics[width=0.95\textwidth]{CT_S02E09/CT_S02E09_result.png}\\ [1cm]
\end{center}



%----------------------------------------------------------------------------------------
%
%	Codeforces Олимпиада школьников НН
%
%----------------------------------------------------------------------------------------
\newpage
\subsection{Codeforces Олимпиада школьников Нижегородской обл.}

\textbf{{\large Задача A - Выравнивание вещественных чисел}} \\
\begin{center}
\includegraphics[width=0.9\textwidth]{CT_school_nn/CT_school_nn_A.png}\\ [1cm]
\end{center}
\textbf{{\large Алгоритм}} \\
{\Huge ???????????????????} \\ 
\\
%\newpage
\textbf{{\large Исходный код}}
\begin{lstlisting}[language=C]
#include <iostream>
#include <cmath>
#define ll long long
#define ull unsigned long long
using namespace std;
int main()
{
    ll n;
    cin >> n;
    char num[1000][2010];
    int logs[1000];
    ll maxLog = 1;
    cin.get();
    for(ll i=0; i<n; ++i)
    {
        ll j=0;
        logs[i] = 0;
        bool pf = 0;
        while((num[i][j]=cin.get())!='\n')
        {
            if(num[i][j] == '.')
                pf = 1;
            if(!pf)
                ++logs[i];
            ++j;
        }
        num[i][j++] = '\0';
        if(logs[i]>maxLog)
            maxLog = logs[i];
    }
    for(ll i=0; i<n; ++i)
    {
        for(ll j=0; j<maxLog-logs[i]; ++j)
            cout.put('#');
        cout << num[i] << '\n';
    }
    return 0;
}
\end{lstlisting}


\newpage
\textbf{{\large Задача F - Фоторамка}} \\
\begin{center}
\includegraphics[width=0.9\textwidth]{CT_school_nn/CT_school_nn_F.png}\\ [1cm]
\end{center}
\textbf{{\large Алгоритм}} \\
В этой задаче можно заметить закономерность и предпосчитать ответ, так как всего может быть 20 различных входных данных. Таким образом, сложность составляет $O(1)$.
\newpage
\textbf{{\large Исходный код}} \\
\begin{lstlisting}[language=C]
#include <iostream>
using namespace std;
int main() {
    long long m[]={0, 0, 0, 24, 120, 360, 840, 1680, 3024, 5040, 7920, 11880, 17160, 24024, 32760, 43680, 57120, 73440, 93024, 116280};
    int n;
    cin >> n;
    cout << m[n-1];
    return 0;
}
\end{lstlisting}

\newpage
\textbf{{\large Задача I - Изи}} \\
\begin{center}
\includegraphics[width=0.9\textwidth]{CT_school_nn/CT_school_nn_I.png}\\ [1cm]
\end{center}
\textbf{{\large Алгоритм}} \\
В задаче просто нужно вывести $n - 1$.
\newpage
\textbf{{\large Исходный код}} \\
\begin{lstlisting}[language=C]
#include <iostream>
#include <vector>
#include <string>
#include <algorithm>
#include <queue>
#include <climits>
#include <cctype>
#include <fstream>
#define ll long long
#define ull unsigned long long
using namespace std;
int main()
{
    ll n;
    cin >> n;
    cout << n-1;
    return 0;
}
\end{lstlisting}

\textbf{{\large Результаты}} \\
\begin{center}
\includegraphics[width=0.95\textwidth]{CT_school_nn/CT_school_nn_result.png}\\ [1cm]
\end{center}



%----------------------------------------------------------------------------------------
%
%	OpenCup GP of Central Europe
%
%----------------------------------------------------------------------------------------
\newpage
\subsection{OpenCup GrandPrix of Central Europe}

\textbf{{\large Задача A - Адокат}} \\
\begin{center}
\includegraphics[width=0.9\textwidth]{OC_Central_Europe/OC_Central_Europe_A.png}\\ [1cm]
\end{center}
\newpage

\textbf{{\large Алгоритм}} \\
{\Huge ???????????????????} \\ 
\\
%\newpage
\textbf{{\large Исходный код}}
\begin{lstlisting}[language=C++]
#include <iostream>
#include <fstream>
#include <vector>

using namespace std;

int main()
{
    ll n, m;
    ll a, b, d;
    cin >> n >> m;
    vector< pair<ll, ll> > maxA(m, make_pair(-1, -1)), minB(m, make_pair(-1, -1));
    for (ll i=0; i<n; ++i) {
        cin >> a >> b >> d;
        --d;
        if (maxA[d].second == -1 || maxA[d].second < a) {
            maxA[d].second = a;
            maxA[d].first = i;
        }
        if (minB[d].second == -1 || minB[d].second > b) {
            minB[d].second = b;
            minB[d].first = i;
        }
    }
    for (ll i=0; i<m; ++i) {
        if (minB[i].second < maxA[i].second) {
            cout << "TAK " << minB[i].first+1 << ' ' << maxA[i].first+1 << '\n';
        }
        else {
            cout << "NIE\n";
        }
    }
    return 0;
}

\end{lstlisting}

\textbf{{\large Результаты}} \\
\begin{center}
\includegraphics[width=0.95\textwidth]{OC_Central_Europe/OC_Central_Europe_result.png}\\ [1cm]
\end{center}



%----------------------------------------------------------------------------------------
%
%	Codeforces Training S02E10
%
%----------------------------------------------------------------------------------------
\newpage
\subsection{Codeforces Training S02E10}

\textbf{{\large Задача A - Abnormal Coins}} \\
\begin{center}
\includegraphics[width=0.9\textwidth]{CT_S02E10/CT_S02E10_A.png}\\ [1cm]
\end{center}
\textbf{{\large Алгоритм}} \\
{\Huge ???????????????????} \\ 
\\
%\newpage
\textbf{{\large Исходный код}}
\begin{lstlisting}[language=C]
#include <iostream>
#include <vector>
#include <algorithm
#include <iomanip>
#define ll long long
#define ull unsigned long long
using namespace std;
int main() {
    LL n;
    cin >> n;
    LL count = 0;
    LL sum = 0;
    
    for (LL i = 3; ; i++) {
        sum += i;
        if (sum > n) break;
        count++;
    }
    cout << count << endl;
    return 0;
}
\end{lstlisting}


\newpage
\textbf{{\large Задача B - Fake Coins}} \\
\begin{center}
\includegraphics[width=0.9\textwidth]{CT_S02E10/CT_S02E10_B.png}\\ [1cm]
\end{center}
\textbf{{\large Алгоритм}} \\
{\Huge ???????????????????}
\newpage
\textbf{{\large Исходный код}} \\
\begin{lstlisting}[language=C]
#include <iostream>
#include <map>
#include <vector>
#define ll long long
#define ull unsigned long long
#define ERROR {out << "Error!"; return 0;}
using namespace std;
int main() {
    map<string, int> strings;
    string base, current = "";
    vector<int> spos;
    cin >> base;
    int baseSize = (int)base.size();
    for (int i = 1; i < baseSize; i++) {
        for (int j = i + 1; j <= baseSize; j++) {
            current += base[i - 1];
            current += base[j - 1];
            int cPos = i + j;
            int pPos = j, temp = 0;
            while (cPos <= baseSize) {
                current += base[cPos - 1];
                temp = pPos;
                pPos = cPos;
                cPos += temp;
            }
            strings[current]++;
            current.clear();
        }
    }
    
    cout << strings.size() << endl;
    
    return 0;
}
\end{lstlisting}

\newpage
\textbf{{\large Задача G - Coin Game}} \\
\begin{center}
\includegraphics[width=0.9\textwidth]{CT_S02E10/CT_S02E10_G.png}\\ [1cm]
\end{center}
\textbf{{\large Алгоритм}} \\
{\Huge ???????????????????}
\newpage
\textbf{{\large Исходный код}} \\
\begin{lstlisting}[language=C]
#include <iostream>
#include <vector>
#include <map>
#include <cmath>
#include <algorithm>
#include <iomanip>
#include <list>
#include <iterator>
using namespace std;
#define ll long long
#define ull unsigned long long

int main()
{
    char a[25001], b[25001], c;
    int n = 0, first1 = -1, last1 = -1;
    while((c=cin.get())!='\n' && !cin.eof())
    {
        if(first1 == -1 && c == '1')
            first1 = n;
        if(c == '1')
            last1 = n;
        if(cin.eof())
            break;
        a[n] = b[n] = c;
        ++n;
    }
    a[n] = b[n] = '\0';
    ll aCount = 0, bCount = 0;
    for(ll i=0; i<n; ++i)
    {
        if(a[i] == '1')
            continue;
        for(ll j=i+1; j<n; ++j)
        {
            if(a[j] == '1')
            {
                swap(a[i], a[j]);
                aCount += j-i;
                break;
            }
        }
    }
    for(ll i=n-1; i>=0; --i)
    {
        if(b[i] == '1')
            continue;
        for(ll j=i-1; j>=0; --j)
        {
            if(b[j] == '1')
            {
                swap(b[i], b[j]);
                bCount += i-j;
                break;
            }
        }
    }
    if(aCount < bCount)
        cout << aCount;
    else
        cout << bCount;
    return 0;
}
\end{lstlisting}

\textbf{{\large Результаты}} \\
\begin{center}
\includegraphics[width=0.95\textwidth]{CT_S02E10/CT_S02E10_result.png}\\ [1cm]
\end{center}



%----------------------------------------------------------------------------------------
%
%	OpenCup GP of Europe
%
%----------------------------------------------------------------------------------------
\newpage
\subsection{OpenCup GrandPrix of Europe}

\textbf{{\large Задача E - Express As The Sum}} \\
\begin{center}
\includegraphics[width=0.9\textwidth]{OC_Europe/OC_Europe_E.png}\\ [1cm]
\end{center}
\newpage

\textbf{{\large Алгоритм}} \\
{\Huge ???????????????????} \\ 
\\
%\newpage
\textbf{{\large Исходный код}}
\begin{lstlisting}[language=C++]
#include <iostream>
using namespace std;
#define ll long long
#define ull unsigned long long
void answer(ll n, ll l, ll r) {
    cout << n << " = " << l;
    for(ll i=l+1; i<=r; ++i) {
        cout << " + " << i;
    }
    cout.put('\n');
}
int main()
{
    ll T;
    cin >> T;
    while (T--) {
        ll n;
        cin >> n;
        ll t = n;
        bool f = 0;
        while(t!=0) {
            if (t>1 && t&1) {
                f = 1;
                break;
            }
            t >>= 1;
        }
        if(!f) {
            cout << "IMPOSSIBLE\n";
            continue;
        }
        f = 0;
        for (ll i=2; !f && i<100500; ++i) {
            ll sum = 0;
            ll r = n/i;
            ll l = r-i+1;
            if(l<1) {
                l = 1;
                r = l+i-1;
            }
            for(ll j=l; j<=r; ++j) {
                sum += j;
            }
            if(sum == n) {
                answer(n, l, r);
                break;
            }
            for(ll j=r+1; j<r+i; ++j) {
                sum += j-l;
                if(sum == n) {
                    answer(n, l+1, j);
                    f = 1;
                    break;
                }
                ++l;
            }
        }
    }
    return 0;
}
\end{lstlisting}

\textbf{{\large Задача F - Factory}} \\
\begin{center}
\includegraphics[width=0.9\textwidth]{OC_Europe/OC_Europe_F.png}\\ [1cm]
\end{center}
\newpage

\textbf{{\large Алгоритм}} \\
{\Huge ???????????????????} \\ 
\\
%\newpage
\textbf{{\large Исходный код}}
\begin{lstlisting}[language=C++]
#include <iostream>
#include <vector>
using namespace std;
#define ll long long
#define ull unsigned long long
#define eps 0.00001
ll min(ll a, ll b){return (a<b?a:b);}
ll max(ll a, ll b){return (a>b?a:b);}

struct Gear {
    int x, y, r;
};

struct GearSpeed {
    ll n, d;
    bool dir;
};
int main()
{
    ll T;
    cin >> T;
    while (T--) {
        ll n;
        cin >> n;

        vector< list<int> > m(n);
        vector<bool> u(n, 0);
        vector<Gear> gears(n);
        for (ll i=0; i<n; ++i) {
            cin >> gears[i].x >> gears[i].y >> gears[i].r;
        }
        for (int i=0; i<n; ++i) {
            for (int j=i+1; j<n; ++j) {
                double x = gears[i].x-gears[j].x;
                double y = gears[i].y-gears[j].y;
                if(sqrt(x*x+y*y)<=gears[i].r+gears[j].r) {
                    m[i].insert(m[i].begin(), j);
                    m[j].insert(m[j].begin(), i);
                }
            }
        }
        vector<GearSpeed> speed(n);
        speed[0].n = 1;
        speed[0].d = 1;
        speed[0].dir = 0;
        queue<int> q;
        q.push(0);
        u[0] = 1;
        while (!q.empty()) {
            int a = q.front();
            q.pop();
            list<int>::iterator it = m[a].begin();
            for (; it!=m[a].end(); ++it) {
                int b = *it;
                if(u[b])
                    continue;
                u[b] = 1;
                speed[b].n = speed[a].n * gears[a].r;
                speed[b].d = speed[a].d * gears[b].r;
                speed[b].dir = !speed[a].dir;
                ll u = speed[b].n;
                ll v = speed[b].d;
                ll temp;
                while (v != 0) {
                    temp = u % v;
                    u = v;
                    v = temp;
                }
                speed[b].n /= u;
                speed[b].d /= u;
                q.push(b);
            }
        }
        for (ll i=0; i<n; ++i) {
            if(!u[i])
                cout << "not moving\n";
            else {
                cout << speed[i].n;
                if(speed[i].d != 1) {
                    cout << '/' << speed[i].d;
                }
                cout.put(' ');
                if(speed[i].dir) {
                    cout << "counterclockwise\n";
                }
                else {
                    cout << "clockwise\n";
                }
            }
        }
    }
    return 0;
}
\end{lstlisting}

\textbf{{\large Задача K - Keyboard Troubles}} \\
\begin{center}
\includegraphics[width=0.9\textwidth]{OC_Europe/OC_Europe_K.png}\\ [1cm]
\end{center}
\newpage

\textbf{{\large Алгоритм}} \\
{\Huge ???????????????????} \\ 
\\
%\newpage
\textbf{{\large Исходный код}}
\begin{lstlisting}[language=C++]
#include <iomanip>
#include <limits>
#include <iostream>
#include <algorithm>

#define LL  long long
#define ULL unsigned long long
#define EPS 1e-11

#define more_speed ios_base::sync_with_stdio(false);

using namespace std;

bool g(int x, int y) {

    if (x == y) return true;

    if (x == 1) {
        return true;
    }
    else if (x == 2) {
        if (y == 3 || y == 5 || y == 6 || y == 8 || y == 9 || y == 0) return true;
    }
    else if (x == 3) {
        if (y == 6 || y == 9) return true;
    }
    else if (x == 4) {
        if (y >= 4 || y == 0) return true;
    }
    else if (x == 5) {
        if (y == 6 || y == 8 || y == 9 || y == 0) return true;
    }
    else if (x == 6) {
        if (y == 9) return true;
    }
    else if (x == 7) {
        if (y >= 7 || y == 0) return true;
    }
    else if (x == 8) {
        if (y == 9 || y == 0) return true;
    }
    return false;
}

int main() {

    more_speed

    int z[] = {0,1,1,1,1,1,1,1,1,1,1,1,1,1,1,1,1,1,1,1,1,0,1,1,0,1,1,0,1,1,0,0,0,1,0,0,1,0,0,1,1,0,0,0,1,1,1,1,1,1,1,0,0,0,0,1,1,0,1,1,0,0,0,0,0,0,1,0,0,1,1,0,0,0,0,0,0,1,1,1,1,0,0,0,0,0,0,0,1,1,0,0,0,0,0,0,0,0,0,1,1,0,0,0,0,0,0,0,0,0,1,1,1,1,1,1,1,1,1,1,1,0,1,1,0,1,1,0,1,1,0,0,0,1,0,0,1,0,0,1,1,0,0,0,1,1,1,1,1,1,1,0,0,0,0,1,1,0,1,1,0,0,0,0,0,0,1,0,0,1,1,0,0,0,0,0,0,1,1,1,1,0,0,0,0,0,0,0,1,1,0,0,0,0,0,0,0,0,0,1,1,0,0,0,0,0,0,0,0,0,0,0,0,0,0,0,0,0,0,0,1,0,1,1,0,1,1,0,1,1,0,0,0,1,0,0,1,0,0,1,0,0,0,0,0,0,0,0,0,0,1,0,0,0,0,1};

    int t;
    cin >> t;

    for (int i = 0; i < t; i++) {
        int n;
        cin >> n;
        if (z[n] == 1) {
            cout << n << endl;
        }
        else {
            int y = n;
            int x = n;
            while (z[x] == 0) {
                x++;
            }
            while (z[y] == 0) {
                y--;
            }
            if (n - y > x - n) {
                cout << x << endl;
            }
            else {
                cout << y << endl;
            }
        }
    }

    return 0;
}
\end{lstlisting}

\textbf{{\large Задача K - Allo}} \\
\begin{center}
\includegraphics[width=0.9\textwidth]{OC_Europe/OC_Europe_N.png}\\ [1cm]
\end{center}
\newpage

\textbf{{\large Алгоритм}} \\
{\Huge ???????????????????} \\ 
\\
%\newpage
\textbf{{\large Исходный код}}
\begin{lstlisting}[language=C++]
#include <iostream>
#include <vector>
using namespace std;
#define ll long long
#define ull unsigned long long
ll min(ll a, ll b){return (a<b?a:b);}
ll max(ll a, ll b){return (a>b?a:b);}
int main()
{
    ll n;
    vector < stack< pair<ll, int> > > vars(26);
    int scope = 0;
    cin >> n;
    char str[20];
    for (int i=0; i<n; ++i) {
        cin >> str;
        if(str[0] == '{') {
            ++scope;
        }
        else if(str[0] == '}') {
            for (int i=0; i<26; ++i) {
                if(!vars[i].empty() && vars[i].top().second == scope) {
                    vars[i].pop();
                }
            }
            --scope;
        }
        else if(str[1] == '=') {
            int c1 = str[0]-'a';
            ll a;
            if(isalpha(str[2])) {
                int c2 = str[2]-'a';
                a = vars[c2].top().first;
                int c1Scope = vars[c1].top().second;
                vars[c1].pop();
                vars[c1].push(make_pair(a, c1Scope));
            }
            else {
                a = 0;
                for (int i=2; str[i]!='\0'; ++i)
                    a = a*10+str[i]-'0';
            }
            int c1Scope = vars[c1].top().second;
            vars[c1].pop();
            vars[c1].push(make_pair(a, c1Scope));
        }
        else if(!strcmp(str, "int")) {
            cin >> str;
            int c;
            c = str[0]-'a';
            vars[c].push(make_pair(-1, scope));
        }
        else if(!strcmp(str, "print")) {
            cin >> str;
            int c = str[0]-'a';
            cout << vars[c].top().first << '\n';
        }
    }
    return 0;
}
\end{lstlisting}

\textbf{{\large Задача O - Game}} \\
\begin{center}
\includegraphics[width=0.9\textwidth]{OC_Europe/OC_Europe_O.png}\\ [1cm]
\end{center}
\newpage

\textbf{{\large Алгоритм}} \\
{\Huge ???????????????????} \\ 
\\
%\newpage
\textbf{{\large Исходный код}}
\begin{lstlisting}[language=C++]
#include <iostream>
#include <vector>
using namespace std;
#define ll long long
#define ull unsigned long long
ll min(ll a, ll b){return (a<b?a:b);}
ll max(ll a, ll b){return (a>b?a:b);}
int main()
{
    ll T;
    cin >> T;
    cin.get();
    while (T--) {
        vector<bool> nums(10, 0);
        char c;
        while((c=cin.get())!='\n' && !cin.eof())
        {
            if(cin.eof())
                break;
            nums[c-'0'] = 1;
        }
        bool f = 0;
        for(ll i=0; i<10; ++i)
        {
            if(!nums[i])
            {
                f = 1;
                cout << i;
            }
        }
        if(!f)
            cout << "allo";
        cout << '\n';
            
    }
    return 0;
}
\end{lstlisting}

\textbf{{\large Результаты}} \\
\begin{center}
\includegraphics[width=0.95\textwidth]{OC_Europe/OC_Europe_result.png}\\ [1cm]
\end{center}



%----------------------------------------------------------------------------------------
%
%	OpenCup GP of Peterhof
%
%----------------------------------------------------------------------------------------
\newpage
\subsection{OpenCup GrandPrix of Peterhof}

\textbf{{\large Задача H - Некратчайший путь}} \\
\begin{center}
\includegraphics[width=0.9\textwidth]{OC_Peterhof/OC_Peterhof_H1.png}\\ [1cm]
\includegraphics[width=0.9\textwidth]{OC_Peterhof/OC_Peterhof_H2.png}\\ [1cm]
\end{center}
\newpage

\textbf{{\large Алгоритм}} \\
{\Huge ???????????????????} \\ 
\\
%\newpage
\textbf{{\large Исходный код}}
\begin{lstlisting}[language=C++]
#include <iostream>
#include <vector>
#include <map>
using namespace std;
#define ll long long
#define ull unsigned long long
#define eps 0.00001
ll min(ll a, ll b){return (a<b?a:b);}
ll max(ll a, ll b){return (a>b?a:b);}

vector < vector<char> > paths;
vector<char> path;
size_t __min;

void foo(char map[4][4], int x, int y) {
    if (x == 3 && y == 3) {
        if (path.size() < __min) {
            __min = path.size();
        }
        paths.push_back(path);
    }
    if (x<3 && map[y][x+1] == '.') {
        path.push_back('R');
        map[y][x] = 'X';
        foo(map, x+1, y);
        map[y][x] = '.';
        path.pop_back();
    }
    if (x>0 && map[y][x-1] == '.') {
        path.push_back('L');
        map[y][x] = 'X';
        foo(map, x-1, y);
        map[y][x] = '.';
        path.pop_back();
    }
    map[y][x] = '.';
    if (y<3 && map[y+1][x] == '.') {
        path.push_back('D');
        map[y][x] = 'X';
        foo(map, x, y+1);
        map[y][x] = '.';
        path.pop_back();
    }
    map[y][x] = '.';
    if (y>0 && map[y-1][x] == '.') {
        path.push_back('U');
        map[y][x] = 'X';
        foo(map, x, y-1);
        map[y][x] = '.';
        path.pop_back();
    }
}

int main()
{
    char map[4][4];
    bool e = 1;
    while (e) {
        for (int i=0; i<4; ++i) {
            for (int j=0; j<4; ++j) {
                map[i][j] = cin.get();
            }
            cin.get();
        }
        __min = 100500;
        foo(map, 0, 0);
        bool f = 0;
        for (int i=0; i<paths.size(); ++i) {
            if (paths[i].size() != __min) {
                for (int j=0; j<paths[i].size(); ++j) {
                    cout.put(paths[i][j]);
                }
                f = 1;
                break;
            }
        }
        if (!f) {
            cout << -1;
        }
        cout.put('\n');
        for (int i=0; i<5; ++i) {
            cin.get();
            if (cin.eof()) {
                e = 0;
                break;
            }
        }
        paths.clear();
        path.clear();
    }
    return 0;
}
\end{lstlisting}

\textbf{{\large Результаты}} \\
\begin{center}
\includegraphics[width=0.95\textwidth]{OC_Peterhof/OC_Peterhof_result.png}\\ [1cm]
\end{center}



%----------------------------------------------------------------------------------------
%
%	OpenCup GP of Japan
%
%----------------------------------------------------------------------------------------
\newpage
\subsection{OpenCup GrandPrix of Japan}

\textbf{{\large Задача K - Beads}} \\
\begin{center}
\includegraphics[width=0.9\textwidth]{OC_Japan/OC_Japan_K.png}\\ [1cm]
\end{center}
\newpage

\textbf{{\large Алгоритм}} \\
{\Huge ???????????????????} \\ 
\\
%\newpage
\textbf{{\large Исходный код}}
\begin{lstlisting}[language=C++]
#include <iostream>

using namespace std;

long min_cyclic_shift (string s) {
    s += s;
    long n = (long) s.length();
    long i = 0, ans = 0;
    while (i < n/2) {
        ans = i;
        long j = i + 1, k = i;
        while (j < n && s[k] <= s[j]) {
            if (s[k] < s[j])
                k = i;
            else
                ++k;
            ++j;
        }
        while (i <= k)  i += j - k;
    }
    return ans;
}

int main(int argc, const char * argv[]) {
    long n;
    string s;
    cin >> n >> s;
    long answer = min_cyclic_shift(s);
    cout << answer + 1 << endl;
    return 0;
}
\end{lstlisting}

\textbf{{\large Задача L - The Maximum Sum}} \\
\begin{center}
\includegraphics[width=0.9\textwidth]{OC_Japan/OC_Japan_L.png}\\ [1cm]
\end{center}
\newpage

\textbf{{\large Алгоритм}} \\
{\Huge ???????????????????} \\ 
\\
%\newpage
\textbf{{\large Исходный код}}
\begin{lstlisting}[language=C++]
#include <iomanip>
#include <limits>
#include <iostream>
#include <algorithm>

int main() {
    int n, M;
	cin >> n >> M;

	vector <int> num;
	int temp;
	for (int i = 0; i < n; i++) {
		cin >> temp;
		num.push_back(temp);
	}

	int max_sum = 0;
	for(int i = 0; i < n; i++)
		for(int j = i + 1; j < n; j++) {
			temp = num[i] + num[j];
			if(temp <= M && temp > max_sum)
				max_sum = temp;
		}

	cout << max_sum << endl;
    cin >> n;
    return 0;
}
\end{lstlisting}

\textbf{{\large Задача L - The Maximum Sum}} \\
\begin{center}
\includegraphics[width=0.9\textwidth]{OC_Japan/OC_Japan_L.png}\\ [1cm]
\end{center}
\newpage

\textbf{{\large Алгоритм}} \\
{\Huge ???????????????????} \\ 
\\
%\newpage
\textbf{{\large Исходный код}}
\begin{lstlisting}[language=C++]
#include <iomanip>
#include <limits>
#include <iostream>
#include <algorithm>

int main() {
    int n, M;
	cin >> n >> M;

	vector <int> num;
	int temp;
	for (int i = 0; i < n; i++) {
		cin >> temp;
		num.push_back(temp);
	}

	int max_sum = 0;
	for(int i = 0; i < n; i++)
		for(int j = i + 1; j < n; j++) {
			temp = num[i] + num[j];
			if(temp <= M && temp > max_sum)
				max_sum = temp;
		}

	cout << max_sum << endl;
    cin >> n;
    return 0;
}
\end{lstlisting}

\textbf{{\large Задача M - Spellcheck}} \\
\begin{center}
\includegraphics[width=0.9\textwidth]{OC_Japan/OC_Japan_M.png}\\ [1cm]
\end{center}
\newpage

\textbf{{\large Алгоритм}} \\
{\Huge ???????????????????} \\ 
\\
%\newpage
\textbf{{\large Исходный код}}
\begin{lstlisting}[language=C++]
#include <iostream>
#include <vector>

using namespace std;

int main()
{
    long T;
    cin >> T;
    char aStr[100], bStr[100];
    cin.get();
    while (T--) {
        char c;
        long a = 0, aN = 0, bN = 0, res = 0;
        while (c = cin.get()) {
            if (c == ' ' || c == '\n') {
                if (a&1) {
                    if (!strcmp("u", bStr) || !strcmp("ur", bStr) || strstr(bStr, "lol") != NULL) {
                        ++res;
                    }
                    else if((!strcmp("would", aStr) || !strcmp("should", aStr)) && !strcmp("of", bStr)) {
                        ++res;
                    }
                    aN = 0;
                }
                else {
                    if (!strcmp("u", aStr) || !strcmp("ur", aStr) || strstr(aStr, "lol") != NULL) {
                        ++res;
                    }
                    else if(a && (!strcmp("would", bStr) || !strcmp("should", bStr)) && !strcmp("of", aStr)) {
                        ++res;
                    }
                    bN = 0;
                }
                if (c == '\n') {
                    break;
                }
                ++a;
                continue;
            }
            if (a&1) {
                bStr[bN++] = c;
                bStr[bN] = '\0';
            }
            else {
                aStr[aN++] = c;
                aStr[aN] = '\0';
            }
        }
        cout << res << '\n';
    }
    return 0;
}
\end{lstlisting}

\textbf{{\large Задача N - Bluetooth}} \\
\begin{center}
\includegraphics[width=0.9\textwidth]{OC_Japan/OC_Japan_N.png}\\ [1cm]
\end{center}
\newpage

\textbf{{\large Алгоритм}} \\
{\Huge ???????????????????} \\ 
\\
%\newpage
\textbf{{\large Исходный код}}
\begin{lstlisting}[language=C++]
#include <iomanip>
#include <limits>
#include <iostream>
#include <algorithm>


double dist(const pair<int, int> &a, const pair<int, int> &b) {
    double x, y;
    x = (a.first - b.first);
    y = (a.second - b.second);
    x *= x;
    y *= y;
    return sqrt(x + y);
}

int main() {

    more_speed
    int n, d;
    cin >> n >> d;
    vector<pair<int, int> > points(n);
    for (int i = 0; i < n; i++) {
        cin >> points[i].first >> points[i].second;
    }

    vector<vector<int> > g(n, vector<int>());

    for (int i = 0; i < n; i++) {
        for (int j = 0; j < n; j++) {
            if (i != j) {
                double dis = dist(points[i], points[j]);
                if (dis <= d) {
                    g[i].push_back(j);
                }
            }
        }
    }

    int s = 0;

    queue<int> q;
    q.push(s);
    vector<bool> used (n);
    used[s] = true;
    while (!q.empty()) {
        int v = q.front();
        q.pop();
        for (size_t i = 0; i < g[v].size(); ++i) {
            int to = g[v][i];
            if (!used[to]) {
                if (to == n - 1) {
                    cout << "y" << endl;
                    return 0;
                }
                used[to] = true;
                q.push (to);
            }
        }
    }

    cout << "n" << endl;
    return 0;
}
\end{lstlisting}

\textbf{{\large Результаты}} \\
\begin{center}
\includegraphics[width=0.95\textwidth]{OC_Japan/OC_Japan_result.png}\\ [1cm]
\end{center}



%----------------------------------------------------------------------------------------
%
%	OpenCup GP Northern
%
%----------------------------------------------------------------------------------------
\newpage
\subsection{OpenCup Northern GrandPrix}

\textbf{{\large Задача K - Kill The PSU}} \\
\begin{center}
\includegraphics[width=0.9\textwidth]{OC_Northern/OC_Northern_K1.png}\\ [1cm]
\includegraphics[width=0.9\textwidth]{OC_Northern/OC_Northern_K2.png}\\ [1cm]
\end{center}
\newpage

\textbf{{\large Алгоритм}} \\
{\Huge ???????????????????} \\ 
\\
%\newpage
\textbf{{\large Исходный код}}
\begin{lstlisting}[language=C++]
#include <iomanip>
#include <iostream>
#include <algorithm>
#include <vector>
#include <fstream>

#define LL  long long
#define ULL unsigned long long
using namespace std;

ifstream in("killthepsu.in");
ofstream out("killthepsu.out");

class item {
public:
    string name;
    virtual void message() = 0;
};

class cat1 : public item {
public:
    void message() {
        out << "wake " << name << endl;
    }
    cat1(string n) {
        name = n;
    }
};

class cat2 : public item {
public:
    bool load;
    void message() {
        if (load) {
            out << "unload ";
            load = false;
        }
        else {
            out << "load ";
            load = true;
        }
        out << name << endl;
    }
    cat2(string n) {
        name = n;
        load = false;
    }
};

int reserve = 20;
bool nextIter = false;

class cat3 : public item {
public:
    int power;
    void message() {
        if (reserve) {
            reserve -= 10;
            out << "power fail on " << name << endl;
            return;
        }
        power -= 10;
        if (power > 10) {
            out << "power fail on " << name << endl;
        }
        else if (power <= 10) {
            out << "buy the new PSU" << endl;
            nextIter = true;
        }
    }
    cat3(string n) {
        power = 100;
        name = n;
    }
};

int main() {
    int t;
    in >> t;

    for (int k = 0; k < t; k++) {
        reserve = 20;
        map<string, item *> system;
        int a, b, c, d;
        in >> a >> b >> c >> d;
        string name;
        in.get();

        for (int i = 0; i < a; i++) {
            getline(in, name);
            item *newItem = new cat1(name);
            system[name] = newItem;
        }

        for (int i = 0; i < b; i++) {
            getline(in, name);
            item *newItem = new cat2(name);
            system[name] = newItem;
        }

        for (int i = 0; i < c; i++) {
            getline(in, name);
            item *newItem = new cat3(name);
            system[name] = newItem;
        }

        for (int i = 0; i < d; i++) {
            getline(in, name);
            if (!nextIter) {
                map<string, item *>::iterator current = system.find(name);
                if (current != system.end())
                    current->second->message();
            }
        }
        nextIter = false;
    }

    in.close();
    out.close();

    return 0;
}
\end{lstlisting}

\textbf{{\large Задача M - Мозаика}} \\
\begin{center}
\includegraphics[width=0.9\textwidth]{OC_Northern/OC_Northern_M.png}\\ [1cm]
\end{center}
\newpage

\textbf{{\large Алгоритм}} \\
{\Huge ???????????????????} \\ 
\\
%\newpage
\textbf{{\large Исходный код}}
\begin{lstlisting}[language=C++]
#include <iomanip>
#include <limits>
#include <iostream>
#include <algorithm>

using namespace std;

int main() {
    ifstream in("mosaic.in");
    ofstream out("mosaic.out");

	int N, M;
	in >> M >> N;
	int answ;

	if(N < 3 || M < 3) {
		out << "0" << endl;
		return 0;
	}
	if((N - 2)*(M - 2) % 2 == 0)
		answ = (N - 2)*(M - 2) / 2;
	else
		answ = (N - 2)*(M - 2) / 2 + 1;

	out << answ << endl;

	int k = 2;
	for(int i = 2; i < M; i++) {
		for(int j = k; j < N; j+=2) {
			out << (i - 1)*N + j << " ";
		}
		if(k == 2)
			k = 3;
		else k = 2;
	}

    in.close();
    out.close();

    return 0;
}
\end{lstlisting}

\textbf{{\large Результаты}} \\
\begin{center}
\includegraphics[width=0.95\textwidth]{OC_Northern/OC_Northern_result.png}\\ [1cm]
\end{center}



%----------------------------------------------------------------------------------------
%
%	OpenCup GP of Karelia
%
%----------------------------------------------------------------------------------------
\newpage
\subsection{OpenCup GrandPrix of Karelia}

\textbf{{\large Задача I - Jam}} \\
\begin{center}
\includegraphics[width=0.9\textwidth]{OC_Karelia/OC_Karelia_I.png}\\ [1cm]
\end{center}
\newpage

\textbf{{\large Алгоритм}} \\
{\Huge ???????????????????} \\ 
\\
%\newpage
\textbf{{\large Исходный код}}
\begin{lstlisting}[language=C++]
#include <iostream>
#include <algorithm>

int main() {
    ifstream in("jam.in");

    int T;
    in >> T;

    while (T--) {
        int m, zashli, vishli;
        long ans = 0;
        long buf = 0;
        in >> m;
        for (int i = 0; i < m; i++) {
            in >> zashli >> vishli;
            zashli += buf;
            int diff = zashli - vishli;
            if (diff < 0) {
                ans += diff * (-1);
                buf = 0;
            }
            else {
                buf = diff;
            }
        }
        cout << ans << endl;
    }
    in.close();
    return 0;
}
\end{lstlisting}

\textbf{{\large Задача J - King of Guess}} \\
\begin{center}
\includegraphics[width=0.9\textwidth]{OC_Karelia/OC_Karelia_J.png}\\ [1cm]
\end{center}
\newpage

\textbf{{\large Алгоритм}} \\
{\Huge ???????????????????} \\ 
\\
%\newpage
\textbf{{\large Исходный код}}
\begin{lstlisting}[language=C++]
#include <iostream>
#include <algorithm>

int main() {
    ifstream in("kingofguess.in");
	int X;
	int N;
	int Y;
	int mid;
	int step = 0;

	in >> N >> X >> Y;

	while(true) {
		step++;
		mid = (X + Y)/2;
		if(mid == N) {
			cout << step << endl;
			//cin >> N;
			return 0;
		}
		if(mid > N)
			Y = mid;
		if(mid < N)
			X = mid;
	}
    return 0;
}
\end{lstlisting}

\textbf{{\large Задача K - Lesson}} \\
\begin{center}
\includegraphics[width=0.9\textwidth]{OC_Karelia/OC_Karelia_K.png}\\ [1cm]
\end{center}
\newpage

\textbf{{\large Алгоритм}} \\
{\Huge ???????????????????} \\ 
\\
%\newpage
\textbf{{\large Исходный код}}
\begin{lstlisting}[language=C++]
#include <iostream>
using namespace std;
#define ll long long
#define ull unsigned long long
#define eps 0.00001
ll min(ll a, ll b){return (a<b?a:b);}
ll max(ll a, ll b){return (a>b?a:b);}

int main()
{
    ifstream in("lesson.in");

    long T;
    in >> T;
    in.get();
    const string names[] = {"Sail", "Frigate", "Cruiser", "Dreadnought"};
    while (T--) {
        ll n;
        in >> n;
        in.get();
        char mapA[10][10], mapB[10][10];
        vector< pair<int, int> > A(4, make_pair(0, 0)), B(4, make_pair(0, 0));
        for(ll i=0; i<n; ++i) {
            for (ll j=0; j<n; ++j) {
                mapA[i][j] = in.get();
                if (mapA[i][j] != '.')
                    ++A[mapA[i][j]-'0'-1].first;
            }
            in.get();
        }
        for(ll i=0; i<n; ++i) {
            for (ll j=0; j<n; ++j) {
                mapB[i][j] = in.get();
                if (mapB[i][j] != '.')
                    ++B[mapB[i][j]-'0'-1].first;
            }
            in.get();
        }
        vector< pair<ll, ll> > movesA(n*n), movesB(n*n);
        for (ll i=0; i<n*n; ++i) {
            ll x, y;
            in >> y >> x;
            movesA[i] = make_pair(y - 1, x - 1);
        }
        for (ll i=0; i<n*n; ++i) {
            ll x, y;
            in >> y >> x;
            movesB[i] = make_pair(y - 1, x - 1);
        }
        ll shipsA = 4, shipsB = 4;
        bool win = 0;
        for (ll moveA = 0, moveB = 0; moveA < n*n && moveB < n*n && !win;) {
            ll y, x;
            bool f = 1;
            while (f) {
                f = 0;
                x = movesA[moveA].second;
                y = movesA[moveA].first;
                ++moveA;
                if (mapB[y][x] != '.') {
                    ++B[mapB[y][x]-'0'-1].second;
                    if (B[mapB[y][x]-'0'-1].second == B[mapB[y][x]-'0'-1].first) {
                        cout << "Alice sank Bob's " << names[B[mapB[y][x]-'0'-1].first-1] << '\n';
                        f = 1;
                        --shipsB;
                        if (!shipsB) {
                            cout << "Alice\n";
                            win = 1;
                            break;
                        }
                    }
                }
            }
            if (win) {
                break;
            }
            f = 1;
            while (f) {
                f = 0;
                x = movesB[moveB].second;
                y = movesB[moveB].first;
                ++moveB;
                if (mapA[y][x] != '.') {
                    ++A[mapA[y][x]-'0'-1].second;
                    if (A[mapA[y][x]-'0'-1].second == A[mapA[y][x]-'0'-1].first) {
                        cout << "Bob sank Alice's " << names[A[mapA[y][x]-'0'-1].first-1] << '\n';
                        f = 1;
                        --shipsA;
                        if (!shipsA) {
                            cout << "Bob\n";
                            win = 1;
                            break;
                        }
                    }
                }
            }
            if (win) {
                break;
            }
        }
    }

    in.close();
    return 0;
}
\end{lstlisting}

\textbf{{\large Задача L - Maze}} \\
\begin{center}
\includegraphics[width=0.9\textwidth]{OC_Karelia/OC_Karelia_L1.png}\\ [1cm]
\includegraphics[width=0.9\textwidth]{OC_Karelia/OC_Karelia_L2.png}\\ [1cm]
\end{center}
\newpage

\textbf{{\large Алгоритм}} \\
{\Huge ???????????????????} \\ 
\\
%\newpage
\textbf{{\large Исходный код}}
\begin{lstlisting}[language=C++]
#include <iostream>
using namespace std;
#define ll long long
#define ull unsigned long long
#define eps 0.00001
ll min(ll a, ll b){return (a<b?a:b);}
ll max(ll a, ll b){return (a>b?a:b);}

int main()
{
    ifstream in("maze.in");

    long T;
    in >> T;
    in.get();
    while (T--) {

        char map[210][210];
        for (ll i=0; i<210; ++i) {
            for (ll j=0; j<210; ++j) {
                map[i][j] = 'X';
            }
        }
        ll maxX = 0, maxY = 0, minX = 0, minY = 0, x = 0, y = 0;
        int dir = 0;
        char c;
        while((c = in.get()) != '\n') {
            if (c == 'B') {
                if (dir == 0) dir = 2;
                else if (dir == 1) dir = 3;
                else if (dir == 2) dir = 0;
                else if (dir == 3) dir = 1;
            }
            else if (c == 'R') {
                if (dir == 0) dir = 3;
                else if (dir == 1) dir = 0;
                else if (dir == 2) dir = 1;
                else if (dir == 3) dir = 2;
            }
            else if (c == 'L') {
                if (dir == 0) dir = 1;
                else if (dir == 1) dir = 2;
                else if (dir == 2) dir = 3;
                else if (dir == 3) dir = 0;
            }
            map[x+105][y+105] = '.';
            if (dir == 0) ++x;
            else if (dir == 1) --y;
            else if (dir == 2) --x;
            else ++y;
            if (x > maxX) maxX = x;
            else if(x < minX) minX = x;
            if (y > maxY) maxY = y;
            else if(y < minY) minY = y;
        }
        cout << maxY - minY + 3 << ' ' << maxX - minX + 2 << '\n';
        for (ll i=minY-1; i<maxY+2; ++i) {
            for (ll j=minX; j<maxX+2; ++j) {
                cout.put(map[j+105][i+105]);
            }
            cout.put('\n');
        }
    }

    in.close();
    return 0;
}
\end{lstlisting}

\textbf{{\large Результаты}} \\
\begin{center}
\includegraphics[width=0.95\textwidth]{OC_Karelia/OC_Karelia_result.png}\\ [1cm]
\end{center}



%----------------------------------------------------------------------------------------
%
%	OpenCup GP of Udmurtia
%
%----------------------------------------------------------------------------------------
\newpage
\subsection{OpenCup GrandPrix of Udmurtia}

\textbf{{\large Задача A - Коллекционеры}} \\
\begin{center}
\includegraphics[width=0.9\textwidth]{OC_Udmurtia/OC_Udmurtia_A1.png}\\ [1cm]
\includegraphics[width=0.9\textwidth]{OC_Udmurtia/OC_Udmurtia_A2.png}\\ [1cm]
\end{center}
\newpage

\textbf{{\large Алгоритм}} \\
{\Huge ???????????????????} \\ 
\\
%\newpage
\textbf{{\large Исходный код}}
\begin{lstlisting}[language=C++]
#include <iostream>
#include <algorithm>

typedef enum {
    BLACK,
    WHITE,
    GREEN,
    YELLOW,
    BLUE,
    RED,
    ORANGE,
    PURPLE
} color;

typedef enum {
    DOT,
    ARABIC,
    ROMAN,
} numberType;

typedef struct {
    color col;
    numberType num;
    int value;
} cubeFace;

color parseColor(char col) {
    if (col == 'B') return BLACK;
    else if (col == 'W') return WHITE;
    else if (col == 'G') return GREEN;
    else if (col == 'Y') return YELLOW;
    else if (col == 'S') return BLUE;
    else if (col == 'R') return RED;
    else if (col == 'O') return ORANGE;
    else return PURPLE;
}

int parseDots() {
    int value;
    string dots;
    in >> dots;
    value = (int)dots.size();
    return value;
}

int parseArabic() {
    int value;
    in >> value;
    return value;
}

int parseRoman() {
    string roman;
    in >> roman;
    if (roman == "I") return 1;
    else if (roman == "II") return 2;
    else if (roman == "III") return 3;
    else if (roman == "IV") return 4;
    else if (roman == "V") return 5;
    else return 6;
}

int parseValue(numberType num) {
    if (num == DOT) return parseDots();
    else if (num == ARABIC) return parseArabic();
    else return parseRoman();
}

numberType recognizeNumberType(char num) {
    if (num == '.') return DOT;
    else if (isdigit(num)) return ARABIC;
    else return ROMAN;
}

bool john(cubeFace f1, cubeFace f2, cubeFace f3) {
    if (f1.num == DOT && f1.num == f2.num && f2.num == f3.num)
        return true;
    return false;
}

bool david(cubeFace f1, cubeFace f2, cubeFace f3) {
    if (f1.num != ROMAN && f2.num != ROMAN && f3.num != ROMAN)
        return true;
    return false;
}

bool peter(cubeFace f1, cubeFace f2, cubeFace f3) {
    if (f1.col == WHITE && f1.col == f2.col && f2.col == f3.col)
        return true;
    return false;
}

bool robert(cubeFace f1, cubeFace f2, cubeFace f3) {
    if ((f1.col == BLACK || f1.col == WHITE) &&
        (f2.col == BLACK || f2.col == WHITE) &&
        (f3.col == BLACK || f3.col == WHITE))
        return true;
    return false;
}

bool mark(cubeFace f1, cubeFace f2, cubeFace f3) {
    if (f1.value % 2 == 0 && f1.col != BLACK)
        return false;
    if (f1.value % 2 == 1 && f1.col != WHITE)
        return false;
    if (f2.value % 2 == 0 && f2.col != BLACK)
        return false;
    if (f2.value % 2 == 1 && f2.col != WHITE)
        return false;
    if (f3.value % 2 == 0 && f3.col != BLACK)
        return false;
    if (f3.value % 2 == 1 && f3.col != WHITE)
        return false;
    return true;
}

bool paul(cubeFace f1, cubeFace f2, cubeFace f3) {
    if (f1.value == 2 || f1.value == 3 || f1.value == 5) {
        if (f1.num != ARABIC)
            return false;
    }
    if (f2.value == 2 || f2.value == 3 || f2.value == 5) {
        if (f2.num != ARABIC)
            return false;
    }
    if (f3.value == 2 || f3.value == 3 || f3.value == 5) {
        if (f3.num != ARABIC)
            return false;
    }
    if (f1.num == ARABIC) {
        if (f1.value == 1 || f1.value == 4 || f1.value == 6) {
            return false;
        }
    }
    if (f2.num == ARABIC) {
        if (f2.value == 1 || f2.value == 4 || f2.value == 6) {
            return false;
        }
    }
    if (f3.num == ARABIC) {
        if (f3.value == 1 || f3.value == 4 || f3.value == 6) {
            return false;
        }
    }
    return true;
}

bool patrick(cubeFace f1, cubeFace f2, cubeFace f3) {
    if (f1.col == f2.col && f2.col == f3.col) {
        if (f1.col != BLACK && f1.col != WHITE)
            return true;
    }
    return false;
}

bool jack(cubeFace f1, cubeFace f2, cubeFace f3) {
    if (f1.num == ROMAN && f1.col != YELLOW)
        return false;
    if (f2.num == ROMAN && f2.col != YELLOW)
        return false;
    if (f3.num == ROMAN && f3.col != YELLOW)
        return false;
    return true;
}

bool maxx(cubeFace f1, cubeFace f2, cubeFace f3) {
    if (f1.col != f2.col && f1.col != f3.col && f2.col != f3.col)
        return true;
    return false;
}

bool alex(cubeFace f1, cubeFace f2, cubeFace f3) {
    if (f1.num == f2.num && f1.col != f2.col)
        return false;
    if (f1.num != f2.num && f1.col == f2.col)
        return false;

    if (f1.num == f3.num && f1.col != f3.col)
        return false;
    if (f1.num != f3.num && f1.col == f3.col)
        return false;

    if (f2.num == f3.num && f2.col != f3.col)
        return false;
    if (f2.num != f3.num && f2.col == f3.col)
        return false;

    return true;
}

int main() {

    char col, num;

    // FACE 1
    cubeFace face1;
    col = in.get();
    in.get();

    num = in.get();
    in.unget();

    face1.col = parseColor(col);
    face1.num = recognizeNumberType(num);
    face1.value = parseValue(face1.num);
    in.get();

    // FACE 2
    cubeFace face2;
    col = in.get();
    in.get();

    num = in.get();
    in.unget();

    face2.col = parseColor(col);
    face2.num = recognizeNumberType(num);
    face2.value = parseValue(face2.num);
    in.get();

    // FACE 3
    cubeFace face3;
    col = in.get();
    in.get();

    num = in.get();
    in.unget();

    face3.col = parseColor(col);
    face3.num = recognizeNumberType(num);
    face3.value = parseValue(face3.num);
    in.get();

    if (john(face1, face2, face3))
        out << "John ";
    if (david(face1, face2, face3))
        out << "David ";
    if (peter(face1, face2, face3))
        out << "Peter ";
    if (robert(face1, face2, face3))
        out << "Robert ";
    if (mark(face1, face2, face3))
        out << "Mark ";
    if (paul(face1, face2, face3))
        out << "Paul ";
    if (patrick(face1, face2, face3))
        out << "Patrick ";
    if (jack(face1, face2, face3))
        out << "Jack ";
    if (maxx(face1, face2, face3))
        out << "Max ";
    if (alex(face1, face2, face3))
        out << "Alex ";

    in.close();
    out.close();

    return 0;
}
\end{lstlisting}

\textbf{{\large Задача K - Data Mining}} \\
\begin{center}
\includegraphics[width=0.9\textwidth]{OC_Udmurtia/OC_Udmurtia_K.png}\\ [1cm]
\end{center}
\newpage

\textbf{{\large Алгоритм}} \\
{\Huge ???????????????????} \\ 
\\
%\newpage
\textbf{{\large Исходный код}}
\begin{lstlisting}[language=C++]
#include <iostream>
#include <algorithm>

int main() {
    ifstream in("input.txt");
    ofstream out("output.txt");

    long T;
    in >> T;
    while (T--) {
        long dev, runs, execParallel, execCurrent;
        in >> dev >> runs >> execCurrent >> execParallel;
        long keepTime, rewriteTime;
        keepTime = runs * execCurrent;
        rewriteTime = dev + runs * execParallel;
        if (keepTime < rewriteTime) {
            out << "Keep" << endl;
        }
        else if (keepTime > rewriteTime) {
            out << "Rewrite" << endl;
        }
        else {
            out << "Flip a Coin" << endl;
        }
    }

    in.close();
    out.close();

    return 0;
}
\end{lstlisting}

\textbf{{\large Задача L - Performance}} \\
\begin{center}
\includegraphics[width=0.9\textwidth]{OC_Udmurtia/OC_Udmurtia_L.png}\\ [1cm]
\end{center}
\newpage

\textbf{{\large Алгоритм}} \\
{\Huge ???????????????????} \\ 
\\
%\newpage
\textbf{{\large Исходный код}}
\begin{lstlisting}[language=C++]
#include <iostream>
#include <algorithm>

typedef struct {
    long gear;
    long a, b, c;
} eq;

int main() {
    ifstream in("input.txt");
    ofstream out("output.txt");

    int T;
    in >> T;
    while (T--) {
        long maxGear = 0;
        double maxValue = -1000000;
        int n;
        in >> n;
        for (int i = 1; i <= n; i++) {
            long a, b, c;
            in >> a >> b >> c;
            double max = (-1) * (((b * b) - (4 * (-a) * c)) / (4 * (-a)));
            if (max > maxValue) {
                maxValue = max;
                maxGear = i;
            }
        }
        out << maxGear << endl;
    }

    in.close();
    out.close();

    return 0;
}
\end{lstlisting}

\textbf{{\large Задача M - Tic-tac-toe}} \\
\begin{center}
\includegraphics[width=0.9\textwidth]{OC_Udmurtia/OC_Udmurtia_M.png}\\ [1cm]
\end{center}
\newpage

\textbf{{\large Алгоритм}} \\
{\Huge ???????????????????} \\ 
\\
%\newpage
\textbf{{\large Исходный код}}
\begin{lstlisting}[language=C++]
#include <iostream>
using namespace std;
#define ll long long
#define ull unsigned long long
#define eps 0.00001
ll min(ll a, ll b){return (a<b?a:b);}
ll max(ll a, ll b){return (a>b?a:b);}

struct Point{
    int x, y;
};

int check(char map[3][3]) {
    int winner = 2;
    for (int i=0; i<3; ++i) {
        bool oFH = 1, xFH = 1, oFV = 1, xFV = 1;
        for (int j=0; j<3; ++j) {
            if (map[i][j] != 'o') {
                oFH = 0;
            }
            if (map[i][j] != 'x') {
                xFH = 0;
            }
            if (map[j][i] != 'o') {
                oFV = 0;
            }
            if (map[j][i] != 'x') {
                xFV = 0;
            }
        }
        if (oFH || oFV) {
            winner = 1;
        }
        else if (xFH || xFV) {
            winner = 0;
        }
    }
    bool oFD1 = 1, xFD1 = 1, oFD2 = 1, xFD2 = 1;
    for (int i=0; i<3; ++i) {
        if (map[i][i] != 'o') {
            oFD1 = 0;
        }
        if (map[i][i] != 'x') {
            xFD1 = 0;
        }
        if (map[i][2-i] != 'o') {
            oFD2 = 0;
        }
        if (map[i][2-i] != 'x') {
            xFD2 = 0;
        }
    }

    if (oFD1 || oFD2) {
        winner = 1;
    }
    else if (xFD1 || xFD2) {
        winner = 0;
    }
    return winner;
}
int main()
{
    ifstream in("input.txt");
    ofstream out("output.txt");

    Point freeCell[2];
    int n = 0;
    int xN = 0, oN = 0;
    char map[3][3];
    for (int i=0; i<3; ++i) {
        for (int j=0; j<3; ++j) {
            map[i][j] = in.get();
            if (map[i][j] == '.') {
                freeCell[n].x = j;
                freeCell[n].y = i;
                ++n;
            }
            if (map[i][j] == 'x') {
                ++xN;
            }
            else if (map[i][j] == 'o') {
                ++oN;
            }
        }
        in.get();
    }
    char firstPlayer, secondPlayer;
    if (oN < xN) {
        firstPlayer = 'o';
        secondPlayer = 'x';
    }
    else {
        firstPlayer = 'x';
        secondPlayer = 'o';
    }
    map[freeCell[0].y][freeCell[0].x] = firstPlayer;
    int winner = check(map);
    float pX = 0, pO = 0;
    if (firstPlayer == 'o' && winner == 1 || firstPlayer == 'x' && winner == 0) {
        pO += 1;
    }
    else {
        map[freeCell[1].y][freeCell[1].x] = secondPlayer;
        winner = check(map);
        if (firstPlayer == 'o' && winner == 0 || firstPlayer == 'x' && winner == 1) {
            pX += 1;
        }
    }

    map[freeCell[0].y][freeCell[0].x] = '.';
    map[freeCell[1].y][freeCell[1].x] = firstPlayer;
    winner = check(map);
    if (firstPlayer == 'o' && winner == 1 || firstPlayer == 'x' && winner == 0) {
        pO += 1;
    }
    else {
        map[freeCell[0].y][freeCell[0].x] = secondPlayer;
        winner = check(map);
        if (firstPlayer == 'o' && winner == 0 || firstPlayer == 'x' && winner == 1) {
            pX += 1;
        }
    }

    if (pO > pX) {
        if (firstPlayer == 'o')
            out << "o";
        else
            out << "x";
    }
    else if (pO < pX) {
        if (secondPlayer == 'x')
            out << "x";
        else
            out << "o";
    }
    else {
        out << "tie";
    }
    in.close();
    out.close();
    return 0;
}
\end{lstlisting}

\textbf{{\large Результаты}} \\
\begin{center}
\includegraphics[width=0.95\textwidth]{OC_Udmurtia/OC_Udmurtia_result.png}\\ [1cm]
\end{center}



%----------------------------------------------------------------------------------------
%
%	OpenCup GP of China
%
%----------------------------------------------------------------------------------------
\newpage
\subsection{OpenCup GrandPrix of China}

\textbf{{\large Задача N - Ordered Sequences}} \\
\begin{center}
\includegraphics[width=0.9\textwidth]{OC_China/OC_China_N.png}\\ [1cm]
\end{center}
\newpage

\textbf{{\large Алгоритм}} \\
{\Huge ???????????????????} \\ 
\\
%\newpage
\textbf{{\large Исходный код}}
\begin{lstlisting}[language=C++]
#include <iostream>
using namespace std;
#define ll long long
#define ull unsigned long long
#define eps 0.00001
ll min(ll a, ll b){return (a<b?a:b);}
ll max(ll a, ll b){return (a>b?a:b);}

int main()
{
    int T;
    cin >> T;
    cin.get();
    while (T--) {
        ull a = 0;
        char c;
        while ((c=cin.get())!='\n') {
            a <<= 1;
            if (c == 'B') {
                a |= 1;
            }
        }
        cout << a << '\n';
    }
    return 0;
}
\end{lstlisting}

\textbf{{\large Результаты}} \\
\begin{center}
\includegraphics[width=0.95\textwidth]{OC_China/OC_China_result.png}\\ [1cm]
\end{center}



%----------------------------------------------------------------------------------------
%
%	OpenCup GP of Tatarstan
%
%----------------------------------------------------------------------------------------
\newpage
\subsection{OpenCup GrandPrix of Tatarstan}

\textbf{{\large Задача M - The Dress}} \\
\begin{center}
\includegraphics[width=0.9\textwidth]{OC_Tatarstan/OC_Tatarstan_M.png}\\ [1cm]
\end{center}
\newpage

\textbf{{\large Алгоритм}} \\
{\Huge ???????????????????} \\ 
\\
%\newpage
\textbf{{\large Исходный код}}
\begin{lstlisting}[language=C++]
#include <iostream>
#include <algorithm>

int main(){
    int n;
    cin >> n;
    int humans = 0;
    int aliens = 0;
    int fuckers = 0;
    int total = n;
    string answer;
    cin.get();

    while (n--) {
        getline(cin, answer);
        if (answer.find("blue") != -1 && answer.find("black") != -1) {
            humans++;
        }
        else if (answer.find("gold") != -1 && answer.find("white") != -1) {
            aliens++;
        }
        else {
            fuckers++;
        }
    }

    cout << setprecision(10) << (double)humans / total * 100 << endl;
    cout << setprecision(10) << (double)aliens / total * 100 << endl;
    cout << setprecision(10) << (double)fuckers / total * 100 << endl;
    return 0;
}
\end{lstlisting}

\textbf{{\large Результаты}} \\
\begin{center}
\includegraphics[width=0.95\textwidth]{OC_Tatarstan/OC_Tatarstan_result.png}\\ [1cm]
\end{center}



%----------------------------------------------------------------------------------------
%
%	OpenCup GP of America
%
%----------------------------------------------------------------------------------------
\newpage
\subsection{OpenCup GrandPrix of America}

\textbf{{\large Задача J - Zig Zag Nametag}} \\
\begin{center}
\includegraphics[width=0.9\textwidth]{OC_America/OC_America_J.png}\\ [1cm]
\end{center}
\newpage

\textbf{{\large Алгоритм}} \\
{\Huge ???????????????????} \\ 
\\
%\newpage
\textbf{{\large Исходный код}}
\begin{lstlisting}[language=C++]
#include <iostream>
#include <algorithm>

int main(){
    long k;
    cin >> k;
    long length = (k - 1) / 25 + 2;
    
    string s = "";
    for (long i = 0; i < length; i++) {
        if (i  % 2)
            s += "z";
        else
            s += "a";
    }
    
    if (length > 2) {
        s[1] = (char)('n' + (k - 25 * (length - 2)) / 2);
        long last = s.size() - 1;
        if (s[last] == 'z' && k % 2 == 0) {
            s[last] = 'y';
        }
        else if (s[last] == 'a' && k % 2 == 1) {
            s[last] = 'b';
        }
    }
    else {
        s[0] = 'a';
        s[1] = 'a' + k;
    }
    cout << s << endl;
    return 0;
}
\end{lstlisting}

\textbf{{\large Задача K - Knight Jumps}} \\
\begin{center}
\includegraphics[width=0.9\textwidth]{OC_America/OC_America_K.png}\\ [1cm]
\end{center}
\newpage

\textbf{{\large Алгоритм}} \\
{\Huge ???????????????????} \\ 
\\
%\newpage
\textbf{{\large Исходный код}}
\begin{lstlisting}[language=C++]
#include <iostream>
#include <algorithm>
#include <vector>

int main(){
    int n, m;
    cin >> n >> m;

    int x_num = 0;
    int k_num = 0;
    
    char symb;
    
    vector<bool> used(n * m, false);
    vector<vector<int> > g(n * m, vector<int>());
    
    for (int i = 0; i < n; i++) {
        for (int j = 0; j < m; j++) {
            cin >> symb;
            //used[i][j] = false;
            if (symb == '#') {
                used[i * m + j] = true;
            }
            else if (symb == 'K') {
                k_num = i * m + j;
                used[i * m + j] = true;
            }
            else if (symb == 'X') {
                x_num = i * m + j;
            }
        }
    }
    
    for (int i = 0; i < n; i++) {
        for (int j = 0; j < m; j++) {
            int to = i * m + j;
            
            if (i - 2 >= 0 && j - 1 >= 0)
                g[to].push_back((i - 2) * m + (j - 1));
            if (i - 2 >= 0 && j + 1 < m)
                g[to].push_back((i - 2) * m + (j + 1));
            
            if (i - 1 >= 0 && j - 2 >= 0)
                g[to].push_back((i - 1) * m + (j - 2));
            if (i + 1 < n && j - 2 >= 0)
                g[to].push_back((i + 1) * m + (j - 2));
            
            if (i + 2 < n && j - 1 >= 0)
                g[to].push_back((i + 2) * m + (j - 1));
            if (i + 2 < n && j + 1 < m)
                g[to].push_back((i + 2) * m + (j + 1));
            
            if (i - 1 >= 0 && j + 2 < m)
                g[to].push_back((i - 1) * m + (j + 2));
            if (i + 1 < n && j + 2 < m)
                g[to].push_back((i + 1) * m + (j + 2));
        }
    }
    
    queue<int> ways;
    used[k_num] = true;
    ways.push(k_num);

    vector<int> d(n * m), p(n * m);
    p[k_num] = -1;
    
    while (!ways.empty()) {
        int v = ways.front();
        ways.pop();
        for (size_t i = 0; i < g[v].size(); ++i) {
            int to = g[v][i];
            if (!used[to]) {
                used[to] = true;
                ways.push(to);
                d[to] = d[v] + 1;
                p[to] = v;
            }
        }
    }
    
    if (!used[x_num])
        cout << "-1" << endl;
    else {
        vector<int> path;
        for (int v = x_num; v!= -1; v = p[v])
            path.push_back (v);
        cout << path.size() - 1 << endl;
    }    
    return 0;
}
\end{lstlisting}

\textbf{{\large Задача M - Multiple Tom N}} \\
\begin{center}
\includegraphics[width=0.9\textwidth]{OC_America/OC_America_M1.png}\\ [1cm]
\includegraphics[width=0.9\textwidth]{OC_America/OC_America_M2.png}\\ [1cm]
\end{center}
\newpage

\textbf{{\large Алгоритм}} \\
{\Huge ???????????????????} \\ 
\\
%\newpage
\textbf{{\large Исходный код}}
\begin{lstlisting}[language=C++]
#include <iostream>
#include <algorithm>
#include <vector>

#define LL  long long
#define ULL unsigned long long

int main() {
	ULL n;
	cin >> n;

	vector <ULL> v1(n);
	vector <ULL> v2(n);
    cin.get();
	for (ULL i = 0; i < n; i++) {
        ULL a = 0;
        char c;
        while((c=cin.get())!='\n') {
            a = a*26+(c-'A');
        }
		v1[i] = a;
	}

	for (ULL i = 0; i < n; i++) {
        ULL a = 0;
        char c;
        while((c=cin.get())!='\n') {
            a = a*26+(c-'A');
        }
        v2[i] = a;
	}
	
	ULL count = 0;
	ULL sum1 = 0;
	ULL sum2 = 0;

	for(ULL i = 0; i < n; i++) {
		if(v1[i] == v2[i] && count == 0) {
			cout << "1" << endl;
			continue;
		}
		else {
			sum1 += v1[i];
			sum2 += v2[i];
			count++;
		}
		
		if(sum1 == sum2) {
			cout << count << endl;
			count = 0;
			sum1 = 0;
			sum2 = 0;
		}
	}

    return 0;
}
\end{lstlisting}

\textbf{{\large Задача N - New Contest Director}} \\
\begin{center}
\includegraphics[width=0.9\textwidth]{OC_America/OC_America_N.png}\\ [1cm]
\end{center}
\newpage

\textbf{{\large Алгоритм}} \\
{\Huge ???????????????????} \\ 
\\
%\newpage
\textbf{{\large Исходный код}}
\begin{lstlisting}[language=Perl]
#!/usr/bin/perl
<>;
my %h;
my $max = 0;
while($a=<>) {
	++$h{$a};
	if ($max < $h{$a}) {
		$max = $h{$a};
	}
}
my @m = %h;
my @r;
while(($a = shift @m, $b = shift @m)) {
	if ($b == $max) {
		$a =~ s/\n//;
		push @r, $a;
	}
}
print join "\n", sort @r;
\end{lstlisting}


\textbf{{\large Результаты}} \\
\begin{center}
\includegraphics[width=0.95\textwidth]{OC_America/OC_America_result.png}\\ [1cm]
\end{center}



%----------------------------------------------------------------------------------------
%
%	Vekua Cup team
%
%----------------------------------------------------------------------------------------
\newpage
\subsection{Vekua Cup 2015 Командный этап}

Так как соревнование проводилось в центре 1С, исходные коды программ не доступны. \\

\textbf{{\large Результаты}} \\
\begin{center}
\includegraphics[width=0.95\textwidth]{Vekua_team/Vekua_team_result.png}\\ [1cm]
\end{center}



%----------------------------------------------------------------------------------------
%
%	OpenCup GP of Ural
%
%----------------------------------------------------------------------------------------
\newpage
\subsection{OpenCup GrandPrix of Ural}

\textbf{{\large Задача C - Древние ПСП}} \\
\begin{center}
\includegraphics[width=0.9\textwidth]{OC_Ural/OC_Ural_C.png}\\ [1cm]
\end{center}
\newpage

\textbf{{\large Алгоритм}} \\
{\Huge ???????????????????} \\ 

%\newpage
\textbf{{\large Исходный код}}
\begin{lstlisting}[language=Perl]
#!/usr/bin/perl
$\ = "\n";
sub func {
	my $s = shift;
	my $a = int ((sqrt(1+8*$s)-1)/2);
	my $n = $s - ($a+1)*$a/2;
	my $r = "";
	$r = func($n) if $n != 0;
	"(". $r .")". "()" x ($a-1) ;
}
my $a = <>;
print func $a;
\end{lstlisting}

\textbf{{\large Задача F - Фокус}} \\
\begin{center}
\includegraphics[width=0.9\textwidth]{OC_Ural/OC_Ural_F.png}\\ [1cm]
\end{center}
\newpage

\textbf{{\large Алгоритм}} \\
{\Huge ???????????????????} \\ 

%\newpage
\textbf{{\large Исходный код}}
\begin{lstlisting}[language=Perl]
#!/usr/bin/perl
$a = <>;
if ($a == 0) {
	print "-1\n";
	exit;
}
@m = split "", $a;
map{$s+=$_}@m;
while ($s >= 9) {
	$r .= 9;
	$s -= 9;
}
$r .= $s if $s != 0;
if ($r == $a) {
	$r =~ s/^(\d)(\d+)(\d)$/$3$2$1/;
}
if ($r == $a) {
	$r =~ s/^(\d)(\d*)$/"1".($1-1).$2/e;
}
if ($r > 10**9) {
	print "-1\n";
	exit;
}
print $r;
\end{lstlisting}

\textbf{{\large Результаты}} \\
\begin{center}
\includegraphics[width=0.95\textwidth]{OC_Ural/OC_Ural_result.png}\\ [1cm]
\end{center}





%----------------------------------------------------------------------------------------
%
%	ЛИЧНЫЕ КОНТЕСТЫ
%
%----------------------------------------------------------------------------------------

\newpage
\section{Журнал по личным контестам Макарова Н.А.}

%----------------------------------------------------------------------------------------
%
%	Codeforces 267
%
%----------------------------------------------------------------------------------------

\newpage
\subsection{Codeforces Round 267 Div 2}

\textbf{{\large Задача A - Юра и заселение}} \\
\begin{center}
\includegraphics[width=0.9\textwidth]{C_267/C_267_A.png}\\ [1cm]
\end{center}
\textbf{{\large Алгоритм}} \\
Задача на простую реализацию. Достаточно пройти по всем $p_i$ и $q_i$, проверить выполнение условия $p_i + 2 \leq q_i$ и если оно выполняется, то инкрементировать счетчик. Решается за $O(n)$. \\

\textbf{{\large Исходный код}}
\begin{lstlisting}[language=C]
#include <iostream>
#include <cmath>

using namespace std;

int main() {
    int n = 0;
    cin >> n;
    int p = 0, q = 0;
    int res = 0;
    for (int i = 0; i < n; i++) {
        cin >> p >> q;
        if (p + 2 <= q)
            res++;
    }
    cout << res << endl;
    return 0;
}
\end{lstlisting}

\newpage
\textbf{{\large Задача B - Федя и новая игра}} \\
\begin{center}
\includegraphics[width=0.9\textwidth]{C_267/C_267_B.png}\\ [1cm]
\end{center}
\textbf{{\large Алгоритм}} \\
В этой задаче нужно побитово сравнивать число Феди и числа других игроков и инкрементировать счетчик, если количество различных бит меньше $k$. Задача решается за $O(nb)$, где $b$ -- максимальная разрядность числа.\\ 
\\
\textbf{{\large Исходный код}}
\begin{lstlisting}[language=C]
#include <iostream>
#include <cmath>
#include <vector>

using namespace std;

int main() {
    long n, m, k;
    cin >> n >> m >> k;
    vector<long> gamers(m + 1);
    
    for (int i = 0; i < m + 1; i++)
        cin >> gamers[i];
    
    long fedya = gamers[m];
    long friends = 0;
    long diffBits = 0;
    
    for (int i = 0; i < m; i++) {
        diffBits = 0;
        for (int j = 0; j < 20; j++) {
            if (((fedya >> j) & 1) != ((gamers[i] >> j) & 1))
                diffBits++;
            if (diffBits > k)
                break;
        }
        if (diffBits <= k)
            friends++;
    }
    
    cout << friends << endl;
    
    return 0;
}
\end{lstlisting}

\textbf{{\large Результаты}} \\
\begin{center}
\includegraphics[width=0.95\textwidth]{C_267/C_267_result.png}\\ [1cm]
\end{center}



%----------------------------------------------------------------------------------------
%
%	Codeforces 268
%
%----------------------------------------------------------------------------------------

\newpage
\subsection{Codeforces Round 268 Div 2}

\textbf{{\large Задача A - I Wanna Be the Guy}} \\
\begin{center}
\includegraphics[width=0.9\textwidth]{C_268/C_268_A.png}\\ [1cm]
\end{center}
\textbf{{\large Алгоритм}} \\
В этой задаче нужно проверить, дает ли объединение чисел $a_1, a_2, ... , a_p$ и $a_1, a_2, ... , a_q$ множество чисел $1, 2, ... , n$. Решается за $O(n)$.\\

\textbf{{\large Исходный код}}
\begin{lstlisting}[language=C]
#include <iostream>
#include <vector>

using namespace std;

int main() {
    int n, p, q, num;
    cin >> n;
    long sum = ((1 + n) * n) / 2;
    vector<bool> v(n + 1, false);
    cin >> p;
    for (int i = 0; i < p; i++) {
        cin >> num;
        sum -= num;
        v[num] = true;
    }
    cin >> q;
    for (int i = 0; i < q; i++) {
        cin >> num;
        if (!v[num]) sum -= num;
        v[num] = true;
    }
    if (sum)
        cout << "Oh, my keyboard!" << endl;
    else
        cout << "I become the guy." << endl;
    return 0;
}
\end{lstlisting}

\newpage
\textbf{{\large Задача B - Онлайн чат}} \\
\begin{center}
\includegraphics[width=0.9\textwidth]{C_268/C_268_B.png}\\ [1cm]
\end{center}
\textbf{{\large Алгоритм}} \\
Создадим вектор всех моментов времени и пометим все моменты, когда не спит Little Z. Далее будем смотреть время Little X и инкрементировать ответ, если у Little Z помечено время, в которое не спит Little X. Решается за $O(rq)$.\\ 
\\
\textbf{{\large Исходный код}}
\begin{lstlisting}[language=C]
#include <iostream>
#include <cmath>
#include <vector>
#include <algorithm>

using namespace std;

int main() {

    int p, q, l, r;
    cin >> p >> q >> l >> r;
    
    vector< pair<int, int> > timesX(q);
    vector<bool> timesZ(1002, false);
    pair<int, int> z;
    
    for (int i = 0; i < p; i++) {
        cin >> z.first >> z.second;
        for (int j = z.first; j <= z.second; j++)
            timesZ[j] = true;
    }
    
    for (int i = 0; i < q; i++)
        cin >> timesX[i].first >> timesX[i].second;
    
    bool marker = false;
    int answer = 0;
    
    for (int i = l; i <= r; i++) {
        marker = false;
        for (int j = 0; j < q; j++) {
            for (int k = timesX[j].first + i; k <= timesX[j].second + i; k++) {
                if (k > 1000) {
                    marker = true;
                    break;
                }
                if (timesZ[k] == true) {
                    marker = true;
                    answer++;
                }
                if (marker) break;
            }
            if (marker) break;
        }
    }
    
    cout << answer << endl;
    
    return 0;
}
\end{lstlisting}

\textbf{{\large Результаты}} \\
\begin{center}
\includegraphics[width=0.95\textwidth]{C_268/C_268_result.png}\\ [1cm]
\end{center}



%----------------------------------------------------------------------------------------
%
%	Codeforces Отборочный контест СГАУ на четвертьфинал ACM ICPC
%
%----------------------------------------------------------------------------------------

\newpage
\subsection{Codeforces Отборочный контест СГАУ на четвертьфинал ACM-ICPC}

\textbf{{\large Задача D - Игрушечные солдатики}} \\
\begin{center}
\includegraphics[width=0.9\textwidth]{CT_SGAU/CT_SGAU_D.png}\\ [1cm]
\end{center}
\textbf{{\large Алгоритм}} \\
Запомним цвета всех солдатиков. Затем будем идти по всем солдатикам, перекрашивать его в новый цвет и проверять, совпадают ли цвета всех солдатиков после перекраски. Если найдено состояние, когда все цвета совпадают -- нужно завершить проход и вывести ответ. Работает за $O(nm)$.\\ 
\\
\newpage
\textbf{{\large Исходный код}}
\begin{lstlisting}[language=C]
#include <iostream>
#include <vector>
#include <algorithm>
#define ll unsigned long long

using namespace std;

int main() {
ll n, m;
    ll remColor = 0;
    ll currentSoldier, currentColor, answer = 0;
    bool sameColors = true;
    cin >> n;
    vector<ll> soldiers(n + 1);
    for (ll i = 1; i <= n; i++) {
        cin >> soldiers[i];
        if (i == 1) remColor = soldiers[i];
        else if (soldiers[i] != remColor) sameColors = false;
    }
    if (sameColors) {
        cout << "0" << endl;
        return 0;
    }
    cin >> m;
    sameColors = true;
    bool flag = true;
    
    for (ll i = 0; i < m; i++) {
        sameColors = false;
        flag = true;
        cin >> currentSoldier >> currentColor;
        soldiers[currentSoldier] = currentColor;
        if (i == 0) remColor = currentColor;
        if (currentColor == remColor) {
            for (ll j = 1; j <= n; j++) {
                if (soldiers[j] != remColor) {
                    flag = false;
                    break;
                }
            }
            if (flag) sameColors = true;
        }
        if (sameColors) {
            answer = i + 1;
            break;
        }
        remColor = currentColor;
    }
    
    if (sameColors) {
        cout << answer << endl;
    }
    else {
        cout << "-1" << endl;
    }
    
	return 0;
}
\end{lstlisting}

\newpage
\textbf{{\large Задача F - Два конверта}} \\
\begin{center}
\includegraphics[width=0.9\textwidth]{CT_SGAU/CT_SGAU_F.png}\\ [1cm]
\end{center}
\textbf{{\large Алгоритм}} \\
Задача на теорию вероятности. Если в открытом конверте денег больше чем $b$, то не нужно менять выбор, иначе нужно поменять, потому что вероятность того, что в другом конверте в 2 раза меньше денег такая же, как и вероятность того, что в другом конверте в 2 раза больше денег. Сложность $O(1)$.\\ 
\\
\textbf{{\large Исходный код}}
\begin{lstlisting}[language=C]
#include <iostream>

using namespace std;

int main() {
    long long b, c;
    cin >> b >> b >> c;
    if (c > b) cout << "Stay with this envelope" << endl;
    else cout << "Take another envelope" << endl;
    return 0;
}
\end{lstlisting}

\newpage
\textbf{{\large Задача G - Задача о размене монет}} \\
\begin{center}
\includegraphics[width=0.9\textwidth]{CT_SGAU/CT_SGAU_G.png}\\ [1cm]
\end{center}
\textbf{{\large Алгоритм}} \\
Для начала нужно найти, где находится самая дорогая монета, которая не дороже необходимой суммы $s$. После этого будем набирать сумму последовательно с наибольшей монеты. Работает за $O(n)$. \\ 
\\
\textbf{{\large Исходный код}}
\begin{lstlisting}[language=C]
#include <iostream>
#include <vector>
using namespace std;
int main() {
    unsigned long long n, s, startIndex, coinsNeeded = 0;
    cin >> n >> s;
    vector<unsigned long long> k(n + 1);
    k[0] = 1;
    startIndex = n;
    for (unsigned long long i = 1; i <= n; i++) {
        cin >> k[i];
        if (k[i] * k[i - 1] > s) {
            startIndex = i - 1;
            break;
        }
        k[i] *= k[i - 1];
    }
    for (unsigned long long i = startIndex; ; i--) {
        coinsNeeded += s / k[i];
        s %= k[i];
        if (s == 0 || i == 0)
            break;
    }
    cout << coinsNeeded << endl;
    return 0;
}
\end{lstlisting}

\newpage
\textbf{{\large Результаты}} \\
\begin{center}
\includegraphics[width=0.95\textwidth]{CT_SGAU/CT_SGAU_result.png}\\ [1cm]
\end{center}



%----------------------------------------------------------------------------------------
%
%	Codeforces 270
%
%----------------------------------------------------------------------------------------

\newpage
\subsection{Codeforces Round 270 Div 2}

\textbf{{\large Задача A - Уроки дизайна задач: учимся у математики}} \\
\begin{center}
\includegraphics[width=0.9\textwidth]{C_270/C_270_A.png}\\ [1cm]
\end{center}
\textbf{{\large Алгоритм}} \\
Нужно найти такие числа $a$ и $b$, чтобы они не были простыми, а $a + b = n$. Для этого положим $a = 4$, а $b = n - 4$. Далее будем проверять, являются ли числа простыми. Если хоть одно из них простое, то инкрементирум $a$ и декрементируем $b$. В конце концов таким образом найдутся искомые $a$ и $b$. Сложность $O(n)$.\\

\textbf{{\large Исходный код}}
\begin{lstlisting}[language=C]
#include <iostream>
bool prime(long long n) {
    long long sq = sqrt(n);
    for (long long i = 2; i <= sq; i++)
        if (n % i == 0)
            return false;
    return true;
}
using namespace std;
int main() {
    long long n;
    cin >> n;   
    long long a = 4;
    long long b = n - a;
    bool ok = false;
    while (!ok) {
        if (!prime(a) && !prime(b))
            ok = true;
        else { a++; b--; }
    } 
    cout << a << " " << b << endl;
    return 0;
}\end{lstlisting}

\newpage
\textbf{{\large Задача B - Уроки дизайна задач: учимся у жизни}} \\
\begin{center}
\includegraphics[width=0.9\textwidth]{C_270/C_270_B.png}\\ [1cm]
\end{center}
\textbf{{\large Алгоритм}} \\
Чтобы доставить всех людей на нужные этажи за наименьшее количество времени, отсортируем всех по номеру этажа и будем доставлять их порциями по вместимости лифта. Сложность $O(nlog(n))$.\\ 
\\
\textbf{{\large Исходный код}}
\begin{lstlisting}[language=C]
#include <iostream>
#include <vector>
#include <algorithm>
using namespace std;

int main() {
    long n, k;
    cin >> n >> k;    
    vector<long> floor(n);
    for (long i = 0; i < n; i++) {
        cin >> floor[i];
    }
    sort(floor.begin(), floor.end(), greater<long>());
    
    long answer = 0;
    
    for (long long i = 0; i < n; i += k) {
        answer += 2 * (floor[i] - 1);
    }
        
    cout << answer << endl;
    
    return 0;
}
\end{lstlisting}

\textbf{{\large Результаты}} \\
\begin{center}
\includegraphics[width=0.95\textwidth]{C_270/C_270_result.png}\\ [1cm]
\end{center}



%----------------------------------------------------------------------------------------
%
%	Codeforces 273
%
%----------------------------------------------------------------------------------------

\newpage
\subsection{Codeforces Round 273 Div 2}

\textbf{{\large Задача A - Начальная ставка}} \\
\begin{center}
\includegraphics[width=0.9\textwidth]{C_273/C_273_A.png}\\ [1cm]
\end{center}
\textbf{{\large Алгоритм}} \\
В этой задаче достаточно заметить, что суммарное количество монет должно без остатка делиться на 5. Иначе исход игры невозможен. Сложность $O(1)$.\\

\textbf{{\large Исходный код}}
\begin{lstlisting}[language=C]
#include <iostream>

#define LL  long long
#define ULL unsigned long long
using namespace std;

int main() {
    LL c1, c2, c3, c4, c5;
    cin >> c1 >> c2 >> c3 >> c4 >> c5;
    
    c1 = c1 + c2 + c3 + c4 + c5;
    
    if (c1 % 5 == 0 && c1) {
        cout << c1 / 5 << endl;
    }
    else cout << "-1" << endl;

    return 0;
}
\end{lstlisting}

\newpage
\textbf{{\large Задача B - Случайные команды}} \\
\begin{center}
\includegraphics[width=0.9\textwidth]{C_273/C_273_B.png}\\ [1cm]
\end{center}
\textbf{{\large Алгоритм}} \\
Задача на комбинаторику. Минимальное количество получается если всех разделить на команды поровну, а максимальное если в одну из команд поместить максимально возможное количество участников. Сложность $O(nlog(1))$.\\ 
\\
\textbf{{\large Исходный код}}
\begin{lstlisting}[language=C]
#include <iostream>

#define LL  long long
using namespace std;

LL fun(LL n) {
    return (n * n - n) / 2;
}

int main() {
    LL n, m;
    cin >> n >> m;
    
    LL min, max;
    
    LL k = n - (m - 1);
    max = fun(k);
    
    LL g = n / m;
    if (n % m == 0) {
        min = m * fun(g);
    }
    else {
        min = (m - (n % m)) * fun(g);
        min += (n % m) * fun(g + 1);
    }
    
    cout << min << " " << max << endl;
    
    return 0;
}
\end{lstlisting}

\textbf{{\large Результаты}} \\
\begin{center}
\includegraphics[width=0.95\textwidth]{C_273/C_273_result.png}\\ [1cm]
\end{center}



%----------------------------------------------------------------------------------------
%
%	Codeforces 274
%
%----------------------------------------------------------------------------------------

\newpage
\subsection{Codeforces Round 274 Div 2}

\textbf{{\large Задача B - Башни}} \\
\begin{center}
\includegraphics[width=0.9\textwidth]{C_274/C_274_B.png}\\ [1cm]
\end{center}
\textbf{{\large Алгоритм}} \\
Отсортируем башни по высоте и будем перекладывать кубик с самой высокой на самую низкую. После этого нужно снова отсортировать новые высоты. Это нужно проделать $k$ раз. Сложность алгоритма $O(knlog(n))$\\

\textbf{{\large Исходный код}}
\begin{lstlisting}[language=C]
#include <iostream>
using namespace std;

typedef struct {
    int n, h;
} tow;

bool cmp(const tow &a, const tow &b) {
    return a.h > b.h;
}

int main() {
    int n, k;
    cin >> n >> k;
    vector<tow> t(n);
    vector<pair<int, int>> actions;
    for (int i = 0; i < n; i++) {
        cin >> t[i].h;
        t[i].n = i + 1;
    }
    pair<int, int> move;
    int min = inf;
    vector<int> states(k + 1);
    sort(t.begin(), t.end(), cmp);
    states[0] = t[0].h - t[n - 1].h;
    for (int i = 1; i <= k; i++) {
        t[0].h--;
        t[n - 1].h++;
        move.first = t[0].n;
        move.second = t[n - 1].n;
        actions.push_back(move);
        sort(t.begin(), t.end(), cmp);
        states[i] = t[0].h - t[n - 1].h;
    }
    int minPos = -1;
    for (int i = 0; i < k + 1; i++) {
        if (states[i] < min) {
            min = states[i];
            minPos = i;
        }
    }
    cout << min << " " << minPos << endl;
    for (int i = 0; i < minPos; i++) cout << actions[i].first << " " << actions[i].second << endl;
    return 0;
}
\end{lstlisting}

\textbf{{\large Задача C - Экзамены}} \\
\begin{center}
\includegraphics[width=0.9\textwidth]{C_274/C_274_C.png}\\ [1cm]
\end{center}
\textbf{{\large Алгоритм}} \\
Для решения этой задачи нужно отсортировать экзамены сначала по дате сдачи по расписанию, а затем устойчиво отсортировать по датам досрочной сдачи. Потом нужно пройти по отсортированным датам и если у текущего экзамена досрочная дата раньше чем досрочная предыдущего, то текущий нужно сдавать по расписанию. Сложность алгоритма $O(nlog(n))$.\\

\textbf{{\large Исходный код}}
\begin{lstlisting}[language=C]
#include <iostream>
using namespace std;

bool cmp_pair_first(const pair<LL, LL> &a, const pair<LL, LL> &b) {
    return a.first < b.first;
}

bool cmp_pair_second(const pair<LL, LL> &a, const pair<LL, LL> &b) {
    return (a.second < b.second) && (a.first == b.first);
}

int main() {
    int n;
    cin >> n;
    vector<pair<LL, LL>> e(n);
    LL maxFirst = -1, maxSecond = -1;
    for (int i = 0; i < n; i++) {
        cin >> e[i].first >> e[i].second;
        if (e[i].first  > maxFirst)  maxFirst  = e[i].first;
        if (e[i].second > maxSecond) maxSecond = e[i].second;
    }
    sort(e.begin(), e.end(), cmp_pair_first);
    stable_sort(e.begin(), e.end(), cmp_pair_second);
    for (int i = 1; i < n; i++) {
        if (e[i].second < e[i - 1].second) {
            e[i].second = e[i].first;
        }
    }
    cout << e[n - 1].second << endl;
    return 0;
}
\end{lstlisting}

\textbf{{\large Результаты}} \\
\begin{center}
\includegraphics[width=0.95\textwidth]{C_274/C_274_result.png}\\ [1cm]
\end{center}



%----------------------------------------------------------------------------------------
%
%	Codeforces 275
%
%----------------------------------------------------------------------------------------

\newpage
\subsection{Codeforces Round 275 Div 2}

\textbf{{\large Задача C - Разнообразная перестановка}} \\
\begin{center}
\includegraphics[width=0.9\textwidth]{C_275/C_275_C.png}\\ [1cm]
\end{center}
\textbf{{\large Алгоритм}} \\
Получить искомую перестановку можно записывая поочередно $b = 1$ и $e = n$. После записи $b$ нужно инкрементировать, а $e$ декрементировать. В результате будет получена искомая перестановка. Сложность алгоритма $O(n)$.\\

\textbf{{\large Исходный код}}
\begin{lstlisting}[language=C]
#include <iostream>
#include <vector>
using namespace std;

int main() {
    long n, k;
    cin >> n >> k;
    vector<long> v(n, 0);
    long b = 1, e = n;
    for (long i = 0; i < k; i++) {
        if (i % 2 == 0) v[i] = b++;
        else v[i] = e--;
    }
    for (long i = k; i < n; i++) {
        if (k % 2 == 1) v[i] = b++;
        else v[i] = e--;
    }
    for (auto n : v) cout << n << " ";
    cout << endl;
    return 0;
}
\end{lstlisting}

\textbf{{\large Результаты}} \\
\begin{center}
\includegraphics[width=0.95\textwidth]{C_275/C_275_result.png}\\ [1cm]
\end{center}




%----------------------------------------------------------------------------------------
%
%	VK Cup Квалификация
%
%----------------------------------------------------------------------------------------

\newpage
\subsection{VK Cup 2015 Квалификация}

\textbf{{\large Задача A - Репосты}} \\
\begin{center}
\includegraphics[width=0.9\textwidth]{VK_Qual/VK_Qual_A.png}\\ [1cm]
\end{center}
\textbf{{\large Алгоритм}} \\
Так как требуется найти наибольшую цепочку, то задачу можно решить поиском в глубину и запомнить наибольшую длину пути. Сложность $O(N+M)$.\\

\textbf{{\large Исходный код}}
\begin{lstlisting}[language=C]
#include <iostream>

#define LL  long long
#define ULL unsigned long long
using namespace std;

vector<vector<int>> g;
vector<bool> used;
int depth = 0;
int maxDepth = 0;

void dfs(int v) {
    used[v] = true;
    depth++;
    for (vector<int>::iterator i = g[v].begin(); i != g[v].end(); ++i) {
        if (!used[*i]) {
            dfs (*i);
        }
    }
    if (depth > maxDepth) {
        maxDepth = depth;
    }
    depth--;
}

int main(int argc, const char * argv[]) {
    int n;
    cin >> n;
    
    map<string, int> names;
    vector<pair<string, string> > log(n);
    string left, right, dummy;
    int nameNumber = 0;
    
    for (int i = 0; i < n; i++) {
        cin >> left >> dummy >> right;
        transform(left.begin(), left.end(), left.begin(), ::tolower);
        transform(right.begin(), right.end(), right.begin(), ::tolower);
        if (names.find(left) == names.end()) {
            names[left] = nameNumber++;
        }
        if (names.find(right) == names.end()) {
            names[right] = nameNumber++;
        }
        log[i].first = left;
        log[i].second = right;
    }
    
    g.resize(names.size(), vector<int>());
    used.resize(g.size());
    map<string, int>::iterator from, to;
    
    for (size_t i = 0; i < log.size(); i++) {
        from = names.find(log[i].second);
        to = names.find(log[i].first);
        g[from->second].push_back(to->second);
    }

    int start = names.find("polycarp")->second;
    dfs(start);
    cout << maxDepth << endl;
    
    return 0;
}
\end{lstlisting}

\textbf{{\large Задача B - Фото на память}} \\
\begin{center}
\includegraphics[width=0.9\textwidth]{VK_Qual/VK_Qual_B.png}\\ [1cm]
\end{center}
\textbf{{\large Алгоритм}} \\
В этой задаче нужно знать высоту самого высокого и второго по высоте. Это можно сделать с помощью $k$-й порядковой статистики. А потом нужно посчитать количество пикселей. Можно заметить, что высота фотографии всегда будет равна высоте самого высокого из друзей, и один раз высоте второго по высоте.\\

\textbf{{\large Исходный код}}
\begin{lstlisting}[language=C]
#include <iostream>

#define LL  long long
#define ULL unsigned long long
using namespace std;

int main() {
	int temp = 0;

	LL n;
	cin >> n;

	vector <int> width(n);
	vector <int> height(n);
	vector <int> height_2(n);

	int max_h = 0, premax_h = 0;
	LL sum_w = 0;

	for (int i = 0; i < n; i++) {

		cin >> width[i] >> height[i];
		height_2[i] = height[i];

		sum_w += width[i];
	}

	nth_element(height_2.begin(), height_2.end() - 1, height_2.end());
	max_h = height_2[n - 1];
	nth_element(height_2.begin(), height_2.end() - 2, height_2.end());
	premax_h = height_2[n - 2];
	
	LL temp_sum, temp_max;
	for (int i = 0; i < n; i++) {
		temp_sum = sum_w - width[i];
		if (height[i] == max_h)
			temp_max = premax_h;
		else temp_max = max_h;
		cout << temp_sum * temp_max << " ";
	}
	return 0;
}
\end{lstlisting}

\textbf{{\large Результаты}} \\
\begin{center}
\includegraphics[width=0.95\textwidth]{VK_Qual/VK_Qual_result.png}\\ [1cm]
\end{center}




%----------------------------------------------------------------------------------------
%
%	VK Cup Раунд 1
%
%----------------------------------------------------------------------------------------

\newpage
\subsection{ VK Cup 2015 - Уайлд-кард раунд 1}

\textbf{{\large Задача A - Квадратное уравнение}} \\
\begin{center}
\includegraphics[width=0.9\textwidth]{VK_1/VK_1_A.png}\\ [1cm]
\end{center}
\textbf{{\large Алгоритм}} \\
Сложность этой задачи состоит в том, что написать ее нужно на эзотерическом языке $Picat$.\\

\textbf{{\large Исходный код}}
\begin{lstlisting}[language=C]
 main =>
  A = read_int(),
  B = read_int(),
  C = read_int(),
  
  D = (B * B) - (4 * A * C),
  
  X1 = ((-1)* B + sqrt(D)) / (2 *A ),
  
  X2 = ((-1)* B - sqrt(D)) / (2 * A),
  
  
  if(X1 < X2) then
      println(X1),
      println(X2)
      end,

  if(X1 > X2) then
      println(X2),
      println(X1)
      end,
  if(X1 == X2) then
      println(X2)
      end.
\end{lstlisting}

\textbf{{\large Результаты}} \\
\begin{center}
\includegraphics[width=0.95\textwidth]{VK_1/VK_1_result.png}\\ [1cm]
\end{center}



%----------------------------------------------------------------------------------------
%
%	Vekua Cup personal
%
%----------------------------------------------------------------------------------------
\newpage
\subsection{Vekua Cup 2015 Личный этап}

Так как соревнование проводилось в центре 1С, исходные коды программ не доступны. \\

\textbf{{\large Результаты}} \\
\begin{center}
\includegraphics[width=0.95\textwidth]{Vekua_personal/Vekua_personal_result.png}\\ [1cm]
\end{center}


%----------------------------------------------------------------------------------------
%
%	RCC Квалификация
%
%----------------------------------------------------------------------------------------

\newpage
\subsection{Mail.ru Russian Code Cup 2015 Квалификация}

\textbf{{\large Задача A - Покупка велосипеда}} \\
\begin{center}
\includegraphics[width=0.9\textwidth]{RCC/RCC_A.png}\\ [1cm]
\end{center}
\textbf{{\large Алгоритм}} \\
В этой задаче по числам $a$, $b$ и $c$ нужно определить, существуют ли такие $a_1$ и $b_1$, что $a_1 \leq a$, $b_1 \leq b$ и $a_1 + 2b_1 = c$. Для этого достаточно проверить, что $a + 2b \geq c$, а если $c$ -- нечетное, то проверить условие $a \geq 1$. Сложность $O(1)$.\\

\textbf{{\large Исходный код}}
\begin{lstlisting}[language=C]
#include <iostream>
using namespace std;
int main() {
	long t, a, b, c;
	cin >> t;
	while (t--) {
		cin >> a >> b >> c;
		if (c % 2) {
			if (a + 2*b >= c && a >= 1) {
				cout << "YES";
			} else {
				cout << "NO";
			}
		} else {
			if (a + 2*b >= c && a >= 1) {
				cout << "YES";
			} else {
				cout << "NO";
			}
		}
		cout << endl;
	}
	return 0;
}
\end{lstlisting}

\textbf{{\large Результаты}} \\
\begin{center}
\includegraphics[width=0.95\textwidth]{RCC/RCC_result.png}\\ [1cm]
\end{center}






\newpage
\section{Журнал по личным контестам Якименко А.В.}


\end{document}